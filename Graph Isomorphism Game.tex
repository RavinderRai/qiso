\documentclass[12pt]{article}
\usepackage{amsthm}
\usepackage[utf8]{inputenc}
\usepackage[english]{babel}
\usepackage{amsmath}
\usepackage{amssymb}
\usepackage{centernot}
\usepackage[export]{adjustbox}
\usepackage{hyperref}
\usepackage{cleveref}

\hypersetup{
    colorlinks=true,
    linkcolor=blue,
    filecolor=magenta,      
    urlcolor=cyan,
}

\usepackage{subcaption} 


\newtheorem{thm}{Theorem}[section]
\newtheorem{lem}[thm]{Lemma}
\newtheorem{proposition}[thm]{Proposition}
\newtheorem{defn}[thm]{Definition}

\title{\textbf{Classical and Quantum Graph Isomorphisms}}
\author{Ravinder Rai}
\date{\today}

\begin{document} 

\maketitle


\begin{abstract}
Quantum computers are continually coming closer to becoming a reality, but the question of what we can do with them is still a mystery. In this work we look at graph isomorphism as it pertains to quantum computers. Graph isomorphism itself is a well known concept, but when looking at two graphs being isomorphic on a quantum computer, a new phenomenon occurs, which we define as quantum graph isomorphism. To see how this works, we introduce a two player game that redefines graph isomorphism, and then use quantum resources to paint the picture of quantum graph isomorphism. 
\end{abstract}

\newpage
\section{Preliminaries} \label{prelim}
Some important basic concepts are presented here. First, a graph is defined as a collection of nodes, with line-connections between them. These nodes are called vertices, and connections are called edges. A bijection is a function that maps vertices from one graph to another, that is one-to-one and onto.

The path between any two vertices $x$ and $y$ in a graph is a sequence of edges which connect a list of vertices starting with either $x$ or $y$ and ending with the other. The edges in a graph may be associated with weights, and the sum of the weights of the edges in a path is called the distance of that path. The graphs that we consider here have no specific weights, so we say the weight of each edge is $1$, so distance of a path would give a positive integer equal to the number of edges in the path. The shortest path between two vertices in a graph is a path with the least amount of edges possible between those two vertices, and so we denote the shortest distance, $\delta$, as the distance of the shortest path. 


Now we define a permutation matrix. A permutation is simply a rearrangement of some number of objects. Using a given permutation, one can construct a permutation matrix, which is constructed as follows: Given a permutation, $\pi$, of size n, (for simplicity, let the objects being permuted be the natural numbers, i.e. $1, 2, ...., n$), the permutation matrix will be an $n \times n$ matrix, where each column will be a standard basis vector. The standard basis vector in each column will be given by $e_{\pi(i)}$ where $i$ is the $i'th$ row, and there will be a $1$ in the $j'th$ column if and only if $\pi(i) = j$. Note that $P^T = P^{-1}$ and $PP^T = I$. 

Another important definition is the adjacency matrix of a graph. Consider some graph G, then $A_G$ is the adjacency matrix of G, and is constructed as follows: the $i'th$ rows and $j'th$ columns both represent the vertices and the element $A_{G_{i, j}}$ is a one if the $i'th$ vertex shares and edge with the $j'th$ vertex, and zero otherwise. Note that adjacency matrices only apply to finite graphs.

The next matrix definition is the \hypertarget{stochastic}{doubly stochastic matrix}, which is a matrix whose elements are non-negative real numbers, and whose rows and columns sum to $1$.

Now, a constraint satisfaction problem (csp) is a something that computers can solve, and is modelled by a three element vector, $(X, D, C)$. In a csp, $X$ is defined to be a finite set of indeterminate variables, and in the case for graphs, it is the set of in determinates associated with the set of vertices. $D$ is a set of domains, such that the elements in $D$ would be domains like the natural numbers, complex numbers, and others. $C$ is the constraint, which is usually some relation defined between the variables in $X$, so naturally in the case of graph theory, this constraint/relation could be whether there is an edge or not between any two vertices in $X$. A csp is said to have a solution is there exists a function that takes a subset of $X$ and maps it to some domain in $D$ such that no constraints in $C$ are violated, and all variables in $X$ are included.

There is an important game that is discussed in this paper, and it relates to some two graphs G and H. The game works as follows: consider two players, Alice and Bob, and two graphs G and H with the same number of vertices. Now consider a referee who gives Alice and Bob a vertex from the two different graphs G and H. Then Alice and Bob both respond with another vertex from either of the two graphs, and are said to win if they meet the following conditions:
\begin{description}
  \item[$\bullet$] $x_A \in V(G) \Leftrightarrow y_A \in V(H)$ and  $x_B \in V(G)\Leftrightarrow y_B \in V(H), $ where $x_A, x_B, y_A, y_B \in V(G) \cup V(H)$
  \item[$\bullet$] $rel(g_A, g_B) = rel(h_A, h_B)$
\end{description}
Vertices $g_A$ and $g_B$ are the vertices among $x_A, x_B, y_A, y_B$ that belong to graph G, and similarly for $h_A$, and $h_B$.
If there is a strategy to this game such that Alice and Bob always win, then the graphs are said to be isomorphic (more below), and the strategy is a winning or perfect strategy. 

Next, some basic quantum theory is presented to ensure understanding of the definition of quantum strategies. A matrix $M \in \mathbb{C}^{d \times d}$ has a conjugate transpose denoted by $M^\dag$. An operator $H$ is Hermitian if $H = {H^{\dag}}$, and the set of these $d \times d$ Hermitian operators are denoted by $\mathcal{H}^d$. A Matrix $M \in \mathcal{H}^d$ is called a positive semidefinite (psd) if $\psi^\dag M \psi \geq 0$ $\forall \psi \in \mathbb{C}^d$ and the set of psd matrices is $\mathcal{H}^d_+$. For two matrices $M, N \in \mathcal{H}^d_+$, the inner product is defined as $\langle M, N\rangle = Tr(M^\dag N)$ and orthogonality then follows: $MN = 0$ iff $\langle M, N \rangle = 0$ $\mathcal{H}^d$. If a matrix E satisfies $E = E^2 = E^\dag$, then we call it a projector. Note that if a family of projectors $\{E_i\}_i$ satisfies $\sum_iE_i = 1$, then $E_iE_j = 0$ $\forall i \neq j$. Let $\mathbb{M}_n(d)$ be a set of $n \times n$ block matrices with elements in $\mathcal{H}^d$. The $(i, j)$-block/element of a block matrix $\mathcal{P} \in \mathbb{M}_n(d)$ is denoted $\mathcal{P}_{i, j}$.

Finally, we introduce the basic formalisms of quantum states. Normally, Hilbert spaces are used to describe quantum systems, and a quantum state is just a vector, usually denoted by $\psi$. The joint system of two state spaces $\mathbb{C}^{d_1}$, $\mathbb{C}^{d_2}$ is given by the tensor product $\mathbb{C}^{d_1} \otimes \mathbb{C}^{d_2}$. If a state $\psi$ of $\mathbb{C}^{d_1} \otimes \mathbb{C}^{d_2}$ can be written as $\psi = \psi_1 \otimes \psi_2$, then we say $\psi$ is separable, otherwise it is known as an entangled state.

The normal form of a quantum state is $\psi = \alpha \begin{bmatrix}
1 \\
0 
\end{bmatrix}
+ \beta
\begin{bmatrix}
0 \\
1
\end{bmatrix}
$, which is called a superposition. We obtain information from a quantum system by measuring it, specifically, if we were to measure $\psi$ as above, we would have a probability of $|\alpha|^2$ of measuring the state $\begin{bmatrix}
1 \\
0 
\end{bmatrix}$ and $|\beta|^2$ of measuring the state $\begin{bmatrix}
0 \\
1 
\end{bmatrix}$. Moreover, any quantum state can be written as a Schmidt decomposition: $\psi = \sum_{i=1}^d \lambda_i \alpha_i \otimes \beta_i$, where $\lambda_i \geq 0$ for all $1 \leq i \leq d$, and the $\{\alpha_i| 1 \leq i \leq d\}$ and $\{\beta_i| 1 \leq i \leq d\}$ are orthonormal bases of $\mathbb{C}^d$. We call these bases the Schmidt bases of $\psi$ and the $\lambda_i$ are called the Schmidt coefficients. We say that $\psi$ has full Schmidt rank if its Schmidt coefficients are all positive. Now, we call a state maximally entangled if all of its Schmidt coefficients are the same, and the canonical maximally entangled state is $\psi_d = \frac{1}{\sqrt{d}} \sum_{i=1}^d e_i \otimes e_i$, where $e_i$ is the $i^{th}$ standard basis vector. Also, an important fact which will be used later is that $\psi^{\dag}_d (A \otimes B) \psi_d = \frac{1}{d} Tr(AB^T)$ for all $A, B \in \mathbb{C}^{dxd}$.

Furthermore, a Positive Operator-Valued Measure (POVM) $\mathcal{M}$ is a family of Hermitian psd matrices $\{ M_i \in \mathcal{H}^d_+; 1 \leq i \leq m \}$ such that $\sum_{i=1}M_i = I$ (where m is some integer), and given some state $\psi$, the probability of outcome $i$ occurring is $\psi^\dag M_i \psi$. Finally, we will call $\mathcal{M}$ projective if all of its elements are projectors.


\section{Classical Graph Isomorphism} \label{classical}
Graph isomorphism is said to occur when two graphs are the essentially the same, aside from labels and manipulations, like rotations or rearrangements of vertices. Intuitively, this means that if two graphs are isomorphic, but perhaps it does not immediately seem obvious, then one will be able to manipulate one of the graphs via rotations or rearrangements of vertices so that it is identical to the other one (again, aside from the vertices' labels). There are multiple formal definitions of graph isomorphism, presented below.



The first and most well known definition of graph isomorphism is defined by a function that maps vertices in one graph to the other, while preserving the edge connections. This function is an edge-preserving bijective function.
\begin{defn}[Graph Isomorphism - Edge-preserving Bijection]
Two graphs G and H are said to be \textit{isomorphic} if there exists a bijection $f:V(G) \to V(H)$ such that for any $x, y \in V(G)$, $x \sim y$ iff $f(x) \sim f(y)$. If G and H are isomorphic, then we write $G \cong H$.
\end{defn}

The second definition of isomorphism follows from the shortest paths of a graph. This definition says that if we have a bijective function from one graph to another, then they are isomorphic if the shortest path of any two vertices of the domain of the function is equal to the shortest paths of their image vertices in the co-domain. In the following theorem, we state this definition of isomorphism and prove its equivalence to the first definition. The proof comes from \cite{shortest}.

\begin{thm}[Graph Isomorphism Equivalence - Shortest Paths]
\label{shortest}
Let G and H be graphs. Then $G \cong H$ iff there exists a bijection $f:V(G) \to V(H)$ such that for any $x, y \in V(G)$, $\delta(x, y) = \delta(f(x), f(y))$.
\end{thm}

\begin{proof}
Since both definitions have a bijective function, $f:V(G) \to V(H)$, it is sufficient to only show that the edge-preserving property is equivalent to the shortest path property. So, consider two isomorphic graphs, G and H.

$\Rightarrow$) If $x_1 \sim x_2$ in G, then $f(x_1) \sim f(x_2)$ in H (and vice versa). So $(x_1, x_2, ..., x_n)$ is a path in G iff $(f(x_1), f(x_2), ..., f(x_n))$ is a path in H. It follows that $(x_1, x_2, ..., x_n)$ is the shortest path in G iff $(f(x_1), f(x_2), ..., f(x_n))$ is the shortest path in H.

$\Leftarrow$) If $x_1 \sim x_2$ in G, then $\delta(x_1, x_2) = 1$. But by our assumption, $\delta(x_1, x_2) = \delta(f(x_1), f(x_2)) = 1$, so $f(x_1) \sim f(x_2)$. Hence, if $x_1 \sim x_2$, then $f(x_1) \sim f(x_2)$ (and vice versa).
\end{proof}

The third definition of isomorphism revolves around permutations, and says that two graphs are isomorphic if their adjacency matrices can be written as $PA_GP = A_H$, where A and B are adjacency matrices for some isomorphic graphs. The following theorem states the definition and proves its equivalence to the first definition.

\begin{thm}[Graph Isomorphism - Permutation Matrix]
\label{permutation}
Let G and H be two finite graphs. Then $G \cong H$ iff there exists a permutation matrix $P$ such that $PA_GP^T = A_H$.
\end{thm}

\begin{proof}
Let G and H be two isomorphic graphs, and $A_G$ and $A_H$ be their adjacency matrices respectively. Also, let the vertices be labelled by the natural numbers, i.e. 1, 2, ..., n.

$\Rightarrow$) Given an edge-preserving bijective function, $f:V(G) \to V(H)$, construct a matrix $P = [e_{f(1)}, e_{f(2)}, ..., e_{f(n)}]$, where $e_{f(i)}$ is a basis vector with 1 in the $f(i)^{th}$ spot. Now, observe that the $i^{th}$ row in $A_G$ is the $f(i)^{th}$ in $PA_G$ and the $j^{th}$ column in $A_G$ is the $f(j)^{th}$ column in $A_GP^{\dag}$. It follows that $PA_{G_{i, j}}P^{\dag} = A_{G_{f(i), j}}P^{\dag} = A_{G_{f(i), f(j)}}$. Since $f(i), f(j) \in V(H)$ and $A_{G_{i, j}} = 1$ if and only if $i \sim j$, then $A_{G_{f(i), f(j)}} = 1$ if and only if $f(i) \sim f(j)$. Let $i' = f(i)$ and $j' = f(j) \in H$. Since we know if $i \sim j \in G \Leftrightarrow i' \sim j' \in H$, then $A_{G_{f(i), f(j)}} = A{H_{i', j'}}$ $\forall i, j \in 1, 2, ..., n$. Thus $PA_GP^{\dag} = H$.

$\Leftarrow$) Let $j \in 1, 2, ..., n$ and let $e_i$ be a column basis vector with a 1 in the $j^{th}$ row. By our assumption, we have $PA_GP^{\dag} = H$, so then $Pe_j = e_k$, for some $1 \leq k \leq n$. Now construct a function: $f(j) = k$. Observe that if $f(j) = f(j')$, then this implies that $Pe_j = Pe_{j'} \Rightarrow P^{\dag}Pe_j = P^{\dag}Pe_{j'} \Rightarrow e_j = e_{j'}$. Thus $f$ is injective. Let $k \in 1, 2, ..., n$. Then $j \in 1, 2, ..., n$ such that $f(j) = k$ which implies that $Pe_j = e_k$. Now, for some $l$, compute $P^{\dag}e_k = e_l$. Let $j = l$, then $Pe_j = Pe_l \Rightarrow Pe_j = PP^{\dag}e_k = e_k$. Thus $f$ is surjective. Now let $j, k \in 1, 2, ..., n$ with $j \sim k$ and $j' = f(j), k' = f(k)$. Now, recall that the $i^{th}$ row in $A_G$ is the $f(i)^{th}$ in $PA_G$ and the $j^{th}$ column in $A_G$ is the $f(j)^{th}$ column in $A_GP^{\dag}$. Then $PA_{G_{j, k}}P^{\dag} = A_{G_{f(j), f(k)}} = 1$, so $A_{G_{f(j), f(k)}} = A_{H_{j', k'}} = 1 \Rightarrow f(j) \sim f(k)$. Therefore, $f$ is an edge-preserving bijective function.

\end{proof}

The fourth form of graph isomorphism comes from defining a constraint satisfaction problem (csp) as above. A csp can take the form of what is called an Integer Quadratic Programming (IQP) problem. In the case of graph isomorphism, we can formulate this IQP problem in the following way: Given two graphs G and H, there exists real scalar variables $x_{gh}$ for each $g \in V(G)$ and $h \in V(H)$ such that the below conditions are satisfied, in which case G and H would be isomorphic. 

\begin{subequations}
\label{IQP}
\begin{align}
x_{gh}^2 = x_{gh} \text{  } \forall g \in V(G), h \in V(H)
\\
\sum_{h' \in V(H)} x_{gh'} = \sum_{g' \in V(G)} x_{g'h} = 1 \text{  } \forall g \in V(G), \text{  } h \in V(H)
\\
x_{gh}x_{g'h'} = 0 \text{ if } rel(g, g') \neq rel(h, h')
\end{align}
\end{subequations}
One can see how this IQP is equivalent to a csp by taking the variables $x_{gh}$ as the set $X$ of indeterminate variables, $D$ will be a set of domains (in the proof below the natural numbers will suffice), and the above conditions are the constraints. The following theorem will again show the equivalence of this isomorphism definition to the first.

\begin{thm}[Graph Isomorphism - Constraint Satisfaction Problem]
Let G and H be graphs. Then $G \cong H$ iff there exists a solution to the IQP above.
\end{thm}

\begin{proof}
$\Rightarrow$) If $G \cong H$, then we have an edge-preserving bijective function, say $f:V(G) \to V(H)$. Now, denote indeterminate variables $x_{gh}$ for each $g \in V(G)$, $h \in V(H)$. Define a new function: $F:x_{gh} \to \mathbb{R}$ such that 
\centerline{
$F(x_{gh}) = \begin{cases}
1 \text{ } if \text{ } f(g) \text{ } = \text{ } h\\
0 \text{ } if \text{ } f(g) \text{ } \neq \text{ } h
\end{cases}$}
Now consider $g, g' \in V(G)$, $h, h' \in V(H)$. Suppose $h = f(g)$ and $h' = f(g')$ such that $rel(g, g') \neq rel(h, h')$. Then $x_{gh} = 1$ and $x_{g'h'} = 1$, which implies that $x_{gh}x_{g'h'} = 1$. If $g \sim g'$, then $f(g) = h \nsim h' = f(g')$, but this contradicts the edge-preserving bijection, which says that $g \sim g'$ iff $f(g) \sim f(g')$, so either $x_{gh} = 0$ such that $f(g) \neq h$ or $x_{g'h'} = 0$ such that $f(g') \neq h'$, which implies that $x_{gh}x_{g'h'} = 0$ (and similar argument shows the same thing if $g \nsim g'$). If instead $g = g'$, then $f(g) = h \neq h' = f(g')$, but this contradicts the fact $f$ is a one-to-one function, so again either $x_{gh} = 0$ such that $f(g) \neq h$ or $x_{g'h'} = 0$ such that $f(g') \neq    h'$, which implies that $x_{gh}x_{g'h'} = 0$. Hence condition 1c is satisfied.

Since $x_{gh}$ is only ever $0$, or $1$, this automatically satisfies conditions 1a. Since $f$ is a bijective function, for a fixed $g \in V(G)$, $x_{gh} = 1$ for only one $h \in V(H)$, so $\sum_{h' \in V(H)} x_{gh'} = 1$, and similarly for $\sum_{g' \in V(G)} x_{g'h}$. Thus condition 1b is satisfied,  which means there is a solution to the IQP problem and therefore csp.


$\Leftarrow$) Define $f:V(G) \to V(H)$ such that 

\centerline{
$x_{gh} = \begin{cases}
1 \text{ } if \text{ } f(g) \text{ } = \text{ } h
\\
0 \text{ } if \text{ } f(g) \text{ } \neq \text{ } h
\end{cases}$}

Consider some fixed $h \in V(H)$. Then since $\sum_{g' \in V(G)} x_{g'h} = 1$, and all terms are either $1$ or $0$, exactly one $g' \in V(G)$ must exist such that $h = f(g')$, and thus $f$ is bijective. If $g \sim g'$, then $\exists h, h' \in V(H)$ such that $x_{gh}x_{g'h'} = 1$, which implies that $rel(g, g') = rel(h, h')$, and so $f(g) \sim f(g')$. And if $h \sim h'$, then $\exists g, g' \in V(G)$ such that $f(g) = h$, $f(g') = h'$, and so $x_{gh}x_{g'h'} = 1$, which implies that $rel(g, g') = rel(h, h')$. It follows that $g \sim g'$. Hence, $g \sim g'$ iff $f(g) \sim f(g')$. Therefore there is an edge-preserving bijection between graphs G and H.
\end{proof}


The final isomorphism definition revolves around the game from above, and is particularly important for our following discussions. The graphs are said to be isomorphic if there is a winning strategy for Alice and Bob, and the equivalence of this definition to the the first isomorphism definition is below.

\begin{thm}[Graph Isomorphism - Game]
\label{game}
Let G and H be two graphs. Then $G \cong H$ iff there exists a winning strategy for the G-H game.
\end{thm}

\begin{proof}
$\Rightarrow$) Assume G and H are isomorphic such that there is an edge preserving bijection $f: x \in V(G) \to y \in V(H)$. Then there must exist a winning strategy for the graph isomorphism game. 

First show that the first condition holds: Let $x \in V(G)$, then $\phi(x) = y$, for some $y \in V(H)$. Similarly, let $x \in V(H)$, then $f(x) = y$, for some $y \in V(G)$

Next to show the second condition holds. Let $x_1, x_2 \in V(G)$ such that $x_1 \neq x_2$ and $x_1 \sim x_2$. Then since $f$ is an edge preserving bijection, 
\begin{equation}
rel(x_1, x_2) = rel(f(x_1), f(x_2)) = rel(y_1, y_2)
\end{equation}
for $y_1, y_2 \in V(H)$. If $x_1 = x_2$, then 
\begin{equation}
rel(x_1, x_1) = rel(f(x_1), f(x_1)) = rel(y_1, y_1)
\end{equation}
Similarly, if $x_1, x_2 \in V(H)$ such that $x_1 \neq x_2$ and $x_1 \sim x_2$, then  
\begin{equation}
rel(x_1, x_2) = rel(f(x_1), f(x_2)) = rel(y_1, y_2)
\end{equation}
for $y_1, y_2 \in V(G)$. And if $x_1 = x_2$, then 
\begin{equation}
rel(x_1, x_1) = rel(f(x_1), f(x_1)) = rel(y_1, y_1)
\end{equation}
Thus the winning conditions hold, so phi is a winning strategy.

$\Leftarrow$) Conversely, assume that there is a winning strategy, $f: y \in V(G)\cup V(H) \to x \in V(G) \cup V(H)$, that satisfies the above conditions. 

Let $x_A, x_B \in V(G)$, then $y_A, y_B \in V(H)$ from the first winning condition above, where $f(x_A) = y_A$ and $f(x_B) = y_B$. Note that also $rel(f(x_A), f(x_B)) = rel(x_A, x_B)$. If $y_A = y_B$, then $f(x_A) = f(x_B)$, so that $rel(f(x_A), f(x_A)) = rel(x_A, x_B)$, which means that $x_A = x_B$. So $f$ is injective. 

If $y_A \in V(H)$, then $f(y_A) = x_A$, and $f(x_A) = y_A'$. But $rel(v, v) = rel(y_A, y_A')$, thus $y_A = y_A'$, and so $f$ is surjective. 

Now, let $x_1, x_2 \in V(G)$ and $x_1 \sim x_2$. Then $f(x_1), f(x_2) \in V(H)$, but $rel(x_1, x_2) = rel(f(x_1), f(x_2)$. So $f$ is a  graph homomorphism, and thus this winning strategy function, $f$, is an isomorphism.

\end{proof}

\section{Quantum Graph Isomorphism}
We now define quantum graph isomorphism. 

Given two graphs G and H, there are POVM's, $\mathcal{A}_x = \{A_{xy}: y \in V(G) \cup V(H)\}$ and $\mathcal{B}_x = \{B_{xy}: y \in V(G) \cup V(H)\}$ for each $x \in V(G) \cup V(H)$ associated with Alice and Bob, respectively.
Alice will again receive some $x_A \in V(G) \cup V(H)$ and will perform a measurement $\mathcal{A}_{x_A}$ to get an outcome $y_A \in V(G) \cup V(H)$.
The measurement is performed with some state $\psi$, and the probability of Alice and Bob responding with $y_A$ and $y_B$ given $x_A$ and $x_B$ is called a quantum correlation: 
\begin{equation}
p(y_A, y_B|x_A, x_B) = \psi^\dag(A_{x_Ay_A}\otimes B_{x_By_B})\psi
\end{equation}
If this quantum correlation is zero whenever conditions 1 or 2 fail, then this is a winning strategy.

\begin{defn}[Quantum Graph Isomorphism]
\label{qgame}
If there exists a perfect quantum strategy for the above game, then the graphs $G$ and $H$ are quantum isomorphic, denoted by $G \cong_q H$.
\end{defn}

It is well known that $\psi$ can be assumed to be of full Schmidt rank. A useful mathematical tool that will be used later on is the map $vec: \mathbb{C}^{d_1 x d_2} \longrightarrow \mathbb{C}^{d_1} \otimes \mathbb{C}^{d_2}$, which takes the matrix $uv^{\dag}$ to $u\otimes \bar{v}$. Also, define the inverse as $mat: \mathbb{C}^{d_1} \otimes \mathbb{C}^{d_2} \longrightarrow \mathbb{C}^{d_1 x d_2}$. This map is an isometry, which means that $vec(A)^{\dag}vec(B) = Tr(A^{\dag}B)$, and if we set $\rho = mat(\psi)$, we can show that: 
\begin{equation}
\label{isometry}
\psi^{\dag} (A \otimes B) \psi = vec(\rho)^{\dag} (A \otimes B) vec(\rho) = vec(\rho)^{\dag} vec(A \rho B^T) = Tr(\rho^{\dag} A\rho B^T)
\end{equation}
where we used the identity $vec(AXB^T) = (A \otimes B)vec(X)$.

This is the first definition of quantum isomorphism, and it is the quantum analogue of the game definition provided in section \ref{classical}. There are a few definitions of quantum isomorphism already known, but finding quantum analogue definitions to the more well-known classical definitions in section \ref{classical} is of particular interest as it is still an open problem. However, we now give a special name to the non-local games that we deal with in this paper.

\begin{defn}
A non-local game is called synchronous if the question set $X$ and answer set $Y$ are the same for all players, and also: $V(y, y'|x, x) = 0$ for all $x \in X$ and $y \neq y' \in Y$.
\end{defn}

It follows that the graph isomorphism game is synchronous since both Alive and Bob have the same question and answer set, namely $V(G) \cup V(H)$, and $p(y, y'|x, x) = 0$ whenever $y \neq y'$ (meaning that if Alice and Bob receive the same vertex, but respond with two different vertices, then they always lose). The following lemma shows some given facts about synchronous games, which will help in proving later theorems.

\begin{lem}
\label{synclem}
Let $\psi \in \mathbb{C}^d \otimes \mathbb{C}^d$, $\mathcal{A}_x = \{A_{xy}: y \in Y\}$ and $\mathcal{B}_x = \{B_{xy}: y \in Y\}$ for all $x \in X$ be a perfect quantum strategy for a synchronous game. If $\psi$ and the operators $A_{xy}, B_{xy}$ are expressed in the Schmidt basis of $\psi$ and $\rho = mat(\psi)$, then:
\begin{itemize}
\item $A_{xy} = B_{xy}^T$ for all $x \in X$, $y \in Y$
\item $A_{xy}$ and $B_{xy}$ are projectors for all $x \in X$, $y \in Y$
\item $A_{xy} \rho = \rho A_{xy}$ and $B_{xy} \rho = \rho B_{xy}$ for all $x \in X$, $y \in Y$
\item $p(y, y'|x, x') = \psi^{\dag} (A_{xy} \otimes B_{x'y'}) \psi = 0$ if and only if $A_{xy}A_{x'y'} = 0$
\end{itemize}
\end{lem}

Now, since the graph isomorphism game is synchronous, we can restate these above conditions in regards to quantum strategies, and introduce some other facts related to lemma \ref{synclem}.

\begin{thm}
\label{syncthm}
Consider two graphs $G$ and $H$. Let $\psi \in \mathbb{C}^d \otimes \mathbb{C}^d$, $\mathcal{A}_x = \{A_{xy}: y \in V(G) \cup V(H)\}$ and $\mathcal{B}_x = \{B_{xy}: y \in V(G) \cup V(H)\}$ for all $x \in V(G) \cup V(H)$ be a perfect quantum strategy. If $\psi$ and the operators $A_{xy}, B_{xy}$ are expressed in the Schmidt basis of $\psi$ and $\rho = mat(\psi)$, then:
\begin{itemize}
\item $A_{xy} = B_{xy}^T$ for all $x, y \in V(G) \cup V(H)$
\item $A_{xy}$ and $B_{xy}$ are projectors for all $x, y \in V(G) \cup V(H)$
\item $A_{xy} \rho = \rho A_{xy}$ and $B_{xy} \rho = \rho B_{xy}$ for all $x, y \in V(G) \cup V(H)$
\item $p(y, y'|x, x') = \psi^{\dag} (A_{xy} \otimes B_{x'y'}) \psi = 0$ if and only if $A_{xy}A_{x'y'} = 0$
\item $A_{xy} = 0$ if $x, y \in V(G)$ or $x, y \in V(H)$
\item $A_{xy} = A_{yx}$ for all $x, y \in V(G) \cup V(H)$
\end{itemize}
\end{thm}
\begin{proof}
The first four conditions are simply the same conditions from lemma \ref{synclem} but restated in terms of a perfect quantum strategy, rather than just a synchronous game. For the fifth condition, consider two vertices, $g, g' \in V(G)$, and - using equation \ref{isometry} - observe that \begin{equation}
0 = \sum_{y \in V(G) \cup V(H)} p(g', y|g, x) = \sum_{y \in V(G) \cup V(H)} \psi^{\dag}(A_{gg'} \otimes B_{xy})\psi = \psi^{\dag}(A_{gg'} \otimes I)\psi = Tr(\rho \rho^{\dag} A_{gg'})
\end{equation}
Recall that $\psi$ is of Schmidt basis, so the matrix $\rho$ has positive diagonal entries only, and thus $\rho \rho^{\dag}$ has full rank. Therefore $A_{gg}' = 0$, and similarly, $B_{gg}' = 0$.

For sixth condition, recall that by the rules of the graph isomorphism game, if $x \neq x'$, then $p(y, x'|y, x) = 0$. So by the fourth condition $A_{xy}A_{yx'} = 0$ $\forall x \neq x'$. Hence: 
\begin{eqnarray}
A_{xy} = A_{xy} \sum_{x'} A_{yx'} = A_{xy}A_{yx} \\
A_{yx} = \sum_{y'} A_{xy'} A_{yx}= A_{xy}A_{yx}
\end{eqnarray}

Therefore, we can combine the two equations to get $A_{xy} = A_{yx}$
\end{proof}


Defining graph isomorphism using adjacency and permutation matrices is one of the well-known classical definitions, but we are now well-equipped to present a quantum analogue definition using projective permutation matrices instead.

\begin{defn}
A projective permutation matrix (ppm) $\mathcal{P} \in \mathbb{M}_n(d)$ is a unitary block matrix of size $d$ whose elements are projectors.
\end{defn}

\begin{lem} \label{ppm}
A matrix $\mathcal{P} = [[A_{ij}]] \in \mathbb{M}_n(d)$ is a projective permutation matrix (ppm) if and only if its elements, $A_{ij}$ are projectors $\forall 1 \leq i, j \leq n$ and:
\begin{enumerate}
\item $\sum_{i=1}^n A_{ij} = I$ $ $ $1 \leq j \leq n$
\item $\sum_{j=1}^n A_{ij} = I$ $ $ $1 \leq i \leq n$
\end{enumerate}
\end{lem}
\begin{proof}
$\Rightarrow$ Suppose we have $\mathcal{P} = [[A_{ij}]] \in \mathbb{M}_n(d)$ that is unitary and all $A_{ij}$ are projectors. Then $\mathcal{P} \mathcal{P}^{\dag} = I$. It follows that $I = (\mathcal{P} \mathcal{P}^{\dag})_{i, i} = \sum_j A_{ij}^2 = \sum_j A_{ij}$. Similarly: $I = (\mathcal{P}^{\dag} \mathcal{P})_{i, i} = \sum_i A_{ij}^2 = \sum_i A_{ij}$. Thus conditions 1 and 2 hold.
$\Leftarrow$ Conversely, suppose conditions 1 and 2 hold. Use these projectors to construct a projective permutation matrix $\mathcal{P} = [[A_{ij}]]$. Since $A_{ij}$ are projectors, we have that $A_{ik}A_{jk} = 0$ for all $i \neq j$. Then take the product: 
\[ (\mathcal{P}\mathcal{P}^{\dag})_{i,j} = \sum_k A_{ik}A_{jk} = \begin{cases} 
\sum_k A_{ik}^2 = I, & \text{ if } i = j \\
0, & \text{ if } i \neq j  \end{cases}
\]
It follows that $\mathcal{P}\mathcal{P}^{\dag} = I$, and similarly $\mathcal{P}^{\dag} \mathcal{P} = I$. Therefore, $\mathcal{P}$ is unitary.
\end{proof}

Before we can use ppm's to define quantum graph isomorphism, we show a different, but related, way to define graph isomorphism. Using this theorem, it will be easier to prove the next theorem which is the quantum analogue of theorem \ref{permutation}.

\begin{thm} \label{projqiso}
Two graphs $G$ and $H$ are quantum isomorphic if and only if there exists projectors $A_{gh}$ for $g \in V(G)$ and $h \in V(H)$ such that:
\begin{enumerate}
\item $\sum_{h \in V(H)} A_{gh} = I$  $\forall g \in V(G)$
\item $\sum_{g \in V(G)} A_{gh} = I$  $\forall h \in V(G)$
\item $A_{gh}A_{g'h'} = 0$, if $rel(g, g') \neq rel(h, h')$
\end{enumerate}
\end{thm}
\begin{proof}
$\Rightarrow$) If the graphs $G$ and $H$ are quantum isomorphic, then we can say there exists a quantum strategy as in Theorem \ref{syncthm}. Then by the second condition in Theorem \ref{syncthm}, and by definition of families of projectors (see section \ref{prelim}), conditions 1 and 2 are satisfied. By the rules of the game $rel(g, g') \neq rel(h, h')$, then $\psi^{\dag} (A_{xy} \otimes B_{x'y'}) \psi = 0$ holds necessarily, and so condition 3 follows.

$\Leftarrow$) On the other hand, define $A_{gh} = A_{hg}$, and $A_{gg'} = A_{hh'} = 0$ $\forall g, g' \in V(G)$ and $\forall h, h' \in V(H)$. Now suppose that these $A_{gh}$ satisfy the above conditions. Moreover, let $B_{xy} = A_{xy}^T$. Then $\mathcal{A}_x = \{A_{xy}: y \in V(G) \cup V(H)\}$ and $\mathcal{B}_x = \{B_{xy}: y \in V(G) \cup V(H)\}$ are both valid measurements for all $x \in V(G) \cup V(H)$. Now consider the quantum strategy where Alice and Bob uses these measurements on a shared maximally entangled state $\psi_d = \frac{1}{\sqrt{2}} \sum_{i=1}^d e_i \otimes e_i$. Then $p(y, y'|x, x') = \psi^{\dag}_d (A_{xy} \otimes B_{x'y'}) \psi_d= \frac{1}{d}Tr(A_{xy}A_{x'y'})$ $\forall x, x', y, y' \in V(G) \cup V(H)$. It follows that this correlation combined with condition 3 satisfies the rules of the game, and hence there exists a perfect quantum strategy.
\end{proof}

Finally, we are well-equipped to introduce and prove the quantum analogue of Theorem \ref{permutation}.

\begin{thm}
For any two graphs $G$ and $H$, $G \cong_q H$ if and only if there exists a projective permutation matrix $\mathcal{P}$ such that: $(A_G \otimes I_d)\mathcal{P} = \mathcal{P} (A_H \otimes I_d)$
\end{thm}
\begin{proof}
$\Rightarrow$) Let $\mathcal{P} = [[A_{gh}]]$ for $g \in V(G)H$ and $h \in V(H)$. By Lemma \ref{ppm}, the elements are projectors and satisfy conditions 1 and 2 from Theorem \ref{projqiso}. Then to prove $(A_G \otimes I_d)\mathcal{P} = \mathcal{P} (A_H \otimes I_d)$, we can just show that it is equivalent to condition 3 from Theorem \ref{projqiso}. Observe that conditions 1 and 2 from Theorem \ref{projqiso} already tells us that $A_{gh}A_{g'h'} = 0$ whenever $g = g'$ and $h \neq h'$ or vice versa, and so we only need to show the orthogonality aspect. 

The (g, g')-block of $A_G \otimes I_d$ is equal to $I_d$ if $g \sim g'$ but zero otherwise, and similarly for $A_H \otimes I_d$. Moreover, observe that for $g \in V(G)$ and $h \in V(H)$:
\begin{equation}
\label{ppmelement}
((A_G \otimes I_d)\mathcal{P})_{g,h} = \sum_{g':g' \sim g} A_{g'h} \textit{  and  } ((A_H \otimes I_d)\mathcal{P})_{g,h} = \sum_{h':h' \sim h} A_{g'h}
\end{equation} 
So if condition 3 of Theorem \ref{projqiso}, we get that:
\begin{equation}
\sum_{g' \sim g} A_{g'h} = \sum_{g' \sim g} A_{g'h} \sum_{h'} A_{gh'} = \sum_{g' \sim g} A_{g'h} \sum_{h' \sim h} A_{gh'} = \sum_{g'} A_{g'h} \sum_{h' \sim h} A_{gh'} = \sum_{h' \sim h} A_{gh'}
\end{equation}
And so by equation (\ref{ppmelement}), $(A_G \otimes I_d)\mathcal{P} = \mathcal{P} (A_H \otimes I_d)$.

$\Leftarrow$) Conversely, if $(A_G \otimes I_d)\mathcal{P} = \mathcal{P} (A_H \otimes I_d)$, equation (\ref{ppmelement}) says that, $\forall g \in V(G), h \in V(H)$ :
\begin{equation}
\sum_{g':g' \sim g} A_{g'h} = \sum_{h':h' \sim h} A_{gh'} 
\end{equation}
\end{proof}

Before we move on to the next theorem, a useful lemma is provided to help prove the next theorem with ease. This lemma proves that ppm's indeed form a group, and more importantly, are closed under multiplication.

\begin{lem}
Projective permutation matrices form a group under multiplication.
\end{lem}
\begin{proof}
Identity: There exists an identity ppm, which is just a ppm with identity matrices in all of the diagonal elements.

Associativity: Since ppm's are still essentially matrices, associativity holds as it would for any family of matrices.

Invertibility: Projective permutations matrices are unitary, so take the conjugate transpose of any element to get its inverse.

Closure: Consider two ppm's, $P = \begin{bmatrix}
E_{11} & \dots & E_{1k} \\
\vdots  & \ddots \\
E_{k1} &        & E_{kk}
\end{bmatrix}$ and $Q = \begin{bmatrix}
F_{11} & \dots & F_{1k} \\
\vdots  & \ddots \\
F_{k1} &        & F{kk}
\end{bmatrix}$. 

Need to check if PQ is a ppm:
$PQ = \begin{bmatrix}
E_{11} & \dots & E_{1k} \\
\vdots  & \ddots \\
E_{k1} &        & E_{kk}
\end{bmatrix}
\times
\begin{bmatrix}
F_{11} & \dots & F_{1k} \\
\vdots  & \ddots \\
F_{k1} &        & F_{kk}
\end{bmatrix}
= 
\begin{bmatrix}
E_{11}F_{11} + ... + E_{1k}F_{k1} & \dots  \\
\vdots  & \ddots \\
E_{k1}F_{k1} + ... + E_{kk}F_{kk} & \dots
\end{bmatrix}
$
\end{proof}

\begin{thm}
Quantum Graph Isomorphism forms an equivalence relation between graphs on $n$ vertices.
\end{thm}
\begin{proof}
Reflexivity: Let $\mathcal{P}$ be the identity ppm. Then $\cong_q$ is reflexive.

Symmetry: If $G \cong_q H$, then $(A_G \otimes I_d)\mathcal{P} = \mathcal{P} (A_H \otimes I_d)$

$\Rightarrow$ $\mathcal{P}^\dag (A_G \otimes I_d) = (A_H \otimes I_d) \mathcal{P}^\dag$

$\Rightarrow$ $(A_H \otimes I_d) \mathcal{P}^\dag = \mathcal{P}^\dag (A_G \otimes I_d)$
Thus, $\cong_q$ is symmetric.


Transitivity: Suppose there exists $\mathcal{P}, \mathcal{P}'$, such that $(A_G \otimes I_d)\mathcal{P} = \mathcal{P} (A_H \otimes I_d)$ and $(A_H \otimes I_{d'}) \mathcal{P} = \mathcal{P}' (A_K \otimes I_{d'})$ for some graphs $G$, $H$, and $K$. Now let $D = max(d, d')$. Then we still have $(A_G \otimes I_D)\mathcal{P} = \mathcal{P} (A_H \otimes I_D)$ and $(A_H \otimes I_{D}) \mathcal{P}^\dag = \mathcal{P}'^\dag (A_K \otimes I_{D})$. So, $\mathcal{P}^\dag (A_G \otimes I_D)\mathcal{P} = (A_H \otimes I_D)$, and $(A_H \otimes I_D) = \mathcal{P}' (A_K \otimes I_{D}) \mathcal{P}'^\dag$. 

$\Rightarrow$ $\mathcal{P}^\dag (A_G \otimes I_D)\mathcal{P} = \mathcal{P}' (A_K \otimes I_{D}) \mathcal{P}'^\dag$

$\Rightarrow$ $(A_G \otimes I_D)\mathcal{P} \mathcal{P}' = \mathcal{P} \mathcal{P}' (A_K \otimes I_{D})$

Since ppms are closed under multiplication, $\mathcal{P} \mathcal{P}'$ is a ppm, which means $G \cong_q K$. Hence $\cong_q$ is transitive.
\end{proof}


\section{Comparing Various Definitions}
Now we can look at how quantum and classical graph isomorphisms are related.

\begin{proposition}{First Isomorphic Relation}
\label{prop:isoimpliesqiso}
$G \cong H$ $\implies$ $G \cong_q H$
\end{proposition}
To see this, observe that every classical isomorphism case can be described by the quantum game in the following way: Take two isomorphic graphs, $G$ and $H$, so that there is some classical winning strategy $f$. Let all POVM elements be the zero matrix except for one, which will be the identity. Choose the POVM element that will be identity matrix to be $A_{x_Ay_A} = 
\begin{bmatrix}
1 & 0 \\
0 & 1
\end{bmatrix}$, such that $y_A = f(x_A)$, and similarly for $B_{x_By_B}$. Then $\psi^\dag(A_{x_Ay_A} \otimes B_{x_By_B})\psi = 0$ whenever conditions 1 or 2 fail, thus, this is a winning quantum strategy.

\begin{proposition}{Second Isomorphic Relation}
$G \cong_q H$ $\centernot\implies$ $G \cong H$
\end{proposition}
To see this, we must introduce a new concept to show an important aspect of quantum graph isomorphism, which is that there exists graphs that are quantum graph isomorphic but not isomorphic.

A linear binary constraint system (BCS) $\mathcal{F}$ is a family of binary variables, with a family of constraints. The following is an example that we will refer to later.

\begin{equation*}
\label{bcs}
\openup\jot % make lines a little more far apart
\begin{aligned}[t]
x_1 + x_2 + x_3 &= 0 \\
x_4 + x_5 + x_6 &= 0 \\
x_7 + x_8 + x_9 &= 0
\end{aligned}
\qquad\qquad % adjust to suit
\begin{aligned}[t]
x_1 + x_4 + x_7 &= 0 \\
x_2 + x_5 + x_8 &= 0 \\
x_3 + x_6 + x_9 &= 1
\end{aligned}
\end{equation*}

\begin{defn}
If there is an assignment of values from $\mathbb{F}_2$ to the variables such that every constraint is satisfied, then the BCS is called satisfiable.
\end{defn}
Note that this linear BCS is not satisfiable, since if you add all of the equations together, you will get $1=0$. Now, we can define a game to any BCS as follows: Let $C_l$ and $C_k$ be some constraints. A referee will send a constraint to both Alice and Bob, who will then respond with assignments $f, f'$ respectively. The assignments $f, f'$ will be assignments of variables that appear in $C_l$, and $C_k$ respectively. The goal again is to have some strategy for Alice and Bob to follow that meets two conditions: the first being that the constraints are satisfied, and the second being if $\exists x_i \in C_l \cap C_k$, then $f(x_i) = f'(x_i)$. It follows that Alice and Bob can win the game with probability 1 iff the BCS is satisfiable. Since this is a similar game to the previous one, some classical strategy, surely we can have a quantum strategy as well, which bring us to the next definition.
\begin{defn}
A BCS is called quantum satisfiable if there exists a perfect quantum strategy for the BCS game.
\end{defn}

Now, given any linear BCS, $\mathcal{F}$ with m constraints, we can associate a graph to it in the following way: for each constraint $C_{\ell}$ and each assignment $f:S_{\ell} \rightarrow \mathbb{F}_2$ that satisfies the corresponding constraint, include/create a vertex $(\ell, f)$. Consider some other vertex $(k, f')$, then if there exists an $x_i \in S_{\ell} \cap S_{k}$ such that $f(x_i) \neq  f'(x_i)$, then add an edge between these two vertices.

\begin{defn}
The homogenization of $\mathcal{F}$, denoted $\mathcal{F}_0$, is a linear BCS where the right hand side of all constraints in $\mathcal{F}$ are changed to zero.
\end{defn}
We can then construct another graph using the homogenization of the original BCS. The next two theorems bring us back to quantum isomorphism.


\begin{thm}
\label{classicalBCS}
Let $\mathcal{F}$ be a linear BCS. Then $\mathcal{F}$ is satisfiable iff $G_{\mathcal{F}}$ is isomorphic to $G_{\mathcal{F}_0}$
\end{thm}

\begin{thm}
\label{quantumBCS}
Let $\mathcal{F}$ be a linear BCS. Then $\mathcal{F}$ is quantum satisfiable iff $G_{\mathcal{F}}$ is quantum isomorphic to $G_{\mathcal{F}_0}$
\end{thm}

Using the linear BCS from \ref{bcs}, we can construct two graphs $G_{\mathcal{F}}$ and $G_{\mathcal{F}_0}$ using the methods described. We know that this linear BCS is not classically satisfiable, so theorem \ref{classicalBCS} tells us that the corresponding graphs will not be classically isomorphic. However, this linear BCS is a well known linear BCS and was constructed to specifically be quantum satisfiable, and hence the graphs constructed will be quantum isomorphic. So we finally have two graphs that are quantum isomorphic but not classically isomorphic. The graphs are in Figure \ref{qisoNOTiso}.

\begin{figure}
\label{qisoNOTiso}
\centering
  \begin{subfigure}[b]{0.4\textwidth}
    \includegraphics[width=\textwidth]{"GF_quantum_iso".PNG}
    \caption{Graph for $G_{\mathcal{F}}$}
    \label{fig:Gf}
  \end{subfigure}
  %
  \begin{subfigure}[b]{0.4\textwidth}
    \includegraphics[width=\textwidth]{"GF0_quantum_iso".PNG}
    \caption{Graph for $G_{\mathcal{F}_0}$}
    \label{fig:Gf0}
  \end{subfigure}
\end{figure}



\section{Graphs for which classical and quantum isomorphism coincide}
\subsection{Non-signalling and fractional graph isomorphism}
In this section, we briefly introduce two new types of isomorphism, namely fractional isomorphism and non-signalling isomorphism. Fractional isomorphism is a well studied type of isomorphism, and certain results may be of use in finding relations between quantum isomorphism and classical isomorphism. The significance of fractional isomorphism comes from non-signalling isomorphism, as non-signalling isomorphism is related to quantum isomorphism, and more importantly it implies fractional isomorphism. First we state the formal definitions.

\begin{defn} Let $G$ and $H$ be graphs, and $A_G$, $A_H$ be there adjacency matrices respectively. Then $G$ is fractionally isomorphic to $H$ if there exists a \hyperlink{stochastic}{doubly stochastic matrix} $S$, such that $A_GS = SA_H$. We denote fractional isomorphism by $G \cong_f H$.
\end{defn}

Recall the definition of quantum isomorphism (definition \ref{qgame}), and the description provided above it. The definition follows similarly to the game with quantum resources: suppose Alice and Bob are playing the nonlocal game with and they have a quantum strategy. If Alice is given input $x_A$, and Bob is given input $x_B$, the probability that Alice obtains outcome $y_A$ when she performs measurement $\mathcal{A}_{x_A}$ is given by: $\sum_{y_B}p(y_A, y_B | x_A, x_B) = \sum_{y_B}\psi^\dag(A_{x_Ay_A}\otimes B_{x_By_B})\psi = \psi^\dag(A_{x_Ay_A}\otimes I)\psi$. This essentially says that Bob's output vertex is not dependent on Alice's measurement. Similarly, Bob's measurement is not dependent on Alice's output vertex, and so the formal definition follows.

\begin{defn}
A correlation $p(y_A, y_B | x_A, x_B)$ is non-signalling if 
\begin{itemize}
\item $\sum_{y_B}p(y_A, y_B | x_A, x_B) = \sum_{y_B}p(y_A, y_B | x_A, x_B')$ for all $x_A, y_A, x_B, x_B'$
\item $\sum_{y_A}p(y_A, y_B | x_A, x_B) = \sum_{y_B}p(y_A, y_B | x_A', x_B)$ for all $x_B, y_A, x_A, x_A'$
\end{itemize}
If all vertices come from two graphs $G$ and $H$, then they are non-signalling isomorphic, denoted by $\cong_{ns}$.
\end{defn}

Non-signalling isomorphism corresponds to the restriction of Alice and Bob are not allowed to communicate at all (where normally when the game is played, they are allowed to communicate before the game starts). Since nothing can travel faster than the speed of light, any correlation with two parties at a sufficient distance away from each other will be a non-signalling correlation. For this reason, non-signalling isomorphism is an interesting topic to study, however its interest here comes from its implication. It is known that if $G \cong_q H$, then $G \cong_{ns}H$, but we now prove a theorem shows a relation between quantum isomorphism and fractional isomorphism.

\begin{thm}
Let $G$ and $H$ be graphs. Then $G \cong_{ns} H$ if and only if $G \cong_f H$.
\end{thm}

\subsection{Graph families}
An interesting open question about graph isomorphism is what types of graphs might have quantum isomorphism always implying classical isomorphism. Below is a list of graphs that might be of particular interest.
\begin{itemize}
\item Trees
\item Cycles
\item Chordal graphs
\item Bipartite graphs and complete bipartite graphs
\item Complete graphs 
\item unicyclic graphs
\item circulant graphs (see Well-covered circulant graphs by Brown and Hoshino)
\end{itemize}
Now, together with a few results from other papers, we can establish a family of graphs for which classical and quantum graph isomorphism are equivalent relations.

\begin{thm}
Let $G$ and $H$ be trees. Then $G \cong H$ $\Leftrightarrow$ $G \cong_q H$.
\end{thm}
\begin{proof}
By \cref{prop:isoimpliesqiso} we know that if $G$ and $H$ are classically isomorphic then they are also quantum isomorphic. Conversely, if $G$ and $H$ are quantum isomorphic then they are fractionally isomorphic. But fractionally isomorphic trees are classically isomorphic so $G$ and $H$ are classically isomorphic.
\end{proof}


It might be useful to use the fact that graphs are fractionally isomorphic if and only if they are indistinguishable by color refinement.

\newpage

\bibliographystyle{unsrt}
\addcontentsline{toc}{section}{Bibliography}
\bibliography{Bibliography}

\end{document}


