% simple.tex -- a very simple thesis document for demonstrating
%   dalthesis.cls class file
\documentclass[12pt]{dalthesis}
\usepackage{dsfont}
\usepackage{caption}
\usepackage{subcaption}
\usepackage{float}
\usepackage{graphicx}

% ----------------------------------------------------------------------
% Packages.
\usepackage{amsmath,amsfonts,amsthm,amssymb}
\usepackage{braket}
\usepackage{mathdots}
\usepackage{mathtools}
\usepackage{tikz}
\usetikzlibrary{calc,decorations.pathreplacing,positioning,shapes}
\usepackage{hyperref}
\usepackage[capitalize]{cleveref}
\usepackage[normalem]{ulem}
\usepackage{qcircuit}
\usepackage{tikz-cd}
\usetikzlibrary{cd}
\usepackage{mleftright}
\usepackage{comment}
\newcommand{\note}[1]{{\color{red} #1}}

% ----------------------------------------------------------------------
% Some notation.

% Function restriction.
\newcommand\restrict[1]{\raisebox{-.5ex}{$|$}_{#1}}

% Residue of a matrix M.
\newcommand\mbar{\overline{M}}

% Matrix rings.
\newcommand{\matricesmn}[3]{\mathcal{M}_{#1\times #2}(#3)}
\newcommand{\matrices}[1]{\mathcal{M}(#1)}

\newcommand{\pauli}[1]{\mathcal{P}_{#1}}
\newcommand{\clifford}[1]{\mathcal{C}_{#1}}
\newcommand{\realclifford}[1]{\mathcal{C}_{#1}^{\mathbb{R}}}
\newcommand{\weight}{\operatorname{weight}}
\newcommand{\realpauli}[1]{\mathcal{P}_{#1}^{\mathbb{R}}}

% CNOT and SWAP gates.
\newcommand{\CNOT}{\mathrm{CNOT}}
\newcommand{\SWAP}{\mathrm{SWAP}}

% Matrices.
\newcommand{\diag}{\operatorname{diag}}

% Rings.
\newcommand{\Z}{\mathbb{Z}}
\newcommand{\N}{\mathbb{N}}
\newcommand{\F}{\mathbb{Z}_2}
\newcommand{\R}{\mathbb{R}}
\newcommand{\C}{\mathbb{C}}
\newcommand{\Zi}{\Z\left[i\right]}
\newcommand{\Zomega}{\Z\left[\omega\right]}
\newcommand{\Zrt}{\Z\left[\frac{1}{\sqrt{2}}\right]}
\newcommand{\Zrtwo}{\Z\left[\sqrt{2}\right]}
\newcommand{\Zrtwoi}{\Z[1/\sqrt{2},i]}
\newcommand{\Zint}{\Z\left[\sqrt{2}\right]}
\newcommand{\rtZrt}{\frac{1}{\sqrt{2}^k}\Z\left[\sqrt{2}\right]}

% Norm and adjoint.
\newcommand{\norm}[1]{\left\lVert#1\right\rVert}

% Misc math notation.
\newcommand{\inv}{^{-1}}
\newcommand{\s}[1]{\{#1\}}

\newcommand{\Stab}{\mathrm{Stab}}


% ----------------------------------------------------------------------
% Theorem like environments.
\theoremstyle{plain}
\newtheorem{theorem}{Theorem}[section]
\newtheorem{lemma}[theorem]{Lemma}
\newtheorem{proposition}[theorem]{Proposition}
\newtheorem{corollary}[theorem]{Corollary}
\theoremstyle{definition}
\newtheorem{definition}[theorem]{Definition}
\newtheorem{example}[theorem]{Example}

\theoremstyle{remark}
\newtheorem{remark}[theorem]{Remark}

% ----------------------------------------------------------------------
% Macros for bibliography.

\newcommand{\urlalt}[2]{\href{#2}{\nolinkurl{#1}}}
\newcommand{\arxiv}[1]{\urlalt{arXiv:#1}{http://arxiv.org/abs/#1}}

%----------------------------------------------------------------------
% A hack: \m is similar to \vcenter, but works better.
% \mp{0.2}{x}: raise the box x so that 20% of it is below the
% centerline. Unlike \vcenter, don't change the horizontal spacing.

\newlength{\localh}
\newlength{\locald}
\newbox\mybox
\def\mp#1#2{\scalebox{0.8}{\setbox\mybox\hbox{#2}\localh\ht\mybox\locald\dp\mybox\addtolength{\localh}{-\locald}\raisebox{-#1\localh}{\box\mybox}}}
\def\m#1{\mp{0.5}{#1}}


\begin{document}

\title{Lower Bounds for Quantum Circuits}
\author{Ravi Rai}

% The following degrees are included in the current dalthesis.cls
% class file:
\mcs  % options are \mcs, \macs, \mec, \mhi, \phd, and \bcshon

% If you degree is not included, you can set several options manually.
% The following example shows the parameters for the \mcs degree.
% However, if you need to set these parameters manually, please check
% the correct names with the Faculty of Graduate Studies, and let the
% maintainer of this class file know (Vlado Keselj, vlado@cs.dal.ca).
% MCS Example:

\degree{Master of Science}
\degreeinitial{M.Sc.}
\faculty{Computer Science}
\dept{Faculty of Computer Science}

% Month and Year of Defence
\defencemonth{August}\defenceyear{2021}

\dedicate{\ldots}

% This sample thesis contains no tables nor figures, so there is no
% need to include lists of tables and figures in the front matter:
\nolistoftables
\nolistoffigures

\frontmatter

\begin{abstract}
In quantum computing, computational tasks are represented by quantum circuits. Quantum circuits have gates that represent operators, which transform the state of one or many qubits. Applying these gates comes at a cost, so minimizing this cost is of particular interest. Certain gates and operations are considered cheap, those being the gates in the Clifford group. It is well known that the Clifford group is not universal for quantum computing, however, but also that adding just one non-Clifford operation gives universality. Since non-Clifford operations are expensive, it becomes sensible to consider them a resource whose use is to be minimized, and thus gives reason to study lower and upper bounds for such resources.

Finding such upper bounds comes more easily as it can be simply done by observing best known circuits that use a particular resource. Finding lower bounds tends to be a more challenging task and requires the use of monotonic functions. These monotonic functions, or monotones, are real-valued functions with state vectors as inputs, and are monotone under Clifford operations. Considering a state and a transformed version of that state under some Clifford operation, one can obtain a lower bound for a certain resource. Two monotones are introduced, the first called the Stabilizer Nullity, and the second the Dyadic monotone. Some of the resources considered here are the $C^nZ$ gate, the $T$ gate, and the $S$ gate. The results are lower bounds giving how many of these gates are needed using the first monotone, with tighter lower bounds using the second monotone, though with further restrictions. Well-known upper bounds are provided for contrast as well, and even match the lower bound for one particular gate. Lower and upper bounds being equal is ideal, and to continue pursuing this for other gates is the natural next step. Moreover, to achieve this other monotones may be worth considering to relax restrictions on their use.
\end{abstract}

\begin{acknowledgements}
\ldots
\end{acknowledgements}

\mainmatter

\chapter{Introduction}
\section{Motivation}
In Quantum information theory there are certain computational tasks that are expensive to perform. These tasks come in the form of operators, and finding upper and lower bounds on them to perform larger scale computational tasks is of particular interest. In this paper we explore methods using stabilizer operations to do so. To find upper bounds on an expensive gate needed to perform a computational task, one need only state explicitly a quantum circuit to show that it can be done with no more than the upper bound. As such, upper bounds are less interesting as, for commonly used expensive gate, there upper bounds are already well known. Regardless, it is still of interest to compare and contrast them to lower bounds, there more interesting counterpart. 

The focus here is mostly on the new discovery of lower bounds, and the stabilizer operations used to acquire them. Lower bounds are more challenging to find due to the need to first study monotonic functions. These monotonic functions take the state of a quantum system to give real-valued outputs, and in the end can give the lower bounds of interest in this paper. The potential in their use is yet to be fully explored, as defining new monotones can give different, and possibly tighter, lower bounds. 

\section{The Stabilizer and Monotonic Functions}
In chapter \ref{chap Stab} we introduce the stabilizer formalism, which is the backbone of our methods to realizing lower bounds for resources (where resources here refers to either an expensive gate or circuit). We start with some basic definitions regarding the Pauli matrices, to then get into the definition of the stabilizer of a state. The stabilizer relies on these Pauli matrices, but more importantly, the monotonic functions that we introduce relies on this stabilizer. 
Monotonic functions are non-increasing functions, and in our case they are non-increasing under Clifford operations and Pauli measurements (where Clifford operations can be thought of as inexpensive resources, thus negligible in cost). In this paper, there are two important monotones that we discuss. The first one is called the stabilizer nullity, which we use first to get lower bounds for our circuits discussed later on. The stabilizer nullity gives us more general lower bounds, and is easier to compute as well, making it the easier one to deal with. 
The second monotone that we discuss is called the dyadic monotone. This function however us only non-increasing under Clifford operations and Pauli measurements when the Pauli measurements are restricted to having probability $1/2$. Fortunately, with this added restriction comes tighter lower bounds, thus meaning that we can perform certain computational expensive tasks with fewer resources given that our measurements are done with probability $1/2$ only. Since this restriction may not always be feasible, we still have the stabilizer nullity lower bounds regardless. 
A nice observation to make is the comparison between upper and lower bounds, which is precisely what we did for the $C^nZ$ gate, which is the $n$-controlled $Z$ gate. In the case that the lower bound and upper bounds are the same for a given resource, then we know for certain what the optimal number of said resource is required to perform the task at hand. This case only arises once in this paper, and in the other cases having different lower and upper bounds means that we still do not know if we can do better. 

\section{Conversion Examples}
When we refer to a conversion it means we are applying a Clifford or non-Clifford circuit to some gate or circuit to get an equivalence to some other gate or circuit of interest. For example for some circuit $D$ there is some other Clifford circuit $C$ such that $C \otimes D = E$, where $E$ is another circuit of interest. In the case of $C$ not being Clifford, then we call it a catalyst, where we will instead have $C \otimes D = E \otimes D$. As explained above, we are interested in the number of a certain resource required to use this equivalences, and the stabilizer nullity and dyadic monotones both give us this answer. 

The first main conversion is looking at going from the $T$, $CS$, and $CCZ$ gates to the $C^nZ$ gate. All three of these conversions do not require a catalyst, so a Clifford circuit is all that is required. Thus in this case the only expensive resource to consider is the one of interest, being either the $T$, $CS$, or $CCZ$ gate. First we provide and prove lower bounds using the stabilizer nullity, and then the dyadic monotone. For example, using the stabilizer nullity we find that at least $n+1$ $T$ gates are needed to perform a $C^nZ$ gate, but then get that only $2\cdot (n+1)$ $T$ gates are needed when using the dyadic monotone, and similar values are obtained for the other gates. Following this, we show the best known circuit to show these equivalences, explicitly giving upper bounds.

Next we introduce the quantum Fourier transform which allows us to give bounds for the modular adder circuit later on. Here a similar result is proven, where lower bounds for the $T$, $CS$, and $CCZ$ gates are given in order to perform the modular adder circuit. The stabilizer nullity monotone is used first to do this, and then the dyadic monotone next to provide the final result.


\chapter{Foundational Quantum Computing}
\label{foundations}
\section{Preliminaries}
\subsection{Vectors and Matrices}
Let $\mathbb{C}$ be the set of complex numbers. We write $\mathbb{C}^n$ to represent the space of $n$-dimensional column vectors, and we write $\mathbb{C}^{n \times m}$ for the space of matrices with $n$ rows and $m$ columns. Matrices can be multiplies in the usual way. 

The complex conjugate of a scalar $c \in \mathbb{C}$ is denoted $\bar{c}$, and the adjoint of a matrix $M = (c_{ij}) \in \mathbb{C}^{n \times m}$ is $M^\dag = (\bar{c}_{ji}) \in \mathbb{C}^{m \times n}$. The trace of a matrix $M$, denoted $tr(M)$, is the sum of its diagonal entries. Note that $tr(BA) = tr(BA)$. 

For a vector $v \in \mathbb{C}^n$, the norm of $v$ is $||v|| = \sqrt{v^\dag v}$. The vector $v$ is called a unit vector if $||v|| = 1$. A matrix $U \in \mathbb{C}^{n \times n}$ is \emph{unitary} if $U^{-1} = U^\dag$ and \emph{hermitian} if  $A = A^\dag$.

\subsection{Tensor Products}
\label{tensor products}
The \emph{tensor product} is defined as usual and is denoted by $\otimes$. When a basis is fixed, the tensor product can be computed as the \emph{kronecker product}: the tensor product $w = u \otimes v \in \mathbb{C}^{nm}$ of two vectors is defined by $w_{(i, j)} = u_iv_j$. Similarly the tensor product $C = A \otimes B \in \mathbb{C}^{nm \times nm}$ of two matrices is defined by $c_{(i, j)(i', j')} = a_{ii'}b_{jj'}$, with pairs $(i, j)$ ordered lexicographically. Note that $\mathbb{C}^n \otimes \mathbb{C}^m = \mathbb{C}^{nm}$ and $\mathbb{C}^{n \times n} \otimes \mathbb{C}^{m \times m} = \mathbb{C}^{nm \times nm}$. For example, the kronecker product between two $2 \times 2$ matrices looks like:
\begin{equation*}
\begin{pmatrix}
a_{11} & a_{12} \\
a_{21} & a_{22} 
\end{pmatrix} 
\otimes 
\begin{pmatrix}
b_{11} & b_{12} \\
b_{21} & b_{22} 
\end{pmatrix} 
=
\begin{pmatrix}
a_{11}b_{11} & a_{11}b_{12} & a_{12}b_{11} & a_{12}b_{12} \\
a_{11}b_{21} & a_{12}b_{22} & a_{12}b_{21} & a_{12}b_{22} \\
a_{21}b_{11} & a_{21}b_{12} & a_{22}b_{11} & a_{22}b_{12} \\
a_{21}b_{21} & a_{21}b_{22} & a_{22}b_{21} & a_{22}b_{22}
\end{pmatrix}.
\end{equation*}


A basis for a tensor product of vector spaces can be obtained as the tensor product of the basis elements of the individual vector spaces. For example if $\{ a_1, a_2 \}$ and $\{ b_1 , b_2  \}$ are two bases for $\mathbb{C}^2$, then $\{ a_1 \otimes b_1, a_1 \otimes b_2, a_2 \otimes b_1, a_2 \otimes b_2 \}$ is a basis for $\mathbb{C}^4$. Note that not all elements of $\mathbb{C}^4$ are of the form $a \otimes b$ with $a, b \in \mathbb{C}^2$.

\subsection{The Dirac Notation}
In quantum computing, we make use of the \emph{Dirac notation} to represent vectors and operations. In Dirac notation we write a column vector $v$ as a \emph{ket}, denoted by $\ket{v}$. The adjoint of a column vector $u$ is written as a \emph{bra}, denoted by $\bra{u}$. The inner product between two vectors $\ket{v}$ and $\bra{u}$ is then written as a \emph{braket}, denoted by $\braket{u|v}$. for example, the inner product of $\bra{0}$ and $\ket{1}$ is $\braket{0|1}$.

Consider the standard basis vectors $[1, 0]^{\dag}$ and $[0, 1]^{\dag}$. We denote them $\ket{0}$ and $\ket{1}$ respectively. In quantum computing, the basis of $\mathcal{C}^2$ formed by $\{ \ket{0} , \ket{1} \}$ in is known as the \emph{computational basis}. As mentioned before, one can get a basis for higher dimensional vector spaces by taking tensor products of $\ket{0}$ and $\ket{1}$. For example, a basis for $\mathbb{C}^4$ is $\{ \ket{0} \otimes \ket{0}, \ket{0} \otimes \ket{1}, \ket{1} \otimes \ket{0}, \ket{1} \otimes \ket{1} \}$. For brevity, the symbol $\otimes$ is often omitted for elements of the higher dimensional computational bases. For example, $\ket{0} \otimes \ket{0}$ is written as $\ket{00}$ and $\ket{0} \otimes \ket{1}$ is written as $\ket{01}$. In this way, the $j-th$ basis vector of $\mathcal{C}^{2n}$ is denoted $\ket{b_1 b_2 \cdots b_n}$ where $b_1, b_2, \cdots , b_n \in \mathcal{Z}_2$ and $b_1 b_2 \cdots b_n$ is the binary expansion of $j$.

\subsection{Quantum Bits}
The fundamental unit of information in classical computing is the \emph{bit}. In quantum computing the basic unit of information is called a \emph{quantum bit} or \emph{qubit} for short. In classical computing the classical bit can be in the \emph{states} $0$ or $1$, but in quantum computing the state of a qubit is a unit vector in $\mathcal{C}^2$. Hence, the state of a qubit can be any complex linear combination $\alpha \ket{0} + \beta \ket{1}$, where $\alpha , \beta \in \mathbb{C}$ and satisfy $|\alpha | + |\beta | = 1$. The complex numbers $\alpha$ and $\beta$ are called the \emph{amplitudes} of the state. Note that $\ket{0}$ and $\ket{1}$ are valid states, corresponding to $\alpha = 1$ and $\beta = 0$, and to $\alpha = 0$ and $\beta = 1$ respectively. We sometimes call these states \emph{classical}. A state whose amplitudes are both nonzero is said to be a state in \emph{superposition}. For example, a qubit in the state \[ \frac{\ket{0} + \ket{1}}{\sqrt{2}} \] is in (equal) superposition.

Similarly, the state of a collection of $n$ qubits (sometimes called a \emph{register}) is described by a unit vector in $\mathcal{C}^2 \otimes \cdots \otimes \mathcal{C}^2 = \mathcal{C}^{2n}$. For example, a $2$-qubit system could be in the state \[ \frac{\ket{00} + \ket{01} + \ket{10} + \ket{11}}{2}. \]

Interestingly, the state of a multi-qubit system cannot always be expressed as the tensor product of the states of the qubits composing the system. Consider for example the $2$-qubit state \[ \frac{\ket{00} + \ket{11}}{\sqrt{2}}. \]
It can be verified using the definition of the tensor product given in Section \ref{tensor products} that there are no single-qubit states $\ket{v}$ and $\ket{w}$ such that \[ \ket{v} \otimes \ket{w} = \frac{\ket{00} + \ket{11}}{\sqrt{2}}. \]
If two qubits are such that their state can be expressed as a tensor product $\ket{v} \otimes \ket{w}$ then the qubits are said to be \emph{separable}. Otherwise, the qubits are said to be \emph{entangled}.
For example if a pair of qubits is in the state \[ \frac{\ket{00} + \ket{11}}{\sqrt{2}} \] then the qubits are entangled.

In what follows, we will sometimes write $\mathcal{S}(n)$ for the set of all $n$-qubit states and $\mathcal{S}$ for the collection of all states. That is $\mathcal{S} = \cup_n \mathcal{S}(n)$. In addition to the computational basis states defined above, other important states include \[ \ket{+} = \frac{\ket{0} + \ket{1}}{\sqrt{2}} \] and \[ \ket{-} = \frac{\ket{0} - \ket{1}}{\sqrt{2}}. \]
Finally, if $\ket{\psi}$ is a state we write $\ket{\psi}^{\otimes n}$ for the $n$-fold tensor product of $\ket{\psi}$ with itself $\ket{\psi} \otimes \ket{\psi} \otimes \cdots \otimes \ket{\psi}$.

\section{Quantum Operations}
To compute with qubits, one can act on them using two types of operations: unitary evolutions and measurements.
\subsection{Unitary Evolution}
In this case, the state of a quantum system is transformed by applying a unitary transformation to it. For example, say a quantum system is described by the column vector $\ket{\psi}$ and $U$ is some unitary matrix. Then the state after having evolved under $U$ is given by $U \ket{\psi}$. Observe that unitary matrices are isometries since $||Uv|| = ||v||$ holds for all $v$ if and only if $U$ is unitary. A unitary transformation on an $n$ qubit system is also called an $n$-ary \emph{quantum gate}. The following are some notable single qubit gates that will be used extensively in this paper, and are as follows: the Pauli $X$, $Y$, and $Z$ gates, the Hadamard gate $H$, the phase gate $S$, and the $T$ gate. The Pauli matrices are 
\begin{equation*}
  X = \begin{bmatrix}
    0 & 1 \\
    1 & 0
  \end{bmatrix}, \quad
  Y = \begin{bmatrix}
  0 & -i \\
  i & 0
  \end{bmatrix}, \quad \mbox{and} \quad
  Z = \begin{bmatrix}
  1 & 0 \\
  0 & -1
  \end{bmatrix}, \quad
\end{equation*}
and the others are
\begin{equation*}
  H = \frac{1}{\sqrt{2}} \begin{bmatrix}
  1 & 1 \\
  1 & 1
  \end{bmatrix}, \quad
  S = \begin{bmatrix}
  1 & 0 \\
  0 & i
  \end{bmatrix}, \quad \mbox{and} \quad
  T = \begin{bmatrix}
  1 & 0 \\
  0 & e^{i \pi /4}
  \end{bmatrix}.
\end{equation*}
Note that the $X$ gate is synonymous with the $Not$ gate, which can be though of as a simple bit flip gate, i.e. $X\ket{0} = \ket{1}$ and $X\ket{1} = \ket{0}$.

For any unitary $U$ a $CU$ gate represents a controlled-$U$ gate, which means that there will be a single qubit that will act as a controlled qubit so that the $U$ gate will only be applied to other qubits depending on the state of the controlled qubit. The qubits that $U$ will act on are called the target qubits. Note that gates can be multi-controlled as well. Knowing this, there are also important multiqubit gates that will be used later in great detail. They are the $CS$ gate, $CNOT$ gate, and the Toffoli gate (denoted $CCX$), and they have matrices:
\begin{equation*}
  CS = \begin{bmatrix}
  1 & 0 & 0 & 0 \\
  0 & 1 & 0 & 0 \\
  0 & 0 & 1 & 0 \\
  0 & 0 & 0 & i
  \end{bmatrix}, \quad
  CNOT = \begin{bmatrix}
  1 & 0 & 0 & 0 \\
  0 & 1 & 0 & 0 \\
  0 & 0 & 0 & 1 \\
  0 & 0 & 1 & 0 
  \end{bmatrix}.
\end{equation*}
and 
\begin{equation*}
  CCX = \begin{bmatrix}
  1 & 0 & 0 & 0 & 0 & 0 & 0 & 0 \\
  0 & 1 & 0 & 0 & 0 & 0 & 0 & 0 \\
  0 & 0 & 1 & 0 & 0 & 0 & 0 & 0 \\
  0 & 0 & 0 & 1 & 0 & 0 & 0 & 0 \\
  0 & 0 & 0 & 0 & 1 & 0 & 0 & 0 \\
  0 & 0 & 0 & 0 & 0 & 1 & 0 & 0 \\
  0 & 0 & 0 & 0 & 0 & 0 & 0 & 1 \\
  0 & 0 & 0 & 0 & 0 & 0 & 1 & 0 
  \end{bmatrix}.
\end{equation*}

\subsection{Measurement}
Aside from unitary transformations, we can also act on a quantum state by measuring it. Measurement is a probabilistic process, and we say it is like observing the state. We also say that the quantum state collapses due to measurement, and if you were to measure it again you will always get the same result. Measurement itself can be done on different bases of $\mathcal{C}^n$, but for this paper the main important basis is the computational basis. So, say we have a qubit in the state $\ket{\psi} = \alpha \ket{0} + \beta \ket{1}$, and we measure it in the computational basis. Then the post-measurement state will be $\ket{0}$ with probability $|\alpha |^2$ or $\ket{1}$ with probability $|\beta |^2$. 

More formally, quantum measurements are described by a collection of \emph{measurement operators} $\{ M_m \}$ \cite{nielsen00}. The index $m$ refers to its corresponding outcome. When measuring some state $\ket{\phi}$ the probability that the result $m$ occurs is $p(m) = \braket{\phi | M^{\dag}_m M_m | \phi}$ and the post-measurement state is $\frac{M_m \ket{\phi}}{\sqrt{\braket{\phi | M^{\dag}_m M_m | \phi}}}$. In the computational basis for $\mathcal{C}^2$, we have $M_0 = \ket{0} \bra{0}$ and $M_1 = \ket{1} \bra{1}$. Note measuring the state $\ket{\psi}$ above with these formal terms will give the same result as before. 

Measurement is of course not restricted to just a single qubit. This formal method can describe measurement on multiple qubits. Consider an $n$-qubit system described by the state $\ket{\Psi}$. Then measuring in the computational basis, the probability of observing the outcome $j$ is still just $\braket{\Psi | M^{\dag}_j M_j | \Psi}$, where $M_j = \ket{j} \bra{j}$ and $j$ is taken to be in binary form. For example, when $n=3$ one possible outcome could be $\ket{001}$ as this is one of the computational basis states in $\mathcal{C}^4$. Then $M_{001} = \ket{001}\bra{001}$ is used to find the probability of that the measurement outcome will be $\ket{001}$. Note one can also measure an $m$ qubit state in an $n$ qubit system for $m \leq n$, i.e. one can measure only part of a quantum system.


\section{Quantum Circuits}
A quantum circuit describes a sequence of operations acting on a register of qubits. Quantum circuits are made of (horizontal) wires and boxes, where each wire represents a qubit and each box represents a gate. Typically the boxes that represent gates have their corresponding letter inside to indicate what gate it is. Explicitly, letting $U$ be any single qubit gate, a circuit with one qubit and $U$ gate would look like
\begin{equation*}
\Qcircuit @C=0.9em @R=.9em {
   & \gate{U} & \rstick{.} \qw 
}
\end{equation*}

Multiqubit gates are also boxes but with multiple wires connecting to it. Controlled operations are represented with a open or closed circle on the controlled qubit with a vertical wire connecting it to a box on the target qubit. A controlled operation with an closed circle on a wire, and such the controlled qubit, means apply the controlled operation/gate to the other qubits only if the controlled qubit is in the $\ket{1}$ state. An open circle on a controlled qubit means apply the gate if the controlled qubit is in the state $\ket{0}$. The circuit for a controlled operation $CU$ where $U$ is any unitary is
\begin{equation*}
\Qcircuit @C=0.9em @R=.9em {
   & \ctrl{1} & \qw \\
   & \gate{U} & \rstick{.} \qw
}
\end{equation*}
The circuit representation for a multiply controlled gate is written in this fashion also but with multiple controlled qubits.


Quantum circuits are read from left to right, and can be read by tracking what the operations do to the input state. It is worth noting that if two circuits give the same output on a general input state then they are equivalent circuits. In fact, the following example is a circuit that acts exactly as a $CCZ$ gate on qubits $c_1$, $c_2$, and $c_3$, and will be used later on in Section \ref{Upper Bounds for C^nZ}.
\begin{equation*}
\Qcircuit @C=0.5em @R=.7em {
   & \lstick{\ket{c_1}} & \ctrl{2} & \qw & \qw & \qw & \multigate{1}{CZ} & \qw \\
   & \lstick{\ket{c_2}} & \ctrl{1} & \qw & \qw & \qw & \ghost{CZ} & \qw \\
   & \lstick{\ket{0}} & \gate{X} & \ctrl{1} & \gate{H} & \meter & \ctrl{-1} \cw \cwx \\
   & \lstick{\ket{c_3}} & \qw & \targ & \qw & \qw & \qw & \qw
}
\end{equation*}
Tracking from left to right starting with the input state, we get the following steps:
\begin{itemize}
\item[] $\ket{c_1}\ket{c_2}\ket{0}\ket{c_3}$
\item[$\mapsto$] $\ket{c_1}\ket{c_2}\ket{c_1 \cdot c_2}\ket{c_3}$
\item[$\mapsto$] $\ket{c_1}\ket{c_2}\ket{c_1 \cdot c_2}\ket{c_3 \oplus c_1 \cdot c_2}$
\item[$\mapsto$] $\begin{cases} \ket{c_1}\ket{c_2}\ket{+}\ket{c_3 \oplus c_1 \cdot c_2} & \mbox{if } c_1 \cdot c_2 = 0, \\
\ket{c_1}\ket{c_2}\ket{-}\ket{c_3 \oplus c_1 \cdot c_2} & \mbox{if } c_1 \cdot c_2 = 1 \end{cases}$
\item[$\mapsto$] $\begin{cases} \ket{c_1}\ket{c_2}\ket{c_3 \oplus c_1 \cdot c_2} & \mbox{if measurement is } \ket{0} , \\
\pm \ket{c_1}\ket{c_2}\ket{c_3 \oplus c_1 \cdot c_2} & \mbox{if measurement is } \ket{1} \end{cases}$
\item[$\mapsto$] $\ket{c_1}\ket{c_2}\ket{c_3 \oplus c_1 \cdot c_2}$
\end{itemize}
Note that the open circle on the wire for $c_3$ is used interchangeably with the $X$ gate, making this s CNOT or CX gate. Also, the fourth step follows from the fact that $H\ket{0} = \ket{+}$ and $H\ket{1} = \ket{-}$, which can be verified by direct computation. Finally, the last step comes from the fact that $CZ$ here is only applied when the measurement is $\ket{1}$, which then takes care of the $\pm$ giving the result.
 

\chapter{The Clifford Group and Its Universal Extensions}
\label{chap Stab}
\section{The Pauli Operator}
%To begin, we need to introduce and develop the formalism of stabilizer operators and its related quantities, but first some  basic definitions, starting with the Pauli matrices.

Recall the \emph{Pauli matrices} X, Y, and Z from Chapter \ref{foundations}:
  \[
  X= \begin{bmatrix}
    0 & 1 \\
    1 & 0
  \end{bmatrix}, \quad
  Y = \begin{bmatrix}
  0 & -i \\
  i & 0
  \end{bmatrix}, \quad \mbox{and} \quad
  Z = \begin{bmatrix}
  1 & 0 \\
  0 & -1
  \end{bmatrix}.
  \]

If $\mathcal{S}$ is a set of matrices then $\mathcal{S}^{\otimes n}$ is the set of matrices comprised of tensor products of elements in $\mathcal{S}$.

\begin{definition}
\label{PauliGroup}
The \emph{Pauli group on $n$ qubits} $\mathcal{P}(n)$ is the matrix group with elements $\{ \pm I, \pm X, \pm Y, \pm Z, \pm iI, \pm iX, \pm iY, \pm iZ \}^{\otimes n}$.
\end{definition}

Note that the $\pm 1$ and $\pm i$ factors in the elements of $\mathcal{P}(n)$ ensures the closure of the group.


\begin{proposition}
We have $|\mathcal{P}(n)|=4^{n+1}$.
\end{proposition}

\begin{proof}
By induction, first let $n=1$. Then from definition \ref{PauliGroup} we see that there are $16 = 4^2$ elements in $\mathcal{P}(1)$, so the base case is true. Now assume that the statement is true for $m$ qubits, i.e. $|\mathcal{P}(m)| = 4^{m+1}$. Then $\mathcal{P}(m+1)$ has elements that are tensor products of elements in $\mathcal{P}(m)$ and $I$, $X$, $Y$, and $Z$. Thus there are $4 \cdot 4^{m+1} = 4^{(m+1)+1}$ elements in $\mathcal{P}(m+1)$. Note that we only count these tensor products since e.g., $P \otimes -iX = P' \otimes X$ where $P' = -iP$, for $P, P' \in \mathcal{P}(m)$.
\end{proof}

\section{The Clifford Operators}
Recall the gates $H$, $S$, and CNOT defined in Chapter \ref{foundations}.:
  \[
  H = \frac{1}{\sqrt{2}} \begin{bmatrix}
  1 & 1 \\
  1 & 1
  \end{bmatrix}, \quad
  S = \begin{bmatrix}
  1 & 0 \\
  0 & i
  \end{bmatrix}
  \]
and 
  \[
  CNOT = \begin{bmatrix}
  1 & 0 & 0 & 0 \\
  0 & 1 & 0 & 0 \\
  0 & 0 & 0 & 1 \\
  0 & 0 & 1 & 0 \\
  \end{bmatrix}.
  \]
We use these gates to define the \emph{Clifford} group.

\begin{definition}
The Clifford group on $n$ qubits $\mathcal{C}(n)$ consists of the matrices that can be represented by an ancilla-free circuit on $n$ qubits over the gate set $\{H, S, CNOT\}$.
\end{definition}

\begin{proposition}
$H$ and $S$ act on Paulis as follows:
\begin{center}
$HXH^{\dag} = Z$ and $HZH^{\dag} = X$, $SXS^{\dag} = Y$ and $SZS^{\dag} = Z$.
\end{center}
Note that we need not specify the action on $Y$ since $Y = iXZ$, so the action on $X$ and $Z$ gives and fixes the action on $Y$.
\end{proposition}


The state $\ket{0}$ is stabilized by $Z$, and after applying the Hadamard to $\ket{0}$ to get the new state $\ket{+}$, we see that it is now stabilized by $X$. The power in this example can be seen once you consider the general $n$ qubit case. Consider the state $\ket{0}^{\otimes n}$ whose stabilizer is $\langle Z_1, Z_2, \dots , Z_n \rangle$. Then applying the Hadamard to each of of the $n$ qubits gives a state with stabilizer $\langle X_1 , X_2, \dots , X_n \rangle $. Now, observe that the final state vector here requires $2^n$ amplitudes to be specified, compared with the description provided by the generators: $\langle X_1, \dots , X_n \rangle$, which is linear in $n$. 

You can see that there is no entanglement in the quantum computer here, after applying the Hadamard to each qubit. This may lead to thinking that this stabilizer formalism is not so useful, however this changes when you see that there is an efficient description of the controlled-NOT gate, which together with the Hadamard gate can create entanglement. 

It turns out that any unitary in the Clifford group can be composed of Hadamard, phase and controlled-NOT gates. Before we prove this, we need the following two lemma's.

\begin{lemma}
\label{ConjX}
Let $P \in \{\pm X, \pm Y, \pm Z \}$. Then there exists a $C \in <H, S>$ such that $CXC^\dag = P$.
\end{lemma}
\begin{proof}
To prove this we simply list all possible $C$ circuits required for each $P$:
\begin{itemize}
\item If $P = X$ then $C = I$.
\item If $P = -X$ then $C = S^2$.
\item If $P = Y$ then $C = S$.
\item If $P = -Y$ then $C = HS$.
\item If $P = Z$ then $C = H$.
\item If $P = -Z$ then $C = HS^2$.
\end{itemize}
\end{proof}

\begin{lemma}
\label{ConjZ}
Let $Q \in \{\pm Y, \pm Z \}$. Then there exists a $D \in <H, S>$ such that $DQD^\dag = Z$ and $DXD^\dag = X$.
\end{lemma}
\begin{proof}
To prove this we simply list all possible $D$ circuits required for each $Q$:
\begin{itemize}
\item If $Q = Y$ then $D = S^\dag HS^\dag$.
\item If $Q = -Y$ then $D = (S^\dag H)^2 (S^\dag)^2 H$.
\item If $Q = Z$ then $D = I$.
\item If $Q = -Z$ then $D = HS^2$.
\end{itemize}
\end{proof}

\begin{theorem}
\label{HSCNOT}
Let $U \in \mathcal{C}(2)$, then up to a global phase $U$ may be composed of $H$, $S$, and $CNOT$ gates.
\end{theorem}
\begin{proof}
We proceed by induction on $n$, the number of qubits.
For the base case, let $U$ be a single qubit unitary operator that maps $\mathcal{P}(1)$ to itself under conjugation. 

%Since conjugation by $U$ on $P$ essentially acts as an automorphism map, observe that $P, Q \not\in \{iX, iY, iZ\}$, since for example $X^2 = I$ so $(UXU^\dag )^2 = I$, but if $X \mapsto iX = Q$, then we will have $I = (UXU^\dag)^2 = (iX)^2 = -I$, which is obviously not true. Also, we have that $P \neq iI$ and $P \neq I$. Furthermore, since $\mathcal{P}(1)$ is generated by $iXZ$, we can now restrict possibilities of $P$ to $X$ and $Z$, and $Q$ to $\{\pm X, \pm Y, \pm Z \}$.

Now, let $UXU^\dag = Q$ and $UZU^\dag = R$. Then let $F = V^\dag U$, where $V$ comes from Lemma \ref{ConjX} (so $VXV^\dag = Q$). Then $FXF^\dag = X$ and $FZF^\dag = V^\dag R V = R'$ for some $R' \in \mathcal{P}(1)$. Moreover, by Lemma \ref{ConjZ}, there exists a $G$ such that $GR'G^\dag = Z$ and $GXG^\dag = X$. So:
\[ 
G FXF^\dag G^\dag = G XG^\dag = X  \quad \mbox{ and } \quad
G FZF^\dag G^\dag = G R' G^\dag = Z 
\].
Note $\mathcal{P}(1)$ is generated by $X$, $Z$, and $i$, and neither $Q$, nor $R$ can be $i$ since $U$ is an automorphism. Thus we know how $U$ acts on the generators of $\mathcal{P}(1)$, so $U = VG^\dag$. From this $U$ is comprised of only $H$ and $S$ gates as desired.

Now, suppose $U$ is an $n+1$ qubit gate in $N(G_{n+1})$ such that $UZ_1U^{\dag} = X \otimes g$ and $UX_1U^{\dag} = Z \otimes g'$ for some $g, g' \in G_n$.
Define a circuit $C$ as:
\begin{center}
$
\Qcircuit @C=0.5em @R=.7em {
   & \qw & \qw & \ctrl{1} & \qw & \gate{H} & \qw & \ctrl{1} & \qw  \\
   & {/} \qw & \qw & \gate{g'} & \qw & \qw & \qw & \gate{g} & \qw  
   }
$
\end{center}


Now observe that $CZ_1C^{\dag} = X \otimes g$ as follows:
\begin{align*}
CZ_1C^\dag = \hspace{2mm} & \Qcircuit @C=0.5em @R=.7em {
   & \qw & \ctrl{1} & \qw & \gate{H} & \qw & \ctrl{1} & \qw & \gate{Z} & \qw & \ctrl{1} & \gate{H} & \ctrl{1} & \qw  \\
   & \qw & \gate{g} & \qw & \qw & \qw & \gate{g'} & \qw & \qw & \qw & \gate{g'} & \qw & \gate{g} & \qw
   } \\ 
= \hspace{2mm} & \Qcircuit @C=0.5em @R=.7em {
   & \qw & \ctrl{1} & \qw & \gate{H} & \qw & \ctrl{1} & \qw & \ctrl{1} & \qw & \gate{Z} & \gate{H} & \ctrl{1} & \qw  \\
   & \qw & \gate{g} & \qw & \qw & \qw & \gate{g'} & \qw & \gate{g'} & \qw & \qw & \qw & \gate{g} & \qw
   } \\ 
= \hspace{2mm} & \Qcircuit @C=0.5em @R=.7em {
   & \qw & \ctrl{1} & \qw & \gate{H} & \qw & \gate{Z} & \gate{H} & \ctrl{1} & \qw  \\
   & \qw & \gate{g} & \qw & \qw & \qw & \qw & \qw & \gate{g} & \qw
   } \\ 
= \hspace{2mm} & \Qcircuit @C=0.5em @R=.7em {
   & \qw & \ctrl{1} & \qw & \gate{X} & \ctrl{1} & \qw  \\
   & \qw & \gate{g} & \qw & \qw & \gate{g} & \qw 
   } \\ 
= \hspace{2mm} & \Qcircuit @C=0.5em @R=.7em {
   & \qw & \gate{X} & \qw & \ctrlo{1} & \ctrl{1} & \qw  \\
   & \qw & \qw & \qw & \gate{g} & \gate{g} & \qw 
   } \\ 
= \hspace{2mm} & \Qcircuit @C=0.5em @R=.7em {
   & \qw & \gate{X} & \qw  \\
   & \qw & \gate{g} & \qw 
   } 
\end{align*}


Note that $g$ and $g'$ commute, as can be seen below:
\begin{align}
XZ \otimes gg' &= (X \otimes g)(Z \otimes g') \\
 &= UZ_1U^{\dag}UX_1U^{\dag} \\ 
 &= -UX_1Z_1U^\dag \\
 &= -UX_1U^\dag UZ_1U^\dag \\ 
 &= -(Z \otimes g')(X \otimes g) \\
 &= -(ZX \otimes g'g) \\
 &= XZ \otimes g'g \\
\end{align}
Then we have that $I \otimes gg' = I \otimes g'g$, so applying the inverse of $I \otimes gg'$ to both sides, we have:
\begin{align}
I_{n+1} &= (I \otimes g'g)(I \otimes gg')^{-1} \\
 &= (I \otimes g'g)(I \otimes g'^{-1}g^{-1}) \\ 
 &= I \otimes g'gg'^{-1}g^{-1} \\
\end{align}
where $I_{n+1}$ is the identity on $n+1$ qubits. This implies that $g'gg'^{-1}g^{-1} = I_n$, thus $g'g = gg'$.

Now we make use of this commutation to show that $CX_1C^\dag = Z_1 \otimes g'$:


\begin{align*}
CX_1C^\dag = \hspace{2mm} & \Qcircuit @C=0.5em @R=.7em {
   & \qw & \ctrl{1} & \qw & \gate{H} & \qw & \ctrl{1} & \qw & \gate{X} & \qw & \ctrl{1} & \gate{H} & \ctrl{1} & \qw  \\
   & \qw & \gate{g} & \qw & \qw & \qw & \gate{g'} & \qw & \qw & \qw & \gate{g'} & \qw & \gate{g} & \qw
   } \\ 
= \hspace{2mm} & \Qcircuit @C=0.5em @R=.7em {
   & \qw & \ctrl{1} & \qw & \gate{H} & \qw & \ctrl{1} & \qw & \ctrlo{1} & \qw & \gate{X} & \gate{H} & \ctrl{1} & \qw  \\
   & \qw & \gate{g} & \qw & \qw & \qw & \gate{g'} & \qw & \gate{g'} & \qw & \qw & \qw & \gate{g} & \qw
   } \\ 
= \hspace{2mm} & \Qcircuit @C=0.5em @R=.7em {
   & \qw & \ctrl{1} & \qw & \gate{H} & \qw & \gate{X} & \gate{H} & \ctrl{1} & \qw  \\
   & \qw & \gate{g} & \qw & \qw & \qw & \gate{g'} & \qw & \gate{g} & \qw
   } \\ 
= \hspace{2mm} & \Qcircuit @C=0.5em @R=.7em {
   & \qw & \ctrl{1} & \qw & \gate{Z}  & \qw & \ctrl{1} & \qw  \\
   & \qw & \gate{g} & \qw  & \gate{g'} & \qw & \gate{g} & \qw
   } \\ 
= \hspace{2mm} & \Qcircuit @C=0.5em @R=.7em {
   & \qw & \gate{Z} & \qw & \ctrl{1} & \ctrl{1} & \qw  \\
   & \qw & \gate{g'} & \qw & \gate{g} & \gate{g} & \qw 
   } \\ 
= \hspace{2mm} & \Qcircuit @C=0.5em @R=.7em {
   & \qw & \gate{Z} & \qw  \\
   & \qw & \gate{g'} & \qw 
   } 
\end{align*}


Consider $U':= C^\dag U$. Then 
\begin{center}
$U'X_1U'^\dag = C^\dag U X_1 U^\dag C = C^\dag (Z \otimes g') C = X_1$
\end{center} and 
\begin{center}
$U'Z_1U'^\dag = C^\dag U Z_1 U^\dag C = C^\dag (X \otimes g) C = Z_1$
\end{center}
so $U' \in \mathcal{C}(2)$.

Now let $U \in \mathcal{C}(n+1)$ so $UX_1U^\dag = P$ and $UZ_1U^\dag = Q$ for some $P, Q \in \mathcal{P}(n+1)$. Then we can write $P = A \otimes g$ and $Q = B \otimes g'$ for $g, g' \in \mathcal{P}(n)$. From the inductive step, we can conjugate $UX_1U^\dag = P$ with some circuit $C$, which is comprised of only Hadamards and phase gates, such that 
\begin{center}
$CUX_1U^\dag C^\dag = CPC^\dag = C(A \otimes g)C^\dag = Z \otimes g$
\end{center}
and 
\begin{center}
$CUZ_1U^\dag C^\dag = CQC^\dag = C(B \otimes g')C^\dag = X \otimes g'$
\end{center}
Let $U' = CU$, then by the inductive step, $U' \in \mathcal{C}(n)$. Next we repeat this process on $U'$ to get a $U'' = C'U' \in \mathcal{C}(n-1)$. Then $U = C^\dag U'$, but $U' = C'^\dag U''$, so $U = C^\dag C'^\dag U''$. Continuing this process results in $U$ being a product of components comprised of only $CNOT$, $H$, and $S$ gates as desired (note the final $U'$ will be a one-qubit Clifford so we know it is comprised of only $H$ and $S$ gates from the base case).
\end{proof}



%This brings us to what we call the Clifford hierarchy. The Pauli group defines the $1st-level$ of the Clifford Hierarchy, which is $\mathcal{C}^{(1)} = e^{i\theta} \cdot \{I, X, Y, Z\}^{\otimes n}$, so it is $\mathcal{P}(1)$ but with all phases. The aforementioned group of automorphisms of the Pauli group is the $2nd-level$ of the Clifford Hierarchy, $\mathcal{C}^{(2)}$, and is also known as both the Clifford Group and the normalizer of $\mathcal{P}(n)$. We now define the $k^{th}-level$ of the Clifford Hierarchy, and follow it with an aforementioned theorem.
%
%\begin{definition} The k-th level of the Clifford Hierarchy is:
%\label{CliffHierarchy}
%\begin{equation}
%\mathcal{C}^{(k)} = \{ U | UPU^{\dag} \in \mathcal{C}^{(k-1)}, \forall P \in \mathcal{P}(n) \}
%\end{equation}
%\end{definition}
%
%
%To prove this theorem, we need the following two lemma's.
%

Both Lemma \ref{ConjX} and Lemma \ref{ConjZ} can be generalized to show that the Clifford group is the normalizer of the Pauli group.

\begin{proposition}
If $P \in \mathcal{P}(n)$ and $C \in \mathcal{C}(n)$ then $CPC^{\dag} \in \mathcal{P}(n)$.
\end{proposition}
\begin{proof}
By computation.
\end{proof}



\section{Universal extensions of the Clifford gates}
We can do many interesting things with the Clifford circuits, namely creating  superpositions and entanglement of states. A superposition is a linear combination of single ket vectors, so simply applying $H$ to $\ket{0}$ would give $\ket{+}=(\ket{0}+\ket{1})/\sqrt{2}$. To achieve entanglement means to find some circuit that when applied to some un-entangled qubits the output state is an entangled state. As an example, consider the entangled state $\frac{\ket{00}+\ket{11}}{\sqrt{1}}$, then the following circuit with input $\ket{00}$ with output the entangled state.

\begin{equation*}
\Qcircuit @C=0.8em @R=.8em {
   \lstick{\ket{0}} & \qw & \gate{H} & \ctrl{1} & \qw & & & \raisebox{-8ex}{$\Bigg\}\frac{|00\rangle + |00\rangle}{\sqrt{2}}$} \\
   \lstick{\ket{0}} & \qw & \qw & \targ & \qw & & &
   } 
\end{equation*}


Note the Clifford group is finite. This comes from the fact that it maps Pauli's to Pauli's under conjugation, effectively meaning that Clifford circuits act as permutations of the Pauli group. Since there are $4^{n+1}$ Pauli's, consider a fixed Pauli. Then there are $|\mathcal{P}(n)| = 4^{n+1}$ Clifford circuits that permute this Pauli into all other possible Pauli's. Repeat this for each Pauli to get $|\mathcal{C}(n)| \leq 4^{(n+1)^{4^{n+1}}}$. And in fact this simplifies to ...

Moreover, the Gottesman Knill theorem shows a relation in power between the quantum computer and classical computer.

\begin{theorem}[Gottesman-Knill theorem]
Suppose a quantum computation is performed which involves only the following elements: state preparations in the computational basis, Hadamard gates, phase gates, controlled-NOT gates, Pauli gates, and measurements of observables in the Pauli group, together with the possibility of classical control conditioned on the outcome of such measurements. Such a computation may be efficiently simulated on a classical computer.
\end{theorem}

In short, the Gottesman-Knill theorem essentially states that stabilizer operations can be classically simulated efficiently, meaning in polynomial time. There is an algorithm to do this on a classical computer with $O(n^2m)$ operations, called the Tableau algorithm. It works by keeping track of the generators of the stabilizer, and continuing to update it as various operations are performed throughout the computation. 

This means they are not universal, so there are quantum computations that cannot be simulated effectively with just the Clifford group and measurements. The next theorem provides a fix for this, so we state it here without proof.

\begin{theorem}
If $G$ is a non-Clifford gate then $\{ H, S, CNOT, G \}$ is universal for quantum computing.
\end{theorem}

A typical gate used to extend the Clifford group to be universal is the $T$ gate, though the $CS$ and $CCZ$ gates are two other notable gates that will be used. Recall the matrices for the $T$, $CS$, and $CCZ$ gates:
\begin{equation*}
T = 
\begin{bmatrix}
1 & 0 \\
0 & e^{i\pi/4}
\end{bmatrix}
\hspace{5mm}
CS = 
\begin{bmatrix}
1 & 0 & 0 & 0 \\
0 & 1 & 0 & 0 \\
0 & 0 & 1 & 0 \\
0 & 0 & 0 & i \\
\end{bmatrix}
\end{equation*}
and

\begin{equation*}
CCZ = 
\begin{bmatrix}
1 & 0 & 0 & 0 & 0 & 0 & 0 & 0 \\
0 & 1 & 0 & 0 & 0 & 0 & 0 & 0 \\
0 & 0 & 1 & 0 & 0 & 0 & 0 & 0 \\
0 & 0 & 0 & 1 & 0 & 0 & 0 & 0 \\
0 & 0 & 0 & 0 & 1 & 0 & 0 & 0 \\
0 & 0 & 0 & 0 & 0 & 1 & 0 & 0 \\
0 & 0 & 0 & 0 & 0 & 0 & 1 & 0 \\
0 & 0 & 0 & 0 & 0 & 0 & 0 & -1 
\end{bmatrix}.
\end{equation*}
Note the $CCZ$ gate is used interchangeably with the Toffoli or $CCX$ gate as they are equal upon simple conjugation by a $H$ gate on the target qubit. Unfortunately analysing quantum circuits that include these gates via the stabilizer formalism is rather inconvenient. To see why these gates are not in the Clifford group, simply observe the following example conjugation calculations from \cite{beverland2019lower}:
\begin{center}
$TXT^{\dag} = \frac{X+Y}{\sqrt{2}}$
\end{center}
\begin{center}
$UX_1U^{\dag} = CNOT_{1,2}S_1S_2^{\dag}CNOT_{1, 2}X_1$
\end{center}
\begin{center}
$VX_1V^{\dag} = X_1 \otimes \frac{I + Z_2 +X_3 -Z_2X_3}{2}$
\end{center}
where we let $U$ and $V$ be the $CS$ and Toffoli or $CCX$ gates respectively, where qubit $1$ is the control and qubit $2$ is the target for the $CS$ gate, and qubits $1$ and $2$ are controls and qubit $3$ is the target for the $CCX$ gate. Fortunately, we are still able to make use of the stabilizer formalism and relate it to these gates, particularly to the $T$ gate, later on.

This naturally gives the sense that non-Clifford gates are a resource that enables full quantum computing, meaning that each non-Clifford gate is an expensive commodity that ideally is used as little as possible. This is also corroborated by the fact that in fault-tolerant quantum computing Clifford gates are typically cheap whereas non-Clifford gates are typically expensive. Thus this is the motivation behind attempting to limit the number of non-Clifford gates used. Finding lower bounds and upper bounds also gives an idea as to whether or not more or less of a certain non-Clifford gate can be used. If there is a circuit the uses the same number of non-Clifford gate as the calculated lower bound, then we know cannot do any better.

\section{Computing with states}
Instead of using non-Clifford gates to perform universal quantum computing, one can use a special kind of state to \emph{inject} the desired gate into a circuit. The useful states for this purpose are defined below.

\begin{definition}
We define the following resource states:
\begin{equation*}
\ket{T} = T\ket{+}, \mbox{ } \ket{CS} = CS \ket{+}^{\otimes 2}, \mbox{ and } \ket{CCZ} = CCZ \ket{+}^{\otimes 3}.
\end{equation*}
\end{definition}

The above states can be used to apply the corresponding gates using the injection circuits given below.




In contrast to resource states, we also define a \emph{stabilizer state}.

\begin{definition}
A \emph{stabilizer state} is a state of the form $\ket{\phi} = C \ket{0}^{\otimes n}$ for some Clifford unitary C. 
\end{definition}

\begin{definition}
Let $\ket{\psi}$ be a quantum state and $U$ be an operator, then we say that $\ket{\psi}$ is \emph{stabilized} by $U$ if $U\ket{\psi} = \ket{\psi}$. 
\end{definition}

\begin{definition}[Stabilizer]
Let $\ket{\psi}$ be a non-zero $n$-qubit state. The \emph{stabilizer} of $\ket{\psi}$ is the subgroup $\mathcal{P}(n)$ for which $\ket{\psi}$ is a $+1$ eigenstate, and is denoted by Stab$\ket{\psi}$. 
\end{definition}

This means that Stab$\ket{\psi}$ = $\{P \in \mathcal{P}(n) \: P \ket{\psi} = \ket{\psi} \}$. States for which the stabilizer contains only the identity matrix are said to have a trivial stabilizer. 

\begin{definition}[Stabilizer States]
Let $\ket{\psi}$ be a state for which the size of its stabilizer is $2^n$. The we call $\ket{\psi}$ a stabilizer state.
\end{definition}

The reason behind using this stabilizer formalism is that we can easily describe many quantum states by working with operators that stabilize them, rather than by explicitly working with the states themselves. A key component to making use of the stabilizer formalism lies in the Pauli group $\mathcal{P}(n)$ defined above. Next are a few propositions regarding the stabilizer.

\begin{proposition}
\label{stabilizerfacts}
Let $\ket{\psi}$ be a non-zero n qubit state. Then we have the following facts about Stab$\ket{\psi}$:
\begin{enumerate}
\item Stab$\ket{\psi}$ does not contain -$I$.
\item All Pauli group elements contained in Stab$\ket{\psi}$ commute with each other and are Hermitian matrices.
\item The size of the stabilizer is equal to some power of two.
\item Given any Clifford Unitary $C$, the size of Stab$\ket{\psi}$ is always equal to the size of Stab$(C \ket{\psi})$.
\item Finally, the size of the stabilizer is multiplicative for the tensor products of states, that is $| \text{Stab} (\ket{\psi} \ket{\phi})| = |\text{Stab}\ket{\psi}| \cdot |\text{Stab}\ket{\phi}|$. 
\end{enumerate}
\end{proposition}

\begin{proof}
\hspace{20mm}
\begin{enumerate}
\item If $\-I \in \text{Stab}\ket{\psi}$, then $-\ket{\psi} =-I\ket{\psi} = \ket{\psi}$, which of course is not true for non-zero states.

\item First note that for any two Pauli's $P, Q$, they either commute or anti-commute. Now suppose $P, Q \in \text{Stab}\ket{\psi}$ anti-commute. Then $\ket{\psi} = PQ\ket{\psi} = -QP\ket{\psi} = -\ket{\psi}$. This implies that $-I \in \text{Stab}\ket{\psi}$, which from above can't be true, so $P$ and $Q$ must commute.

\item It is known that the Pauli group's cardinality is a power of two, and since $\text{Stab}\ket{\psi}$ is a subgroup of the Pauli group, $|\text{Stab}\ket{\psi}|$ must divide a power of two, thus it must also be a power of two.

\item First note that Clifford unitaries normalize pauli matrices, i.e. for some Clifford unitary $C$, and some pauli $P$, $CPC^{\dag} = P'$, where $P'$ is also a pauli. Now let $P \in \text{Stab}\ket{\psi}$ and let $C$ be some Clifford unitary. Then $P'C\ket{\psi} = CPC^{\dag}C\ket{\psi} = CP\ket{\psi} = C\ket{\psi}$, so $P' \in \text{Stab}(C\ket{\psi})$. Now consider the map $\theta_C : \text{Stab}\ket{\psi} \rightarrow \text{Stab}(C\ket{\psi})$ which takes elements $P \mapsto CPC^{\dag} = P'$. This map has an inverse, $\theta_{C^\dag}: \text{Stab}(C\ket{\psi}) \longrightarrow \text{Stab}(C^{\dag}C\ket{\psi})$, which takes elements $P' \mapsto C^{\dag}P'C$ (where we note that $\text{Stab}(C^{\dag}C\ket{\psi}) = \text{Stab}\ket{\psi}$). Thus $\theta_C$ is a bijection, and so we have that $|\text{Stab}\ket{\psi}| = |\text{Stab}(C\ket{\psi})|$.

\item Let $\ket{\phi}$ be another non-zero state on n qubits, and let $P \in \text{Stab}\ket{\psi}$ and $Q \in \text{Stab}\ket{\phi}$. Then $P \otimes Q \ket{\psi}\ket{\phi} = P\ket{\psi} \otimes Q\ket{\phi} = \ket{\psi}\ket{\phi}$. So $P \otimes Q \in \text{Stab}\ket{\psi}\ket{\phi}$. Now let $R \in \text{Stab}\ket{\psi}\ket{\phi}$. Then since $R$ is a Pauli, we can write $R = R_1 \otimes R_2$, and $R\ket{\psi}\ket{\phi} = \ket{\psi}\ket{\phi} = R_1 \otimes R_2 \ket{\psi}\ket{\phi} = R_1\ket{\psi} \otimes R_2\ket{\phi}$. Now let $\ket{\psi '} = R_1\ket{\psi}$ and $\ket{\phi '} = R_2\ket{\phi}$, and since $R_1$ and $R_2$ are both Clifford unitaries, fact 4 gives us that $|\text{Stab}\ket{\psi '}| = |\text{Stab}(R_1\ket{\psi})|$ and $|\text{Stab}\ket{\phi '}| = |\text{Stab}(R_2\ket{\phi})|$. We also have that $|\text{Stab}(R\ket{\psi}\ket{\phi})| = |\text{Stab}\ket{\psi '}\ket{\phi '}|$.


 So every element in $\text{Stab}\ket{\psi}\ket{\phi}$ is of the form $R_1 \otimes R_2$ as above, thus $\text{Stab}\ket{\psi}\ket{\phi} = \text{Stab}\ket{\psi} \otimes \text{Stab}\ket{\phi}$. Then we have a bijection (from the direct product) $\theta:\text{Stab}\ket{\psi} \times \text{Stab}\ket{\phi} \longrightarrow  \text{Stab}\ket{\psi} \otimes \text{Stab}\ket{\phi}$, which gives us $|\text{Stab}\ket{\psi}\ket{\phi}| = |\text{Stab}\ket{\psi} \otimes \text{Stab}\ket{\phi}| = |\text{Stab}\ket{\psi}| \cdot |\text{Stab}\ket{\phi}|$.
\end{enumerate}
\end{proof}



An example of a stabilizer state is the $\ket{0}$ state, since there are $2^1$ pauli's that stabilize it, namely $I$ and $Z$. Note that, from fact 4 in Proposition \ref{stabilizerfacts}, for any Clifford $C$, $|\text{Stab}\ket{0}| = |Stab(C\ket{0})| = 2^1 \implies C\ket{0}$ is a stabilizer state. And an example of a non-stabilizer state is $\ket{\psi} = \frac{1}{\sqrt{2}}(\ket{00} + \frac{i}{2} \ket{11})$, which through computation one can find that it has the following stabilizers: $I \otimes I$ and $Z \otimes Z$. Since there are only $2$ stabilizers, and $2 \neq 2^2$, $\ket{\psi}$ cannot be a stabilizer state.

\begin{corollary}
The computational basis state $\ket{00 \dots 0}$ on n-qubits is a stabilizer state. If $\ket{\psi}$ is a basis state, then it is a stabilizer state. 
\end{corollary}
\begin{proof}
First we prove that $\ket{00 \dots 0}$ is a stabilizer state by induction, where the base case is the $\ket{0}$ state which we know is a stabilizer state from above (and $|\text{Stab}\ket{0}| = 2^1$). Now assume that $\ket{00 \dots 0}$ is a stabilizer state on n-qubits with $|\text{Stab}\ket{00 \dots 0}| = 2^n$. Let $P \in \text{Stab}\ket{0}$ and $Q \in \text{Stab}\ket{00 \dots 0}$, then $(P \otimes Q)\ket{0}\ket{00 \dots 0} = P\ket{0} \otimes Q\ket{00 \dots 0} = \ket{0}\ket{00 \dots 0} = \ket{00 \dots 00}$ (n+1 qubits). So $P \otimes Q \in \text{Stab}\ket{00 \dots 00}$, and from above arguments we know that every element in $\text{Stab}\ket{00 \dots 00}$ is of this form, thus there are $2^{n+1}$ elements in $\text{Stab}\ket{00 \dots 00}$, making it a stabilizer state.


Next, if $\ket{\psi}$ is an $n$-qubit basis state, then it differs from the $n$-qubit computational basis state $\ket{00 \ldots 0}$ by only a Clifford unitary, i.e. $C\ket{00 \ldots 0} = \ket{\psi}$ for some Clifford circuit C. Then as remarked above, it follows that $\ket{\psi}$ is a stabilizer state.
\end{proof}

\begin{theorem}
If $\ket{\psi}$ is a stabilizer state, then there is a circuit $C$ such that $C\ket{\psi} = \ket{00 \dots 0}$.
\end{theorem}
\begin{proof}
By Theorem 8 in \cite{}, every stabilizer state can be represented by a tableau, which can then be converted into a tableau that represents the $\ket{00 \dots 0}$ state using only Clifford operations. Applying these Clifford operators is equivalently applying Clifford operators to the appropriate qubits of $\ket{\psi}$, thus resulting in a Clifford circuit $C$ such that $C\ket{\psi} = \ket{00 \dots 0}$. 
\end{proof}

\begin{theorem}
If $\ket{\psi}$ is an $n$-qubit stabilizer state, then $\textit{Stab}\ket{\psi}$ uniquely determines $\ket{\psi}$.
\end{theorem}
\begin{proof}
First note that each $P \in \textit{Stab}\ket{\psi}$ imposes the linear constraint that $P\ket{\psi} = \ket{\psi}$ on the $2^n$-dimensional vector space. The subspace of vectors that satisfy this constraint has half of this space, meaning that it has dimension $2^{n-1}$. Let gen($\ket{\psi}$) be the set of generators for $\textit{Stab}\ket{\psi}$, where we add independent generators to gen($\ket{\psi}$) one by one. This way we subsequently half the dimension of the vector subspace that satisfies all the linear constraints, and since $|\textit{gen}(\ket{\psi})| = n$, this dimension reduces to $1$. Thus there is only one possible vector and so $\ket{\psi}$ is uniquely determined.
\end{proof}

\begin{definition}
Let $S$ be a subgroup of $\mathcal{P}(n)$, and let $V_S$ be the set of $n$ qubit states which are fixed by every element of $S$. $V_S$ is the vector space stabilized by $S$, and we call $S$ the stabilizer of the space $V_S$. 
\end{definition}

We can also make use of the Pauli groups generators, where to check if a particular vector is stabilized by a group $S$, checking if the generators stabilize it will be sufficient.


\begin{proposition}
Let $S$ be a subgroup of $\mathcal{P}(n)$ and $V_S$ its associated vector space stabilized by $S$. Then if $V_S$ is not trivial $S$ is a commutative group that does not contain $-I$.
\end{proposition}
\begin{proof}
Consider the subgroup $\{\pm I, \pm X \}$ of $\mathcal{P}(1)$. Obviously the only solution to $(-I)\ket{\psi} = \ket{\psi}$ is $\ket{\psi} = 0$, and so $\{\pm I, \pm X \}$ stabilizes the trivial vector space. An immediate consequence of this is that we need $-I$ to not be an element of $S$. Furthermore, all elements of $S$ must commute, because if they did not then they would anti-commute, so $-I$ would necessarily be an element of the group.
\end{proof}



%\begin{proposition}
%Let $S = \langle g_1, \dots , g_{n-k} \rangle$ be generated by $n-k$ independent and commuting elements from $\mathcal{P}(n)$, and such that $-I \not \in S$. Then $V_S$ is a $2^k$ - dimensional vector space.
%\end{proposition}
%
%
%
%\section{Relating Unitaries}
%Now we describe how unitaries relate to the stabilizer formalism, starting with a proposition. 
%\begin{proposition}
%$Stab(\ket{\psi}) = \langle g_1 , \dots , g_n \rangle \Longrightarrow Stab(U\ket{\psi}) = \langle Ug_1U^{\dag}, \dots , Ug_lU^{\dag} \rangle$
%\end{proposition}
%\begin{proof}
%First, suppose we apply a unitary operations $U$ to a vector space $V_S$ stabilized by the subgroup $S$ as before, and let $\ket{\psi} \in V_S$. Note that $S = Stab(\ket{\psi})$. Then for any $g \in S$, $U\ket{\psi} = Ug\ket{\psi} = UgU^{\dag}U\ket{\psi}$. Thus, the state $U\ket{\psi}$ is stabilized by $UgU^{\dag}$, and so the vector space $UV_S$ is stabilized by the group $USU^{\dag}$. Furthermore, if $g_1, \dots , g_l$ generate $S$, then $Ug_1U^{\dag}, \dots , Ug_lU^{\dag}$ generates $USU^{\dag}$.
%\end{proof}
%
%There is a particularly appealing and advantageous form that special unitary operations take on. For us, the Hadamard and phase gates are nice examples to consider. We define them now, along with the CNOT gate which will be used later on.
%
%\begin{definition}
%The Hadamard ($H$), phase ($S$) and controlled-NOT (CNOT) gates are defined as follows:
%\begin{equation}
%H = 
%\frac{1}{\sqrt{2}} 
%\begin{bmatrix}
%1 & 1 \\
%1 & -1
%\end{bmatrix}, \quad
%S = 
%\begin{bmatrix}
%1 & 0 \\
%0 & i
%\end{bmatrix}, \quad \mbox{and} \quad
%CNOT = 
%\begin{bmatrix}
%1 & 0 & 0 & 0 \\
%0 & 1 & 0 & 0 \\
%0 & 0 & 0 & 1 \\
%0 & 0 & 1 & 0
%\end{bmatrix}.
%\end{equation}
%\end{definition}





%\section{Measurements}
%Yet another benefit of the stabilizer formalism shows up in the topic of measurements, specifically because we can quite easily describe measurements in the computational basis. Consider a $g \in \mathcal{P}(n)$, and imagine we make a measurement on it. Recall this means that $g$ is a Hermitian operator and is a product of Pauli's, and also for convenience let it not have any multiplicative factor of $-1$ or $\pm i$. Letting our system be in some state $\ket{\psi}$ with stabilizer $\langle g_1, \dots , g_n \rangle $, how does the stabilizer of the state transform under measurement? 
%
%\begin{proposition}
%Given a state $\ket{\psi}$ and Pauli $g$, after measurement the state transforms into $(I + g)\ket{\psi}/\sqrt{2}$ when the $+1$ result occurs, and $(I - g)\ket{\psi}/\sqrt{2}$ when the $-1$ result occurs, with probabilities $p(+1) = p(-1) = 1/2$.
%\end{proposition}
%
%\begin{proof}
%We have two possible scenario's to consider, the first of which is when $g$ commutes with all generators. Observe that $g_jg\ket{\psi} = gg_j\ket{\psi} = g\ket{\psi}$ for each stabilizer generator, so $g\ket{\psi}$ is in $V_S$ and is thus a multiple of $\ket{\psi}$. Because $g^2 = I$, $g\ket{\psi} = \pm\ket{\psi}$, so either $g$ or $-g$ must be in the stabilizer. It follows that either $g$ or $-g$ is an element of the stabilizer. Suppose $g$ is in the stabilizer, then $g\ket{\psi} = \ket{\psi}$ and so a measurement of $g$ yields $+1$ with probability one, and the measurement leaves the stabilizer invariant. The argument for $-g$ being in the stabilizer follows similarly.
%
%The second scenario is when $g$ anti-commutes with one or more generators. Without loss of generality let $g$ anti-commute with $g_1$, and commute with all other generators. Note that if $g$ did anti-commutes with another generator, say $g_2$, then it does commute with $g_1g_2$, and so you could just replace the generator $g_2$ with $g_1g_2$ in the stabilizer of $\ket{\psi}$. Also, $g$ has eigenvalues $\pm 1$ and so the projectors for the measurements outcomes are $(I \pm g)/2$ respectively, so we have measurement probabilities:
%\begin{center}
%$p(+1) = tr(\frac{I+g}{2}\ket{\psi}\bra{\psi})$
%\end{center}
%and 
%\begin{center}
%$p(-1) = tr(\frac{I-g}{2}\ket{\psi}\bra{\psi})$
%\end{center}
%Now, using the equalities $g_1\ket{\psi} = \ket{\psi}$ and $gg_1 = -g_1g$, and the cyclic property of the trace, we get:
%\begin{center}
%$p(+1) = tr(\frac{I+g}{2}g_1\ket{\psi}\bra{\psi}) = tr(g_1\frac{I-g}{2}g_1\ket{\psi}\bra{\psi}) = tr(\frac{I-g}{2}\ket{\psi}\bra{\psi}g_1)$
%\end{center}
%Since $g_1 = g_1^{\dag}$ and $\bra{\psi}g_1 = \bra{\psi}$ we have that:
%\begin{center}
%$p(+1) = tr(\frac{I-g}{2}\ket{\psi}\bra{\psi}) = p(-1)$.
%\end{center}
%Moreover, $p(+1) + p(-1) = 1$, so $p(+1) = p(-1) = 1/2$. So when the $+1$ result occurs, the new state of the system will be $(I + g)\ket{\psi}/\sqrt{2}$ with stabilizer $\langle g, g_2, \dots , g_n \rangle$ and probability $1/2$. And similarly, when the $-1$ result occurs, the state will be $(I - g)\ket{\psi}/\sqrt{2}$ with stabilizer $\langle -g, g_2, \dots , g_n \rangle$ and probability $1/2$.
%\end{proof}
%
%It is important to note that circuits made from only gates in $\mathcal{C}^{(2)}$ are not universal for quantum computing. To get around this issue, we can use gates from the $3rd-level$ of the Clifford Hierarchy: $\mathcal{C}^{(3)} = \{ U | UPU^{\dag} \in \mathcal{C}^{(2)}, \forall P \in \mathcal{P}(n)$, and so we will use unitaries from $\mathcal{C}^{(3)}$ frequently.




\chapter{Monotones}
In this chapter, we introduce \emph{monotones}, which are real-valued functions of states. In the next chapter, we will use these monotones to derive lower bounds for quantum circuits. 

\section{Abstract Monotones}
Recall that S is the collection of all states. We say that a collection of states $I \subseteq S$ is \emph{closed under tensor products} if $\ket{\phi}, \ket{\psi} \in I$ implies $\ket{\phi} \otimes \ket{\psi} \in I$. Similarly, we say that I is \emph{closed under the action of stabilizers} if $\ket{\phi} \in I$ and $C \in \mathcal{C}(n)$ implies $C\ket{\phi} \in I$.

%$\mathcal{S}$ is the set of $n$-qubit (pure) states $\mathcal{S} = \cup_n \mathcal{S}(n)$.

\begin{definition}
Let $\mathcal{J} \subseteq \mathcal{S}$ and suppose that $\mathcal{J}$ is closed under tensor products and the action of stabilizers. A \emph{monotone} for $\mathcal{J}$ is a function $M: \mathcal{J} \rightarrow \mathbb{R}^{\geq 0}$ such that $M(\ket{\phi}) = 0$ if $\ket{\phi}$ is a stabilizer state and if $C$ is a stabilizer operator then $M(C\ket{\phi}) \leq M(\ket{\phi})$ for all $\ket{\phi} \in \mathcal{J}$. Also, $M(\ket{\phi}) \geq 0$ and $M(\ket{\phi}) = 0 \Leftrightarrow \ket{\phi} \in Stab(\phi)$.
\end{definition}

\begin{definition} 
We have the following properties for monotones:
\begin{itemize}
\item $M(\ket{\phi} \otimes \ket{\psi}) = M(\ket{\phi}) + M(\ket{\psi})$ (additive) 
\item $M(\ket{\phi} \otimes \ket{\psi}) = M(\ket{\phi}) \cdot M(\ket{\psi})$ (multiplicative)
\item $M(C\ket{\phi}) \leq M(\ket{\phi})$ where C is a Pauli measurement (non-increasing under Pauli measurements)
\item $M(\ket{\phi}) = M(\ket{\phi}\ket{\psi})$ for all stabilizer states $\ket{\psi}$ (stable under stabilizer ancillas)
\end{itemize}
\end{definition}



%Note: for the above properties to hold, $\mathcal{J}$ must contain Stabilizer states, measurements, and be closed under Clifford operations and tensor products.
%
%We make note here that the stabilizer nullity has properties $1-6$ except $5$, the stabilizer has property $4$ for certain states, and the dyadic monotone has properties $1, 2, 3, 4, 6$ and a version of property $5$ for certain measurements. Also, observe that if a monotone has property $1$ and $3$, then it immediately has property $6$.

\section{Stabilizer Nullity}


\begin{definition}[Stabilizer Nullity]
Let $\ket{\psi}$ be a non-zero n-qubit state. The \emph{Stabilizer Nullity} of $\ket{\psi}$ is $\nu(\ket{\psi}) = n - \log_2 |\text{Stab}\ket{\psi}|$.
\end{definition}

\begin{proposition}
The stabilizer nullity is an additive monotone.
\end{proposition}
\begin{proof}
This follows from the properties of the stabilizer. To show the $\nu$ is a monotone, recall that for a Clifford unitary $C$ and a state $\ket{\psi}$, $\Stab(C\ket{\psi}) = \Stab\ket{\psi}$. Thus $\nu(C\ket{\psi}) = n - \log_2|\Stab(C\ket{\psi})| = n - \log_2|\Stab\ket{\psi}| = \nu(\ket{\psi})$. If $\ket{\psi}$ is a stabilizer state, then $\Stab\ket{\psi} = 2^n$, so $\nu(\ket{\psi}) = n - \log_2(\Stab\ket{\psi}) =  n - \log_2(2^n) = 0$. Finally, note that $\Stab\ket{\psi}$ is never more than $2^n$, which means $\nu$ is always a positive integer, making it a real valued function.

For the additive part, recall that the stabilizer is multiplicative under tensor products, i.e. $|\Stab(\ket{\psi} \ket{\phi})| = |\Stab\ket{\psi}| \cdot |\Stab{\ket{\phi}}|$. Letting $\ket{\psi}$ and $\ket{\phi}$ be size $n$ and $m$ vectors respectively, we have that 
\begin{equation*}
\begin{split}
\nu (\ket{\psi}\ket{\phi}) & = n + m - \log_2 |\Stab (\ket{\psi}\ket{\phi})| \\
& = n + m - \log_2 |\Stab \ket{\psi}| - \log_2|\Stab|\ket{\phi}| \\
& = \nu(\ket{\psi}) + \nu (\ket{\phi}).
\end{split}
\end{equation*}

\end{proof}

\begin{proposition}
Let $\ket{\psi}$ be a non-zero n-qubit state and let $P$ be an n-qubit Pauli matrix and suppose that the probability of a $+1$ outcome when measuring $P$ on $\ket{\psi}$ is non-zero. Then there are two alternatives for the state $\ket{\phi}$ after measurement: either $|\text{Stab}\ket{\phi}| = |\text{Stab}\ket{\psi}|$ or $|\text{Stab}\ket{\phi}| \geq 2|\text{Stab}\ket{\psi}|$, both of which satisfy $\nu(\ket{\phi}) \leq \nu(\ket{\psi})$.
\end{proposition}

\begin{proof}
First consider the simple case when $P$ is in Stab$\ket{\psi}$. In this case, the "+1" measurement outcome occurs with probability $1$ and $\ket{\psi}$ is unchanged. When $P$ is not in Stab$\ket{\psi}$ we consider two alternatives. The first alternative is that $P$ commutes with all elements of Stab$\ket{\psi}$. Recall we have the post-measurement state $\ket{\phi} = \frac{P\ket{\psi}}{\sqrt{\bra{\psi}P^{\dag}P\ket{\psi}}} = \frac{P\ket{\psi}}{\sqrt{\bra{\psi}\ket{\psi}}}$, and let $Q \in \text{Stab}\ket{\psi}$. Then $Q\ket{\phi} = \frac{QP\ket{\psi}}{\sqrt{\bra{\psi}\ket{\psi}}} = \frac{PQ\ket{\psi}}{\sqrt{\bra{\psi}\ket{\psi}}} = \frac{P\ket{\psi}}{\sqrt{\bra{\psi}\ket{\psi}}} = \ket{\phi}$. Note also that Stab$\ket{\phi}$ also contains $P$Stab$\ket{\psi}$, and thus Stab$\ket{\phi}$ contains Stab$\ket{\psi}$ $\cup$ $P$Stab$\ket{\psi}$ and thus its size is at least $2| \text{Stab}\ket{\psi}|$.

The second alternative is when $P$ anti-commutes with some element $Q \in$ Stab$\ket{\psi}$. Note that $Q\ket{\psi} = \ket{\psi}$ and $QPQ = -P$, so the probability of the $+1$ outcome is $\bra{\psi}(I+P)\ket{\psi}/2 = \bra{\psi}Q(I+P)Q\ket{\psi}/2 = \bra{\psi}(I-P)\ket{\psi}/2$, which is the probability of the $-1$ outcome. Thus the probability of the $+1$ outcome is $1/2$. Then $\ket{\phi} = (I+P)/\sqrt{2}\ket{\psi}$ where we fixed the normalization condition such that $\braket{\phi} = \braket{\psi}$. Also, observe that we can write $\ket{\phi} = (I+PQ)/\sqrt{2}\ket{\psi}$. Since $(I+PQ)/\sqrt{2}$ is a Clifford unitary equal to exp($i\pi P'/4)$ for $P' = iPQ$, we see that $\ket{\phi}$ and $\ket{\psi}$ differ by a Clifford and therefore $|$Stab$\ket{\psi}| = |$Stab$\ket{\phi}|$.
\end{proof}

\begin{corollary}
The stabilizer nullity is non-increasing under Pauli measurements, i.e. for a state $\ket{\psi}$ and it's post measurement state $\ket{\phi}$ we have that $\nu(\ket{\phi}) \leq \nu(\ket{\psi})$.
\end{corollary}



\begin{definition}[Pauli Spectrum]
Let $\ket{\psi}$ be a non-zero n-qubit state. The Pauli spectrum Spec$\ket{\psi}$ of $\psi$ is:
\begin{equation}
\text{Spec}\ket{\psi} = \left\{ \frac{|\bra{\psi}P\ket{\psi}|}{\braket{\psi|\psi}}, \forall P \in \{I, X, Y, Z\}^{\otimes n} \right\}
\end{equation}
The Pauli spectrum is a list of $4^n$ real numbers each between 0 and 1 which is invariant under Clifford gates. Consider the following example.
\end{definition}

\begin{proposition}
The Pauli spectrum of the state $\ket{\theta} = (\ket{0} + e^{i\theta}\ket{1})/\sqrt{2}$ is $\{1, cos\theta, sin\theta, 0\}$. The state $\ket{\theta}$ is therefore a stabilizer state only for $\theta = m \pi/2$ for some integer m.
\end{proposition}
\begin{proof}
First note that $\ket{\theta}$ is normalized so $\bra{\theta}\ket{\theta} = 1$. Now by direct computation, we have:
\begin{itemize}
\item $\bra{\theta}I\ket{\theta} = \braket{\theta|\theta} = 1$
\item $\bra{\theta}X\ket{\theta} = (\bra{1}e^{-i\theta} + \bra{0})(\ket{1} + e^{i\theta}\ket{0})/2 = (e^{-i\theta} + e^{i\theta})/2 = cos\theta$
\item $\bra{\theta}Y\ket{\theta} = (\bra{1}e^{-i\theta} + \bra{0})(i\ket{1} - ie^{i\theta}\ket{0})/2 = i(e^{-i\theta} - e^{i\theta})/2 = i(-2sin\theta)/2 = sin\theta$
\item $\bra{\theta}Z\ket{\theta} = (\bra{1}e^{-i\theta} + \bra{0})(\ket{0} - e^{i\theta}\ket{1})/2 = 1 - 1 = 0$
\end{itemize} 
Moreover, if $\theta = 2k\pi /2$ for some integer $k$, then $X \in \text{Stab}\ket{\theta}$, and if $\theta = (2k+1)\pi /2$, then $Y \in \text{Stab}\ket{\theta}$. Observe that $\forall \theta$, $I \in \text{Stab}\ket{\theta}$ and $Z \not\in \text{Stab}\ket{\theta}$, thus $|\text{Stab}\ket{\theta}| = 2$ if and only if either $X$ or $Y \in \text{Stab}\ket{\theta}$, or more generally if $\theta = m \pi /2$, for some integer $m$.
\end{proof}
Note that the number of 1s in the Pauli spectrum of $\ket{\psi}$ is $|\text{Stab}\ket{\psi}|$. Hence the Pauli spectrum can be useful in computing the stabilizer nullity.

%\section{Stabilizer Extent}
%\begin{definition}
%For an arbitrary pure state $\ket{\psi}$, \textbf{the stabilizer extent} $\xi (\ket{\psi})$ is:
%\begin{equation}
%\xi (\ket{\psi}) = min||(c_1, \dots , c_k)||_1^2 \hspace{4mm} s.t. \hspace{3mm} \ket{\psi} = \sum_{\alpha = 1}^k c_{\alpha} \ket{\phi_{\alpha}}
%\end{equation}
%\end{definition}
%
%The stabilizer extent is sub-multiplicative, but multiplicative in specific cases, as in the following lemma.
%
%\begin{lemma}
%\label{stabextent}
%The stabilizer extent is multiplicative with respect to a given set of states $\{ \ket{\psi_1}, \ket{\psi_2}, \dots \ket{\psi_l} \}$, such that, $\xi (\ket{\psi_1}\ket{\psi_2} \dots \ket{\psi_l}) = \prod_{j=1}^l \xi (\ket{\psi_j})$, if for each state at least one of the following conditions are satisfied (where $\ket{\phi_j}$ is always a stabilizer state):
%\begin{enumerate}
%\item $\ket{\psi_j}$ is a state of at most three qubits
%\item There exist states $\ket{\omega_j}$ and $\ket{\phi_j}$ such that $\xi (\ket{\psi_j}) = \frac{|\braket{\psi_j| \omega_j}|^2}{max_{\phi_j}|\braket{\omega_j |\phi_j}|^2}$ \textbf{and} $|\braket{\omega_j | \phi_j}|^2 \geq 1/4$.
%\end{enumerate}
%\end{lemma}

\section{Dyadic Monotone}
Consider quantum states the have entries in $\mathbb{Z}[i, 1/2] = \big\{\frac{a+ib}{2^k}: a, b, k, \in \mathbb{Z} \big\}$ when written in the computational basis. Indeed, $\ket{C^nZ}$ can be written as vectors with entries in the above set. We have a few note worthy facts to observe here:
\begin{itemize}
\item The set $\mathbb{Z}[i, 1/2]$ is a ring since it is closed under addition, negation, multiplication and contains $0$ and $1$.
\item If a state $\ket{\psi}$ has entries in $\mathbb{Z}[i, 1/2]$ then for any Hermitian multi-qubit Pauli operator $P$, the expectation $\braket{\psi | P | \psi}$ can be written as $a/2^k$ for integers $a, k$, and is in $\mathbb{Z}[i, 1/2]$ because the entries of all Pauli matrices are also, and $\mathbb{Z}[i, 1/2]$ is closed under complex conjugation.
\item The expectation is also a real number and all the real numbers in $\mathbb{Z}[i, 1/2]$ are of the form $a/2^k$ for integers $a$ and $k$. 
\item For stabilizer states Pauli expectations can only be $\pm 1$ and $0$.
\end{itemize}

First let us define a function $\bar{v}_2 :\mathbb{Q} \rightarrow \mathbb{Z} \cup \{ \infty \}$. Let $0 \neq q \in \mathbb{Q}$ so we can write $q = n/d$ with $n, d \in \mathbb{Z}$ and gcd$(n, d) = 1$ (so it is in reduced form). By the fundamental theorem of algebra we can write $n = 2^{r_2} \cdot p^{r_{p_1}}_1 \cdots p^{r_{p_m}}_m$ and $d = 2^{l_2} \cdot p^{l_{p_1}}_1 \cdots p^{l_{p_m}}_m$ for $k_i, l_i \in \mathbb{Z} \forall i \in [m]$. So $q = 2^{k} \cdot p^{k_{p_1}}_1 \cdots p^{k_{p_m}}_m$ where $k = r_2 - l_2$ and $k_j = r_{p_j} - l_{p_j}$. This map is unique so we define $\bar{v}_2(q) = k$ and $\bar{v}_2(0) = \infty$.

Remark: Let $q \in \mathbb{Q}$ be in reduced form as above. Then we have three different cases:
\begin{enumerate}
\item $n, d \equiv_2 1 \Rightarrow \bar{v}_2(q)= 0$
\item $n \equiv_2 0, d \equiv_2 1 \Rightarrow \bar{v}_2(q)$ is the largest power, $k$, of $2$ such that $2^k|n$.
\item $n \equiv_2 1, d \equiv_2 0 \Rightarrow -\bar{v}_2(q)$ is the largest power, $k$, of $2$ such that $2^k|d$.
\end{enumerate}

\begin{proposition}

\begin{enumerate}
\item $\bar{v}_2(\pm 1) = 0$
\item $\bar{v}_2 (-q) = \bar{v}_2 (q)$
\item $\bar{v}_2(q_1q_2) = \bar{v}_2 (q_1) + \bar{v}_2 (q_2)$
\end{enumerate}
\end{proposition}

Now, we wish to extend $\bar{v}_2$ to the real subsets of $\mathcal{R}_d = \mathbb{Z}[\zeta_{2^{d+1}}, 1/2]$, where $\zeta_{2^{d+1}}$ is the $d+1$ primitive root of unity. To do this, we need the field norm of a cyclotomic field, which is defined as $N_d: \mathbb{Q}(\zeta_{2^{d+1}}) \longrightarrow \mathbb{Q}$ for a positive integer $d$, where $N_d (\alpha) = \prod_{\sigma \in Gal(\mathbb{Q}(\zeta_{2^{d+1}})/\mathbb{Q}} \sigma (\alpha)$. 

For a cyclotomic field, each $\sigma \in Gal(\mathbb{Q}(\zeta_{2^{d+1}})/\mathbb{Q})$ sends $\zeta_{2^{d+1}}$ to one of it's conjugates, so that we can define the $\sigma$'s in the following way: $\sigma_k (\zeta_{2^{d+1}}) = \zeta_{2^{d+1}}^k$. Note that $Gal(\mathbb{Q}(\zeta_{2^{d+1}})/\mathbb{Q} \cong (\mathbb{Z}/n\mathbb{Z})^x$, and that elements in $Gal(\mathbb{Q}(\zeta_{2^{d+1}})/\mathbb{Q}$ are precisely the $\sigma_k$'s where $gcd(k, 2^{d+1}) = 1$. This means that $k$ must be odd, so that:

\begin{equation}
N_d(\alpha) = \prod_{k-odd} \sigma_k (\alpha) = \prod_{k=0}^{2^d - 1} \sigma_{2k+1} (\alpha)
\end{equation}

Remark if $d \leq d'$ then $\mathbb{Q}(e^{i\cdot \pi / 2^d}) \subseteq \mathbb{Q}(e^{i\cdot \pi / 2^{d'}})$.

\begin{proposition}
Consider $N_{d+1}: \mathbb{Q}(e^{i\cdot \pi / 2^{d+1}}) \longrightarrow \mathbb{Q}$. Then when restricting $N_{d+1}$ to $\mathbb{Q}(e^{i\cdot \pi / 2^{d}})$ we have $N_{d+1} = N_d^2$.
\end{proposition}

\begin{proposition}
Let $d$ be a positive integer. Then $N_d$ is multiplicative.
\end{proposition}

Note: $\cup \mathbb{Q}(e^{i\cdot \pi / 2^{j}}) = \mathbb{R}$
\begin{definition}
Define $v_2: \mathbb{R} \longrightarrow \mathbb{Q}$ by 
\begin{equation}
v_2 (x) = \frac{\bar{v}_2 (N_d (x))}{2^d}
\end{equation}
where $d = min\{ y; x \in \mathbb{Q} (e^{i \cdot \pi / 2^y })\}$.
\end{definition}

Remark: In the above definition we could instead choose any $d$ for which $x \in \mathbb{Q} (e^{i \cdot \pi / 2^d})$ since if $d \leq d'$ and $x \in \mathbb{Q} (e^{i \cdot \pi / 2^d})$ then $x \in \mathbb{Q} (e^{i \cdot \pi / 2^{d'}})$.
Then we have that $\frac{\bar{v}_2 (N_d (x))}{2^d} = \frac{\bar{v}_2 (N_{d'} (x))}{2^{d'}}$.

\begin{proposition}
If $x \in \mathbb{Q}$ then $v_2 (x) = \bar{v}_2 (x)$.
\end{proposition}

\begin{proposition}
If $x, x' \in \mathbb{R}$ then $v_2 (x \cdot x') = v_2 (x) + v_2 (x')$.
\end{proposition}




The power of $2$ in the denominator of the Pauli expectation gives us a sense of how non-stabilizer the state is. The definition of the dyadic monotone below gives us an intuition for this sort of measure. 

%Firstly, let $q$ be a non-zero rational number. It can be written as a product of integer powers of prime numbers in a unique way:
%\begin{center}
%$q = \pm 2^k \cdot p_1^{k(1)} \cdots p_m^{k(m)}$, $p_k$ are odd primes, and $k, k(1), \dots , k(m)$ are integers
%\end{center}
%
%Let us define $v_2 (q)$ to be $k$. Note that function $v_2$ is somewhat similar to log$|\cdot|$ in that $v_2(q_1q_2) = v_2(q_1) + v_2(q_2), v_2(\pm 1) = 0$ and $v_2 (q) = v_2 (-q)$. For odd integer $a$ and integer $k$ the value is $v_2 (a/2^k) = - k$. Note also that $v_2$ is always non-negative for integer arguments. It is convenient to extend $v_2$ to all rational numbers, by defining $v_2(0) = + \infty $ (and note that the multiplicative property still holds).


\begin{definition}
Let $\ket{\psi}$ be an $n$-qubit state with entries in $\mathbb{Z}[i, 1/2]$, the dyadic monotone is 
\begin{center}
$\mu_2 \ket{\psi} = \texttt{max} \big\{-v_2(\braket{\psi | P | \psi}): P \in \{I, X, Y, Z \}^{\otimes n} \big\}$.
\end{center}
\end{definition}
The dyadic monotone basically is the maximum power of two in the denominator over the Pauli spectrum. Now note that it is invariant under Clifford unitaries because they map the set of all multi-qubit Pauli matrices to the set of all of all Pauli matrices up to a sign and $v_2$ is insensitive to the sign of its argument. Moreover, Clifford unitaries map states with entries in $\mathbb{Z}[i, 1/2]$ to states with entries in $\mathbb{Z}[i, 1/2]$ because all Clifford unitaries can be written as matrices with entries in $\mathbb{Z}[i, 1/2]$ up to a global phase. The next proposition shows a similarity between the stabilizer nullity $\nu$ and $\mu_2$, namely the nice behaviour under tensor products.

\begin{proposition}
Let $\ket{\phi}$ and $\ket{\psi}$ be states with entries in $\mathbb{Z}[i, 1/2]$, then 
\begin{center}
$\mu_2 (\ket{\phi} \otimes \ket{\psi}) = \mu_2 \ket{\phi} + \mu_2 \ket{\psi}$
\end{center}
\end{proposition}

\begin{proof}
The results follows from the fact that for Pauli matrices $P$ and $Q$ such that the expectations $\braket{\phi | P |\phi}$ and $\braket{\psi | Q | \psi}$ are non-zero it is the case that:
\begin{center}
$v_2 (\bra{\phi} \otimes \bra{\psi} (P \otimes Q) \ket{\phi} \otimes \ket{\psi}) = v_2 (\braket{\phi | P | \phi}) + v_2(\braket{\psi | Q | \psi})$.
\end{center}
\end{proof}

Another important property is that the dyadic monotone is minimal for stabilizer states:
\begin{proposition}
Let $\ket{\psi}$ be a state in $\mathbb{Z}[i, 1/2]$, then $\mu_2 \ket{\psi} \geq 0$ with equality achieved if and only if $\ket{\psi}$ is a stabilizer state.
\end{proposition}
\begin{proof}
Consider a non-zero Pauli expectation $\braket{\psi |P| \psi}$ and write it as $a/2^k$ for some odd integer $a$. Note that $k$ must be non-negative because $|\braket{\psi |P| \psi}| \leq 1$. This shows that $\mu_2 (\ket{\psi}) \geq 0$. For stabilizer states, the only non-zero expectations can be $\pm 1$ and therefore $\mu_2$ is zero. It remains to show that $\mu_2 (\ket{\psi}) = 0$ implies that $\ket{\psi}$ is a stabilizer state. First note that $\mu_2 (\ket{\psi}) = 0$ implies that all non-zero Pauli expectations are odd integers. Together with the condition $|\braket{\psi |P| \psi}| \leq 1$ this implies that the expectations can only be $\pm 1$, in other words either $P$ or $-P$ is in Stab$\ket{\psi}$. Suppose that $\ket{\psi}$ is an $n$-qubit state and let us compute the size of Stab$\ket{\psi}$. Note that the set $\{ I, X, Y, Z \}^{\otimes n}$ is an orthogonal basis of the space of matrices with respect to the inner product $\langle A, B \rangle = \texttt{Tr} AB^{\dag}$. The norm squared of the density matrix $\ket{\psi} \bra{\psi}$ is given be the following expression:
\begin{center}
$1 = \braket{\psi | \psi}^2 = \frac{1}{2^n} \sum\limits_{P \in \{ I, X, Y, Z \}^{\otimes n}} |\texttt{Tr}(\ket{\psi}\bra{\psi}P)|^2$
\end{center}
which implies that the size of Stab$\ket{\psi}$ is $2^n$ and therefore that $\ket{\psi}$ is a stabilizer state.
\end{proof}

We will now need the fact that Pauli measurements with probability half take states with entries in $\mathbb{Z}[i, 1/2]$ to states with entries in $\mathbb{Z}[i, 1/2]$ (allowing the dyadic monotone to be evaluated). To see this, recall that measuring a $\pm 1$ eigenvalue of a Pauli observable $P$ with probability $1/2$ is equivalent to multiplying the state by the matrix $(I \pm P)/\sqrt{2}$ which is equal to $(1 + i)(I \pm P)/2$ up to a global phase. The matrix $(1 + i)(I \pm P)/2$ has entries in the ring $\mathbb{Z}[i, 1/2]$ and therefore the resulting state will also have entries in $\mathbb{Z}[i, 1/2]$.

The next goal is to show that $\mu_2$ in non-increasing under these measurements, but to do this we will require another property of the function $v_2$ as seen in the next proposition:

\begin{proposition}
\label{prop6.4}
For arbitrary rational numbers $a, b$ the following inequality holds
\begin{center}
$v_2 (a_1 + a_2) \geq min(v_2 (a_1), v_2(a_2))$.
\end{center} 
\end{proposition}
\begin{proof}
Let us first prove the inequality for non-zero $a_1, a_2$. Rewrite $a_j = 2^{k_j}p_j/q_j$ for integer $k_j$ and odd integers $p_j$ and $q_j$ such that
\begin{center}
$a_1 + a_2 = 2^{min(k_1, k_2)} \big( 2^{k_1 - min(k_1, k_2)} p_1q_2 + 2^{k_2 - min(k_1, k_2)}p_2q_1 \big) /q_1q_2$.
\end{center}
Since $q_1$ and $q_2$ are odd, $v_2 (a_1 + a_2)$ is equal to
\begin{center}
min$(k_1, k_2) + v_2 \big( 2^{k_1 - min(k_1, k_2)} p_1q_2 + 2^{k_2 - min(k_1, k_2)}p_2q_1 \big)$
\end{center}
by the multiplicative property of $v_2$. Since $2^{k_1 - min(k_1, k_2)} p_1q_2 + 2^{k_2 - min(k_1, k_2)}p_2q_1$ is an integer, its value of $v_2$ is non-negative. Finally, the case when at least one of $a_j$ is zero follows from the fact that min$(x, +\infty) = x$.
\end{proof}

Next we use this to prove what we initially wanted to show.
\begin{proposition}
\label{dyadic is monotone}
Let $\ket{\psi}$ be a state with entries in $\mathbb{Z}[i, 1/2]$, let $P$ be a Pauli observable such that measuring its eigenvalue $+1$ has probability $1/2$ and let $\ket{\psi_+}$ be the normalized result of that measurement. Then $\mu_2\ket{\psi} \geq \mu_2\ket{\psi_+}$. 
\end{proposition}
\begin{proof}
Let us bound the value of $v_2$ for some Pauli operator $Q$ evaluated on the expectation $\braket{\psi_+ | Q |\psi_+}$. The normalized state is $\ket{\psi_+} = \frac{I + P}{\sqrt{2}}\ket{\psi}$. The expectation of $Q$ is therefore equal to:
\begin{center}
$\braket{\psi_+ | Q |\psi_+} = \braket{\psi | (I + P)Q(I + P) | \psi}/2$.
\end{center}
If $P$ and $Q$ anti-commute, the expectation is zero and does not contribute to the calculation of $\mu_2$. When $P$ and $Q$ commute, the expectation is equal to $\braket{\psi | Q | \psi} + \braket{\psi |PQ |\psi}$. Next we use the inequality from Proposition \ref{prop6.4} to see that:
\begin{center}
$v_2(\braket{\psi |Q| \psi} + \braket{\psi |PQ| \psi}) \geq \texttt{min}(\braket{\psi |Q|\psi}, \braket{\psi |pQ|\psi}) \geq -\mu_2\ket{\psi}$.
\end{center}
Now recall that $v_2(\braket{\psi |Q| \psi} + \braket{\psi |PQ| \psi}) = v_2(\braket{\psi_+ |Q| \psi_+})$ which is exactly equal to $\mu_2\ket{\psi_+}$ given our choice of $Q$. So finally by multiplying by $-1$, we have the result: $\mu_2\ket{\psi} \geq \mu_2\ket{\psi_+}$
\end{proof}


%\section{Catalysis}
%
%\begin{theorem}
%Let F be a number field which contains $\mathbb{Q}(i)$ and which is closed under complex conjugation. Any stabilizer circuit applied to a density matrix with all entries in F produces a density matrix with all entries in F, with both density matrices written in the computational basis.
%\end{theorem}
%
%For example, no stabilizer circuit on any number of $\ket{CS}$ or $\ket{CCZ}$ states (which have density matrices with all entries in $\mathbb{Q}(i)$) can be used to produce a $\ket{T}$ state (which has a density matrix with all entries in $\mathbb{Q}(\zeta_8)$). Similarly, no stabilizer circuit on any number of $\ket{T}$ states can be used to produce a $\ket{\sqrt{T}}$ state (with entries in $\mathbb{Q}(\zeta_{16})$).
% 
%\begin{proof}
%Suppose our stabilizer circuit acts upon $N$ qubits initially in the $\ket{0}$ state. Clearly the density matrix $\rho_{initial} = (\ket{0}\bra{0})^{\otimes n}$ has entries over $\mathbb{Q}$. We point out that all Clifford unitaries can be written as matrices with entries over $\mathbb{Q}(i)$, and therefore as matrices with entries over F. Explicitly, the Clifford group is generated by $H$, $CZ$, and $S$ which are defined as:
%\begin{align*}
%H = \frac{1}{1+i} 
%\left[
%\begin{matrix}
%1 & 1 \\
%1 & 1
%\end{matrix}
%\right]
%\end{align*}
%
%\begin{align*}
%S: \ket{0} \mapsto \ket{0} , \ket{1} \mapsto \ket{1}
%\end{align*}
%\begin{align*}
%CZ: \ket{ab} \mapsto (-1)^{a \wedge b}\ket{ab}
%\end{align*}
%Given any gate U in the circuit is a tensor product of a unitary with entries over F and I and $\rho$ has entries over F the product $U\rho U^{\dag}$ is a density matrix with entries over F. Therefore applying the gates in the circuit preserves the required property. Note that measurement with or without post-selection can be described as:
%\begin{align*}
%\rho \mapsto \frac{P\rho P}{Tr\rho P}
%\end{align*}
%\begin{align*}
%\rho \mapsto \sum_{P \in \mathcal{P}} P\rho P
%\end{align*}
%The projectors $P$ above correspond to measurement in the computational basis and therefore can be written as matrices with entries over $\mathbb{Q}(i)$ and therefore over F. The product of matrices over F is a matrix over F. The trace of a matrix over F is also in F by the definition of a field. The quotient of a matrix over F and an element of F is again a matrix over F because any field is closed under the division operation. This completes the proof. 
%\end{proof}
%$
%\Qcircuit @C=1em @R=.7em {
%   & \ctrl{1} & \qw  & \raisebox{-2.2em}{=}  & & \gate{T} & \ctrl{1} & \qw & \ctrl{1} & \qw & \raisebox{-2.2em}{$\implies$}  & & \gate{T} & \qw & \raisebox{-2.2em}{=} & & \ctrl{1} & \ctrl{1} &\qw & \ctrl{1} & \qw \\
%   & \gate{S} & \qw & & & \gate{T} & \targ & \gate{T^{\dag}} & \targ & \qw & & & \gate{T} & \qw & & & \gate{S} & \targ & \gate{T} & \targ & \qw
%}
%$
%\begin{definition}[Conversion Notation]
%The equation $\ket{A} \rightarrow \ket{B}$ indicates that resource state $\ket{A}$ can be converted into resource state $\ket{B}$ with stabilizer operations in the absences of a catalyst. On the other hand, $\ket{A} \xRightarrow{\ket{C}} \ket{B}$, which is equivalent to $\ket{A}\ket{C} \rightarrow \ket{B}\ket{C}$, indicates the conversion can proceed with the use of a catalyst $\ket{C}$ (which may sometimes be omitted above the arrow). When a process is impossible, we strike through the arrow, for example $\ket{A} \nRightarrow \ket{B}$ signifies that $\ket{A}$ cannot be converted to $\ket{B}$ by stabilizer operations even in the presence of an arbitrary catalyst. In cases involving multiple copies of a given state such as $\ket{A}^{\otimes 2} \xRightarrow{\ket{C}} \ket{B}$, we sometimes write $2\ket{A} \xRightarrow{\ket{C}} \ket{B}$ to avoid clutter.
%\end{definition}



\chapter{Applications}
Suppose we can identify a monotone $\mathcal{M}$ such that $\mathcal{M}(\ket{\psi})$ is real for any state $\ket{\psi}$, and is non-increasing under stabilizer operations. We call such a function a \textit{monotone}, and use it to bound conversion precesses since for example, a resource state $\ket{A}$ cannot be used to produce a resource state $\ket{B}$ with stabilizer operations if $\mathcal{M}(\ket{A}) < \mathcal{M}(\ket{B})$, so we have: $\mathcal{M}(\ket{A}) < \mathcal{M}(\ket{B}) \implies \ket{A} \not \longrightarrow \ket{B} $
%\section{Conversions}
%\begin{theorem}
%\label{min number of T gates}
%Let $\ket{U}$ be an n-qubit magic state for a diagonal unitary $U$ from the $3^{\text{rd}}$ level of the Clifford hierarchy, and let $\tau (U)$ be the minimum number of T gates needed to implement $U$ using the gate set $\{CNOT, S, T\}$. The following resource conversion is possible
%
%\begin{align*}
%\ket{U} \xRightarrow{\ket{T}^{\otimes \tau (U) - \nu (\ket{U}}} \ket{T}^{\otimes 2 \nu (\ket{U}) - \tau (U)}
%\end{align*}
%\end{theorem}
%
%\begin{proof}
%Recall the following phase polynomial formalism. For any diagonal unitary in the $3^{\text{rd}}$ level Clifford hierarchy we have $U_f = \sum_x exp(if(x) \pi/4) \ket{x}\bra{x}$, where $f:\mathbb{Z}_2^n \rightarrow \mathbb{Z}_8$ is of cubic form and so can be decomposed as the phase polynomial $f(x) = \sum_{a_k \neq 0} a_k \lambda_k (x)$ (mod 8) where $a_k \in \mathbb{Z}_8$ and each $\lambda_k$ is a $\mathbb{Z}_2$ linear function. That is, each $\lambda_k$ has the form $\lambda_k (x) = (P_{1, k}x_1) \oplus (P_{2, k}, x_2) \dots (P_{n, k}, x_n)$ (mod 2) where $P_{j, k}$ are binary. Thus we can describe the function by a binary matrix $P$ and vector $a$, with columns corresponding to nonzero $a_k$ (so the number of columns is the number of terms in $f$). For a function with a single term $f(x) = a_k \lambda_k(x)$, an easily verified circuit decomposition is 
%$U_{\lambda_k} = \sum_x exp(i\lambda_k(x) \pi/4) \ket{x}\bra{x} = V^{\dag}_{CNOT(\lambda_k)}T_1^{a_k}V_{CNOT(\lambda_k)}$ where $T_1$ is a $T$ gate acting on qubit 1 and $V_{CNOT(\lambda_k)}$ is a cascade of CNOT gates such that 
%
%$V_{CNOT(\lambda_k)} \ket{x} = V_{CNOT(\lambda_k)} \ket{x_1, x_2, \dots x_n} = \ket{\lambda_k(x_1), x_2, \dots x_n}$.
%
%Now note that if $a_k$ is even then $T_a^{a_k} = S_1^{a_k/2}$ is a Clifford and the whole circuit is Clifford. But if $a_k$ is odd then $T_1^{a_k} = T_1S_1^{(a_k - 1)/2}$ and only a single $T$ gate is used. Now, generalizing to a phase polynomial $f$ with many terms we have $U_f = \displaystyle \prod_k U_{\lambda_k}$  and so the $T$-count for the associated circuit is equal to the number of odd valued $a_k$ (so if all values are even then the unitary is Clifford).
%
%
%This allows us to split the unitary $U_f$ into a Clifford and non-Clifford part. For each $a_k$ coefficient, we define $b_k \in \mathbb{Z}_4$ and $c_k \in \mathbb{Z}_2$ such that $a_k = 2b_k +c_k$ (so $c_k = 1$ if and only if $a_k$ is odd). Now for functions $g(x) = \displaystyle \sum_{c_k \neq 0} c_k \lambda_k(x)$ and $h(x) = \displaystyle \sum_{b_k \neq 0} b_k \lambda_k(x)$ we have that $f = g + h$ and $U_f = U_{g+2h} = U_gU_{2h}$ where $U_{2h}$ is a Clifford Unitary. The non-Clifford part is $U_g$ and all the terms have odd valued co-coefficients, so the number of terms in $g$ gives an upper bound on $\tau(U_g)$ as discussed earlier. It follows that if the function $g$ has $m$ (odd-valued) terms then the state can be prepared using $m$ many $T$ gates/states. Note that for any given unitary $U_g$ there is an equivalence class of different functions $g$ that all result in the same unitary but with different numbers of terms. From now on we will assume that $g$ is the optimal representative with the fewest number of terms, denoted by $\tau(U_g)$. Furthermore, there is a binary matrix $P$ description of $g$ with a number of columns also equal to $\tau (U_g)$. A trivial but relevant example is $U = T^{\otimes n}$ for which $P = \mathds{1}_n$ and $\tau (T^{\otimes n}) = n$.
%
%The next important step is that given a unitary $U_g$ we may also be able to remove terms from g by applying inverse $T$ gates. More generally, given two such unitaries $U_g$ and $U_{g'}$ with phase polynomials $g$ and $g'$, we have that $U_{g'} = U_g U_{\Delta}$ where $\Delta = g - g'$. Therefore, 
%\begin{equation}
%\ket{U_{g'}} = U_{\Delta} \ket{U_g}
%\end{equation}
%and 
%\begin{equation}
%\ket{T}^{\otimes \tau (U_{\Delta})} \ket{U_{g'}} \rightarrow \ket{U_g}
%\end{equation}
%The number of $T$ states needed is $\tau (U_{\Delta})$, which just the number of terms where $g$ and $g'$ differ.
%
%Using arguments from \cite{?}, given any $P$ we can always bring it into row-reduced echelon form using a CNOT circuit. Then
%\begin{equation}
%P = \begin{pmatrix}
%\mathds{1}_r & A \\
%0 & 0
%\end{pmatrix}
%\end{equation}
%where $\mathds{1}_r$ is an identity matrix of size $r := \text{rank}(P)$. If $P$ is full rank the additional 0 padding is not present. Note that if $P$ has any 0 rows then the unitary acts trivially on the corresponding qubits leaving them in the $\ket{+}$ state, meaning that $\ket{U} = U\ket{+} = \ket{\psi}\ket{+}^{\otimes (n-r)}$ for some state $\ket{\psi}$. Also, for an $n$ qubit stabilizer state $\ket{\phi}$, 
%\begin{equation}
%\nu (\ket{\phi}) = 0 \Rightarrow \text{log}_2 |Stab\ket{\phi}| = n
%\end{equation} 
%Next, observe that 
%
%
%\begin{align*}
%\text{log}_2 |Stab\ket{U}| = & \text{log}_2|Stab(\ket{\psi}\ket{+}^{\otimes (n-r)})| \\ 
%= & \text{log}_2 (|Stab \ket{\psi}| \cdot |Stab\ket{+}^{\otimes (n-r)}|) \\
%= & \text{log}_2 |Stab \ket{\psi}| + \text{log}_2 |Stab \ket{+}^{\otimes (n-r)}| \\
%= & \text{log}_2 |Stab \ket{\psi}| + (n-r) \\
%= & \alpha + n - r
%\end{align*}
%for some positive integer $\alpha = \text{log}_2 |Stab \ket{\psi}|$. Hence log$_2|Stab\ket{U}| \geq n - r$, so rearranging we have that $n - \text{log}_2 |Stab \ket{U}| = \nu (\ket{U}) \leq r$.
%
%Using our earlier argument, we can always remove from $P$ the columns corresponding to the matrix $A$ using a number of $T$ states equal to the number of columns in $A$. Since $A$ has $\tau (U_g) - r$ columns, this requires the same quantity of $T$ states. The resulting $U_{g'}$ has $P' = \mathds{1}_r$ (with possibly some 0 row padding) which corresponds to $r$ copies of $T$ states. Therefore, we can perform 
%\begin{equation}
%\ket{U_g}\ket{T}^{\otimes (\tau (U_g) - r)} \rightarrow \ket{T}^{\otimes r}
%\end{equation}
%If $r = \nu (U_g)$ then we have the result of the theorem. If $r \textgreater \nu (U_g)$ then the result is even stronger than the theorem, and so the theorem holds in either case.
%\end{proof}
%
%
%
%\textbf{Claim 1:} Let $U$ be a diagonal unitary from the third level of the Clifford hierarchy with phase polynomial matrix P. If all rows of P have even Hamming weight then $U\ket{+}^{\otimes n} \not \rightarrow \ket{T}$.
%
%To see this, note that every diagonal unitary from the third level of the Clifford hierarchy is (up to Cliffords) a product of $T$, $CS$, $CCZ$ gates \cite{}. In the special case that $U$ has phase polynomial matrix with even rows, then the unitary is a product of $CS$ and $CCZ$ gates. Such a unitary has elements in the ring $\mathbb{Q}(i)$ and so $U\ket{+}^{\otimes n} \not \rightarrow \ket{T}$ follows. Though this transform is impossible without a catalyst, Theorem \ref{min number of T gates} allows us to easily construct concrete examples.
%
%For any $n \geq 2$, we define $W_n$ as the unitary with phase polynomial matrix
%\begin{align*}
%P_n = (\mathds{1}_n, 1) =
%\begin{pmatrix}
%1 & 0 & \dots & 0 & 1 \\
%0 & 1 & \dots & 0 & 1 \\
%\vdots & \vdots & \ddots & 0 & 1 \\
%0 & 0 & \dots & 1 & 1
%\end{pmatrix}
%\end{align*}
%which is the identity matrix padded with an all-one column. More explicitly, we have 
%\begin{equation}
%W_n = \sum_x exp(i\pi g(x)/4)\ket{x}\bra{x}
%\end{equation}
%with $g(x) = (\oplus_{i=1}^n x_i) + \sum_{i=1}^n x_i$, where $\oplus$ is addition modulo 2. Now we can introduce and prove the following lemma.
%
%\begin{lemma}
%$\tau (W_n) = n + 1$
%\end{lemma}
%\begin{proof}
%Since $P$ has a width of $n+1$ columns, we have $\tau (W_n) \leq n+1$. The only full rank phase polynomial matrices that give a unitary that is Clifford equivalent to $T^{\otimes n}$ are square. Since this is not the case we conclude $\tau (W_n) = n + 1$.
%\end{proof}



\section{The $C^nZ$ gate}
\subsection{Lower Bounds for the $C^nZ$ gate}
The $C^n Z$ gate is a gate worth considering in detail since it is used in many important algorithms, like the Grover's search algorithm. We start by computing the Pauli spectrum of $\ket{\C^nZ}$. Lower bounds on the resources required for this gate follow after this proposition.

\begin{proposition}
\label{prop4.2}
For all $n \geq 3$, the Pauli spectrum of the state $\ket{C^n Z}$ is: $\{ 1 \hspace{2mm} (1)$, $0 \hspace{2mm} (-1 +2^{n} + 2^{2n+1})$, $1-2^{1-n} \hspace{2mm} (2^n)$, $2^{1-n}  \hspace{2mm} (1-3 \cdot 2^{n} + 2^{2n+1}) \}$.
\end{proposition}

\begin{proof}
We have 
\begin{center}
$\ket{C^n Z} = \C^n Z \ket{+}^{\otimes n} = \frac{1}{\sqrt{2^n}} \sum\limits_{b \in \{0, 1\}^n} (-1)^{b_1 \cdot b_2 \cdots b_n} \ket{b}$
\end{center}
We want to compute the Pauli expectation $\braket{\Lambda_n (Z)|X^xZ^z|\Lambda_n (Z)}$ where $x$ and $z$ are bit strings. Note that we need not consider the Pauli matrix $Y$ since $Y = iXZ$, and the phase will vanish once we take the absolute value as in the definition of the Pauli spectrum. For one-bit bit strings $i$ and $j$, observe that $Z^i \ket{j} = \ket{j}$ if $i=j=0$ or $i \neq j$, and $Z^i \ket{j} = -\ket{j}$ if $i=j=1$. So in general $Z^z \ket{b} = (-1)^{z \cdot b}\ket{b}$. Furthermore, $X^i \ket{j} = \ket{i \oplus j}$, where $\oplus$ is addition modulo 2, so for arbitrary bit strings it follows that $X^x\ket{b} = \ket{b \oplus x}$, where $b \oplus x$ is extended component-wise. We can now directly calculate as follows:
\begin{equation*}
\begin{split}
2^n\braket{\Lambda_n (Z)|X^xZ^z|\Lambda_n (Z)} & = \sum\limits_{b, b' \in \{0, 1\}^n} (-1)^{b_1 \cdot b_2 \cdots b_n} (-1)^{b'_1 \cdot b'_2 \cdots b'_n} \braket{b'|X^xZ^z|b} \\
& = \sum\limits_{b, b' \in \{0, 1\}^n} (-1)^{b_1 \cdot b_2 \cdots b_n} (-1)^{b'_1 \cdot b'_2 \cdots b'_n} (-1)^{z \cdot b} \braket{b'|X^x|b} \\
& = \sum\limits_{b, b' \in \{0, 1\}^n} (-1)^{b_1 \cdot b_2 \cdots b_n} (-1)^{b'_1 \cdot b'_2 \cdots b'_n} (-1)^{z \cdot b} \braket{b'|b \oplus x} \\
& = \sum\limits_{b \in \{0, 1\}^n} (-1)^{b_1 \cdot b_2 \cdots b_n} (-1)^{(b_1 \oplus x_1) \cdot (b_2 \oplus x_2) \cdots (b_n \oplus x_n)} (-1)^{z \cdot b}.
\end{split}
\end{equation*}
Note that we have the last line since $\braket{b'|b \oplus x} = 0$ when $b' \neq b \oplus x$, and so we can substitute $b'_i = b_i \oplus x_i$. 

When $x$ is the all zero vector $\textbf{0}$, we get 
\begin{center}
$2^n\braket{C^n Z|X^xZ^z|C^n Z} = \sum\limits_{b \in \{0, 1\}^n} ((-1)^{b_1 \cdot b_2 \cdots b_n})^2  (-1)^{z \cdot b} = \sum\limits_{b \in \{0, 1\}^n} (-1)^{z \cdot b}$.
\end{center} This is $2^n$ for $z = \textbf{0}$
 For any other $z$ half of the terms in the summation will be $-1$ and the other half $1$, giving $0$. 

Now consider the case when $x \neq \textbf{0}$. Denoting $\textbf{1}$ as the all one vector, if $b = \textbf{1}$ then $b_1 \cdot b_2 \cdots b_n = 1 \mbox{ and } (b_1 + x_1) \cdot (b_2 + x_2) \cdots (b_n + x_n) = 0$. Similarly, if $b = \textbf{1} + x$, then $b_1 \cdot b_2 \cdots b_n = 0 \mbox{ and } (b_1 + x_1) \cdot (b_2 + x_2) \cdots (b_n + x_n) = 1$, and in either case we get: 
\begin{center}
$\sum\limits_{b \in \{0, 1\}^n} (-1)^{b_1 \cdot b_2 \cdots b_n} (-1)^{(b_1 + x_1) \cdot (b_2 + x_2) \cdots (b_n + x_n)} (-1)^{z \cdot b} = \sum\limits_{b \in \{0, 1\}^n} (-1) (-1)^{z \cdot b}$.
\end{center}
For any other $b$ we have $b_1 \cdot b_2 \cdots b_n = 0 \mbox{ and } (b_1 + x_1) \cdot (b_2 + x_2) \cdots (b_n + x_n) = 0$ giving us 
\begin{center}
$\sum\limits_{b \in \{0, 1\}^n} (-1)^{b_1 \cdot b_2 \cdots b_n} (-1)^{(b_1 + x_1) \cdot (b_2 + x_2) \cdots (b_n + x_n)} (-1)^{z \cdot b} = \sum\limits_{b \in \{0, 1\}^n} (-1)^{z \cdot b}$.
\end{center}
Thus the terms in the sum over $b$ differ from $\sum_{b \in \{0, 1\}^n} (-1)^{z \cdot b}$ only for $b = \textbf{1}$ and $b = \textbf{1} + x$. Therefore, 
\begin{equation*}
\begin{split}
2^n\braket{\Lambda_n (Z)|X^xZ^z|\Lambda_n (Z)} & = \sum\limits_{b \in \{0, 1\}^n} (-1)^{b_1 \cdot b_2 \cdots b_n} (-1)^{(b_1 + x_1) \cdot (b_2 + x_2) \cdots (b_n + x_n)} (-1)^{z \cdot b} \\
& = - (-1)^{z \cdot 1^n} - (-1)^{z \cdot (1^n + x)} \hspace{2mm} + \sum\limits_{b \in \{0, 1\}^n \texttt{\textbackslash} \{1^n, 1^n + x \} } (-1)^{z \cdot b}  \\
& = -2(-1)^{z \cdot 1^n} - 2(-1)^{z \cdot (1^n + x)} + \sum\limits_{b \in \{0, 1\}^n} (-1)^{z \cdot b}.
\end{split}
\end{equation*}
When $z = 0^n$, this is simply $2^n - 4$. When $z \neq 0^n$, it is $-2(-1)^{z \cdot 1^n} - 2(-1)^{z \cdot (1^n + x)}$, and we have two more final cases. The first is when $x \cdot z$ is odd, where we get:
\begin{equation*}
\begin{split}
-2(-1)^{z \cdot 1^n} - 2(-1)^{z \cdot (1^n + x)} & = -2(-1)^{z \cdot 1^n} - 2(-1)^{z \cdot 1^n} (-1)^{z \cdot x} \\
& = -2(-1)^{z \cdot 1^n} - 2(-1)^{z \cdot 1^n} (-1)^1 \\
& = -2(-1)^{z \cdot 1^n} + 2(-1)^{z \cdot 1^n} \\
& = 0
\end{split}
\end{equation*}
and if $x \cdot z$ is even, then we have $-2(-1)^{z \cdot 1^n} - 2(-1)^{z \cdot 1^n} = -4(-1)^{z \cdot 1^n}$ instead, which simply becomes $4$ since the pauli spectrum requires we take the absolute value. We can now summarize:
\begin{equation*}
|\braket{\Lambda_n (Z)|X^xZ^z|\Lambda_n (Z)}| = \begin{cases}
1 & \texttt{if } z = 0^n \texttt{ and } x = 0^n, \\
1 - 2^{2-n} & \texttt{if } z = 0^n \texttt{ and } x \neq 0^n, \\
0 & \texttt{if } z \neq 0^n \texttt{ and either } x = 0^n \texttt{ or } x \neq 0^n \texttt{ and } \\
& x \cdot z \texttt{ is odd} \\
2^{2-n} & \texttt{if } z \neq 0^n \texttt{ and } x \neq 0^n
 \texttt{ and } x \cdot z \texttt{ is even}
\end{cases}
\end{equation*}
We can count the number of each subset of binary vectors $x$ and $z$ to find multiplicities. When, $x=0^n$ and $z = 0^n$, there is clearly one one possible choice for each $x$ and $z$, thus we have a multiplicity of $1$. When $z = 0^n$ and $x \neq 0^n$, we now have only $1$ possibility for $z$ and $2^n - 1$ possibilities for $x$, giving a multiplicity of $2^n - 1$ for this case. Similarly, when $x = 0^n$ and $z \neq 0^n$ we have $2^n - 1$ possibilities again. Also, observe that $x \cdot z$ takes an odd value for $(2^{2n}-2^{n})/2$ different possible combinations of $x$ and $z$ pairings. Adding these and simplifying gives the multiplicity of the third case: $2^{2n-1} - 2^{n+1} - 1$. Finally, there are overall $2^{2n}$ different combinations of $x$ and $z$ pairings, so we can simply subtract all the multiplicities above from this to get the multiplicity of the fourth case, which comes out to be $2^{2n-1} - 3 \cdot 2^{n-1} + 1$ as above.
\end{proof}

\begin{corollary}
\label{nu values}
We have the following stabilizer nullity values: $\nu (\ket{T}) = 1$, $\nu (\ket{CS}) = 2$, $\nu (\ket{CCZ}) = 3$, $\nu (\ket{C^n Z}) = n+1$
\end{corollary}
\begin{proof}

The first three follow by direct calculation. For $\nu (\ket{C^n Z}) = n+1$, recall that if $\ket{\psi}$ is an $m$-qubit state then $\nu
(\ket{\psi}) = m - \log_2|\Stab(\ket{\psi})|$. By
%proposition \ref{prop4.2} we have $|$Stab$(\ket{\psi})| = 1$ so that
%$\nu (\ket{\Lambda_n (Z)}) = n+1 - \textit{log}_2 (1) = n+1-0 = n+1$.
Proposition \ref{prop4.2} we have $|\Stab(\ket{C^n Z})| = 1$ since the size of the stabilizer of a state is the multiplicity of $1$ in its Pauli spectrum. Hence $\nu (\ket{C^n Z}) = n+1 - \log_2 (1) = n+1-0 = n+1$.
\end{proof}

\begin{proposition}
\label{nullity(C^nZ) LB}
For $n \geq 2$, the $C^n Z$ gate cannot be implemented with Clifford gates and measurements using fewer than $n+1$ $T$ gates, or $(n+1)/2$ $CS$ gates, or $(n+1)/3$ $CCZ$ gates.
\end{proposition}

\begin{proof}
First note that proving that a bound holds for the state $\ket{\Lambda_n (Z)}$ implies that it holds for the gate $\Lambda_n (Z)$. Indeed, if we can perform a task with $k$ $T$ gates then we can also perform it with $k$ $\ket{T}$ states. Hence, any lower bound on the number of required states is a lower bound on the number of required gates. Now, $\nu (\ket{T}) = 1$ and $\nu (\ket{\Lambda_n (Z)}) = n+1$ by Corollary \ref{nu values}. Since $\nu$ is a monotone it is non-increasing under Clifford operations and measurements. This implies that at least $n+1$ $\ket{T}$ states are required to implement the $\ket{\Lambda_n (Z)}$ state and hence the $\Lambda_n (Z)$ gate. Similarly, since $\nu (\ket{CS}) = 2$ and $\nu (\ket{CCZ}) = 3$, we get that to implement the $C^n Z$ we need at least $(n+1)/2$ $\ket{CS}$ states or $(n+1)/3$ $\ket{CCZ}$ states.
\end{proof}

The above lower bound can be improved when measurements are restricted.

\begin{corollary}
\label{dyadic values}
We have the following dyadic monotone values: $\mu_2 (\ket{T}) = 1/2$, $\mu_2 (\ket{CS}) = 1$, $\mu_2 (\ket{CCZ}) = 1$, and $\mu_2 (\ket{C^n(z)}) = n-1$.
\end{corollary}
\begin{proof}
The first three follow from direct calculation. For the $\ket{C^nZ}$ state, Proposition \ref{prop4.2} gives us all of its possible values for the Pauli spectrum. Thus to get the dyadic monotone, we simply take the $p$-adic valuation of these values, negate it and take the maximum, which gives $n-1$ as desired.
\end{proof}

\begin{lemma}
\label{CCZ lowerbound}
For $n \geq 2$, the $C^n Z$ gate cannot be implemented with Clifford gates and measurements with probability one half using fewer than $2n-2$ $T$ gates, or $n-1$ $CS$ gates, or $n-1$ $CCZ$ gates.
\end{lemma}
\begin{proof}
We reason as in the proof of Proposition \ref{nullity(C^nZ) LB} but using the dyadic monotone which is also non-increasing under Clifford operations, but with measurements of probability one half, as seen in \ref{dyadic is monotone}. This time the values in Corollary \ref{dyadic values} are used.
\end{proof}

\subsection{Upper Bounds for the $C^nZ$ gate}
\label{Upper Bounds for C^nZ}

The next proposition provides reasoning as to why we do not need more than $n-1$ $CCZ$ gates by showing a specific circuit construction.

\begin{proposition}
\label{CCZ upperbound}
Let $n \in \mathbb{Z}^{\geq 2}$. The $\Lambda_n (Z)$ gate can be implemented using $n-1$ $CCZ$ gates along with Clifford gates and measurements with probability $1/2$.
\end{proposition}
\begin{proof}
In the proof, we use the $CCZ$ and the $CCX$ gates interchangeably because they are equivalent under a simple conjugation by a Hadamard gate)
We reason by induction on $n$. Consider the base case of $n=2$. Then $C^nZ$ is just the $CCZ$ gate, so the result holds in this case.

Now for $n \geq 3$ assume we have a circuit $D$ for $C^{n-1}Z$ which can be performed using $n-2$ $CCZ$ gates. Then consider the following circuit:

\begin{center}
$
\Qcircuit @C=0.7em @R=1.1em {
   & \lstick{\ket{c_1}} & \qw & \ctrl{1} & \qw & \qw & \qw & \qw & \qw & \qw & \multigate{1}{CZ} & \qw \\
   & \lstick{\ket{c_2}} & \qw & \ctrl{1} & \qw & \qw & \qw & \qw & \qw & \qw & \ghost{CZ} & \qw \\
   & \lstick{\ket{0}} & \qw & \targ & \qw & \multigate{4}{D} & \qw & \qw & \gate{H} & \meter & \cctrlo{-1} \\
   & \lstick{\ket{c_3}} & \qw & \qw & \qw & \ghost{D} & \qw & \qw & \qw & \qw & \qw & \qw \\
   & \lstick{\vdots}  \\
   & \lstick{\vdots}  \\
   & \lstick{\ket{t}} & \qw & \qw & \qw & \ghost{D} & \qw & \qw & \qw & \qw & \qw & \qw 
   }
$
\end{center}

Starting with the input state and tracking the state of this system step by step, we get the following: 
\begin{itemize}
\item[] $\ket{c_1}\ket{c_2}\ket{0}\ket{c_3} \cdots \ket{c_{n-1}}\ket{t}$
\item[$\mapsto$] $\ket{c_1}\ket{c_2}\ket{(c_1 \cdot c_2)}\ket{c_3} \cdots \ket{c_{n-1}}\ket{t}$
\item[$\mapsto$] $\ket{c_1}\ket{c_2}\ket{(c_1 \cdot c_2)}\ket{c_3} \cdots \ket{c_{n-1}} \ket{t \oplus ((c_1 \cdot c_2) \cdot c_3 \cdots c_{n-1})}$
\item[$=$] $\ket{c_1}\ket{c_2}\ket{(c_1 \cdot c_2)}\ket{c_3} \cdots \ket{c_{n-1}} \ket{t \oplus (c_1 \cdots c_{n-1})}$
\item[$\mapsto$] $\begin{cases} \ket{c_1}\ket{c_2}\ket{+}\ket{c_3} \cdots \ket{c_{n-1}} \ket{t \oplus (c_1 \cdots c_{n-1})} & \mbox{if } c_1 \cdot c_2 = 0, \\
\ket{c_1}\ket{c_2}\ket{-}\ket{c_3} \cdots \ket{c_{n-1}} \ket{t \oplus (c_1 \cdots c_{n-1})} & \mbox{if } c_1 \cdot c_2 = 1 \end{cases}$
\item[$\mapsto$] $\begin{cases} \ket{c_1}\ket{c_2}\ket{c_3} \cdots \ket{c_{n-1}} \ket{t \oplus (c_1 \cdots c_{n-1})} & \mbox{if measurement is } \ket{0} , \\
\pm \ket{c_1}\ket{c_2}\ket{c_3} \cdots \ket{c_{n-1}} \ket{t \oplus (c_1 \cdots c_{n-1})} & \mbox{if measurement is } \ket{1} \end{cases}$
\item[$\mapsto$] $\ket{c_1}\ket{c_2}\ket{c_3} \cdots \ket{c_{n-1}}\ket{t \oplus (c_1 \cdots c_{n-1})}$
\end{itemize}
Hence, this circuit acts as the $C^nZ$ gate. Note that the uncomputation of the added toffoli gate follows from \ref{one bit adder b}.
\end{proof}


We can also give upper-bounds similar to the ones in Proposition \ref{CCZ upperbound} for the $CS$ and $T$ gates.

\begin{proposition}
\label{CS upperbound}
The $C^{n}Z$ gate can be implemented using exactly $2n-2$ $CS$ gates along with Clifford gates and measurements with probability $1/2$.
\end{proposition}
\begin{proof}
Simply put, we can implement the $CCZ$ gate with a circuit using only two $CS$ gates and then use this circuit to replace all the CCZ gates used in the circuit in proposition \ref{CCZ upperbound}. The circuit for the $CCZ$ gate using two $CS$ gates is given below.

\begin{center}
$
\Qcircuit @C=0.7em @R=1.1em {
   & \lstick{\ket{c_1}} & \qw & \qw & \qw & \targ & \gate{S^{\dag}} & \targ & \ctrl{1} & \qw & \qw & \qw & \qw & \qw & \multigate{1}{CZ} & \qw \\
   & \lstick{\ket{c_2}} & \qw & \qw & \gate{S} & \qw & \ctrl{-1} & \qw & \gate{Z} & \qw & \qw & \qw & \qw & \qw & \ghost{CZ} & \qw \\
   & \lstick{\ket{0}} & \qw & \gate{H} & \ctrl{-1} & \ctrl{-2} & \qw & \ctrl{-2} & \qw & \gate{H} & \ctrl{1} & \gate{S} & \gate{H} & \meter & \cctrlo{-1} \\
   & \lstick{\ket{c_3}} & \qw & \qw & \qw & \qw & \qw & \qw & \qw & \qw & \gate{Z} & \qw & \qw & \qw & \qw & \qw \\
   }
$
\end{center}

Thus, since we used $n-1$ $CCZ$ gates in proposition \ref{CCZ upperbound}, we can implement the $C^{n-1}Z$ gate with $2\cdot (n-1)$ $CS$ gates.

\end{proof}

\begin{proposition}
\label{T upperbound}
The $C^{n}Z$ gate can be implemented using exactly $4n-4$ $T$ gates along with Clifford gates and measurements with probability $1/2$.
\end{proposition}
\begin{proof}
Simply put, we can write the $CCZ$ gate with an equivalent circuit using four $T$ gates and then use this circuit to replace all the CCZ gates used in the circuit in proposition \ref{CCZ upperbound}. The circuit for the $CCZ$ gate using four $T$ gates is given below.

\begin{center}
$
\Qcircuit @C=0.7em @R=1.1em {
   & \lstick{\ket{c_1}} & \qw & \qw & \qw & \ctrl{2} & \qw & \targ & \gate{T^{\dag}} & \targ & \qw & \qw & \qw & \qw & \qw & \qw & \multigate{1}{CZ} & \qw \\
   & \lstick{\ket{c_2}} & \qw & \qw & \qw & \qw & \ctrl{1} & \targ & \gate{T^{\dag}} & \targ & \qw & \qw & \qw & \qw & \qw & \qw & \ghost{CZ} & \qw \\
   & \lstick{\ket{0}} & \qw & \gate{H} & \gate{T} & \targ & \targ & \ctrl{-2} & \gate{T} & \ctrl{-2} & \gate{H} & \gate{S} & \ctrl{1} & \qw & \gate{H} & \meter & \cctrlo{-1} \\
   & \lstick{\ket{c_3}} & \qw & \qw & \qw & \qw & \qw & \qw & \qw & \qw & \qw & \qw & \gate{Z} & \qw & \qw & \qw & \qw & \qw \\
   }
$
\end{center}

Again, since we used $n-1$ $CCZ$ gates in proposition \ref{CCZ upperbound}, we can implement the $C^{n}Z$ gate with $4\cdot (n-1)$ $T$ gates.

\end{proof}

\section{Quantum Fourier Transform}

%To establish lower bounds for adder circuits we will use the fact that adder can create a complex conjugate copy of a Fourier state. Our strategy is to generalize $\mu_2$ to be defined on a wider set of states including Fourier states. This is achieved by extending the domain of $v_2$ to a wider set of values. Now we will list and discuss properties of $v_2$ needed for the lower-bound proof and prove the lower bound for the adder using them. 
%
%
%We will now need to define rings we can use to write down the coordinates of Fourier states, and not just $\ket{C^nZ}$ states. To do this we extend the domain of $\mu_2$ to the union of the following family of sets:
%\begin{center}
%$\mathcal{R}_d = \mathbb{Z} \big[ exp(i\pi /2^d), 1/2 \big] = \Big\{ \frac{1}{2^k} \sum\limits_{j=0}^{2^d - 1} a_j \texttt{exp}(i\pi j/2^d): \texttt{where } a_j, k \texttt{ are integers} \Big\}$
%\end{center}
%Note that $\mathcal{R}_d$ is closed under addition, negation, and multiplication making it a ring, and furthermore that it is closed under taking complex conjugates. Also note that $\mathcal{R}_1 = \mathbb{Z}[i, 1/2]$ and $\mathcal{R}_d \subset \mathcal{R}_{d+1}$ for all positive $d$.
%
%We extend the domain of $v_2$ so it is defined on values of Pauli expectations of Fourier states, and so $v_2$ must be defined at least on the real subsets of $\mathcal{R}_d$. The proof of the lower bound for multiply-controlled-$Z$ gate relied on additivity for a tensor product of states and monotonicity under measurements with probability $1/2$ of dyadic monotone $\mu_2$. In turn, our proofs of the mentioned properties of $\mu_2$ relied on the properties: $v_2(a \cdot b) = v_2(a) + v_2(b)$ and $v_2(a+b) \geq \texttt{min}(v_2(a), v_2(b))$. All these properties also hold for the extension of $v_2$, and now note some explicit values of $v_2$ that we will need to compute $\mu_2$ for Fourier states:
%\begin{equation}
%\mbox{For all odd integers } k \mbox{, integers } d \geq 2: v_2(cos(\pi k/2^d)) = v_2(sin(\pi k /2^d)) = \frac{1}{2^{d-1}} - 1
%\end{equation}
%
%For example, using above we see that $\mu_2\ket{T} = 1/2$ because $v_2(1/\sqrt{2}) = -1/2$. We can immediately conclude that the $C^nZ$ gate requires at least $2(n-2)\ket{T}$ states. Next we proceed to calculate $\mu_2$ for Fourier states.


Now we introduce the Quantum Fourier Transform so that later we can establish lower bounds for the adder circuit. We will use the fact that the adder can create a complex conjugate copy of a Fourier state, but first note some explicit values of $v_2$ that we will need to compute $\mu_2$ for Fourier states:
\begin{equation}
\mbox{For all odd integers } k \mbox{, integers } d \geq 2: v_2(cos(\pi k/2^d)) = v_2(sin(\pi k /2^d)) = \frac{1}{2^{d-1}} - 1
\end{equation}

\begin{proposition}
Consider Fourier state
\begin{center}
$\ket{QFT_n^a} = \sum\limits_{y=o}^{2^n-1} \texttt{exp}\Big[ \frac{i2\pi ay}{2^n} \Big] \ket{y} = \otimes_{k=1}^n \big( \ket{0} + e^{i2\pi a/2^k}\ket{1}\big)$,
\end{center}
For all odd $a$, $\mu_2\ket{QFT_n^a} = n-3 +(1/2)^{n-2}$.
\end{proposition}
\begin{proof}
Recall that Pauli expectations of $(\ket{0} + e^{i2\pi a/2^k}\ket{1})/\sqrt{2}$ are
\begin{center}
$\{ 0, cos(2\pi a/2^k), sin(2\pi a/2^k), 1 \}$.
\end{center}
So for $k \geq 2$ we have:
\begin{center}
$\mu_2 \big( (\ket{0} + e^{i2\pi a/2^k} \ket{1}/\sqrt{2} \big) = v_2 \big( sin(\pi a/2^{k-1}) \big) = 1 - 1/2^{k-2}$
\end{center}
Using the multiplicative property of $\mu_2$ we get:
\begin{center}
$\mu_2 (\ket{QFT_n^a}) = \sum\limits_{k=2}^n \big( 1-1/2^{k-2}) = n-3 + 1/2^{n-2}$
\end{center}
\end{proof}


\section{Modular adder}
\subsection{Lower Bounds for the modular adder}
The adder circuit is a fundamental quantum operation because it implements basic arithmetic, being addition, on pairs of registers in superposition. We can lower bound the resources required to implement it as follows:

\begin{proposition}
An adder circuit on two n-qubit registers acts on basis states as 
\begin{center}
$A(\ket{i} \ket{j}) = \ket{i} \ket{i+j}$
\end{center}
with $i + j$ evaluated modulo $2^n$. It is not possible to implement the adder circuit with Clifford gates and measurements using fewer than $(n-2) \ket{T}$ states, $(n-2)/2 \ket{CS}$ states or $(n-2)/3 \ket{CCZ}$ states.
\end{proposition}

\begin{proof}
We proceed in two steps. First, we show that the adder circuit $A$ acting on the $n$-qubit quantum Fourier state $\ket{QFT_n^b}$ and the stabilizer state $\ket{+}^{\otimes n}$ has the action $A(\ket{+}^{\otimes n} \ket{QFT_n^b}) = \ket{QFT_n^{-b}}\ket{QFT_n^b}$. This tells us that if $A$ is implemented by a set of Clifford gates and Pauli measurements along with some input resource state $\ket{\psi}$, it must be that $\nu (\ket{\psi} \ket{QFT_n^b}) \geq \nu(\ket{QFT_n^{-b}} \ket{QFT_n^b})$ which implies that $\nu (\ket{\psi}) \geq \nu(\ket{QFT_n^{-b}})$ since the stabilizer nullity is additive. Second we show that $\nu(\ket{QFT_n^{-1}}) = n-2$, which then directly implies our bounds since if the bounds are not satisfied, $\nu(\ket{\psi}) \geq \nu(\ket{QFT_n^{-1}})$ would not be satisfied.

First recall the family of quantum Fourier states for each integer $a = 0, 1, \dots, 2^n - 1$:
\begin{equation}
\label{QFTdef}
\ket{QFT_n^a} = \frac{1}{\sqrt{2^n}} \sum\limits_{y=0}^{2^n-1} exp \Big[ \frac{i2 \pi a y}{2^n} \Big] \ket{y} = \otimes_{k=1}^n \frac{\ket{0} + e^{i2\pi a/2^k} \ket{1}}{\sqrt{2^n}}
\end{equation}
where $\ket{y}$ is an $n$-qubit basis state (with $y$ expressed in binary), and note that $\ket{QFT_n^0} = \ket{+}^{\otimes n}$, and $\ket{QFT_n^a} = \ket{a + 2}$. Consider applying the adder to a pair of such states:
\begin{equation}
\begin{split}
A \big(\ket{QFT_n^a} \ket{QFT_n^b} \big) & = \frac{1}{2^n} \sum\limits_{y=0}^{2^n-1} \sum\limits_{z=0}^{2^n-1} exp \Big[ \frac{i2 \pi (ay + bz)}{2^n} \Big] \ket{y}\ket{z+y} \\
& = \frac{1}{2^n} \sum\limits_{y=0}^{2^n-1} \sum\limits_{x=0}^{2^n-1} exp \Big[ \frac{i2 \pi (ay + b(x-y))}{2^n} \Big] \ket{y}\ket{x} \\
& = \ket{QFT_n^{a-b}}\ket{QFT_n^b}
\end{split}
\end{equation}

Setting $a=0$ gives $A(\ket{+}^{\otimes n} \ket{QFT_n^b}) = \ket{QFT_n^{-b}}\ket{QFT_n^b}$, as desired. Finally, to calculate $\nu \ket{QFT_n^{-b}}$, we use the sum of the stabilizer nullity for each state in the tensor product. From the tensor product variant of the definition of the $\ket{QFT_n^b}$ in equation \ref{QFTdef} the first two terms will have $\nu = 0$, since they will be computational basis states. The rest of the terms all have a relative phase ($\neq i$), and for a single qubit it is easy to verify that this results in a trivial stabilizer, thus giving $\nu = 1$. Since the stabilizer nullity is additive, we have that $\nu(\ket{QFT_n^{-1}}) = n-2$. Then considering the following possible implications: $r\ket{T} \Longrightarrow \ket{QFT_n^{-1}}$, $r\ket{CS} \Longrightarrow \ket{QFT_n^{-1}}$, and $r\ket{CCZ} \Longrightarrow \ket{QFT_n^{-1}}$, and the respective stabilizer nullity values: $\nu (\ket{T}) = 1$, $\nu (\ket{CS}) = 2$ and $\nu (\ket{CCZ}) = 3$, we get the final results above using methods from section \ref{ConvBounds}.
\end{proof}

Note the calculation of $\nu(\ket{QFT_n^{-1}}) = \nu(\ket{QFT_n^{1}}) = n-2$ above also implies that the Quantum Fourier Transform on $n$ qubits can not be performed using fewer than $n-2$ copies of $\ket{T}$.

Now we give lower bounds using the dyadic monotone.


\begin{lemma}
At least $n-2$ $\ket{CCZ}$ states are needed to implement the $n$-qubit modular adder for $n \geq 3$ by using stabilizer operations with measurement probabilities one half.
\end{lemma}
\begin{proof}
Recall that by applying a circuit for modular adder $\ket{+}^{\otimes n} \otimes \ket{QFT_n^1}$ we can create a state $\ket{QFT_n^{-1}} \otimes \ket{QFT_n^1}$. If there existed a circuit that used $k$ $CCZ$ gates for $k < n-2$, we would be able to prepare states $\ket{QFT_n^{-1}}$ starting from $k \ket{CCZ}$ and then using Clifford unitaries and Pauli observable measurements with probability half by using $\ket{QFT_n^1}$ as a catalyst. Let us show that this is impossible. Indeed for the input state we would have value $\mu_2$ equal to $k + \mu_2 \ket{QFT_n^1}$ and for the output state we would have $\mu_2 \ket{QFT_n^{-1}} + \mu_2 \ket{QFT_n^1}$. We know that $\mu_2$ is non-increasing when we apply Clifford unitaries and measurements with probability $1/2$, therefore $k \geq \mu_2 \ket{QFT_n^{-1}} = n - 3 +(1/2)^{n-2}$ which implies that $k \geq n-2$.
\end{proof}

%\section{Conversion Protocols}
%Here we introduce some conversion protocols that use catalysis to convert resource states for the third level of the Clifford hierarchy to resource states for higher levels. First an important building block for these protocols is the following circuit in Figure \ref{one bit adder}.
%
%\vspace{2mm}
%
%\begin{figure}[H]
%
%\caption{Circuits for applying three $\sqrt{T}$ gates using five $T$ gates and one $\sqrt{T}$ gate}
%
%\begin{subfigure}[b]{0.5\textwidth}
%
%
%$
%\small
%\Qcircuit @C=0.07em @R=.3em {
%   \lstick{\ket{a}} & \gate{\sqrt{T}}  & \qw & \rstick{\zeta_{16}^{(a+b+c)_0} \zeta_8^{(a+b+c)_1} \ket{a}} & & & \lstick{\ket{a}} &  \ctrl{2} & \qw & \qw & \qw & \ctrl{3} & \qw & \qw & \qw & \qw & \qw & \qw & \qw & \qw & \qw & \qw & \qw & \qw & \qw & \qw & \qw & \qw & \qw & \qw & \qw & \qw & \qw & \qw & \qw & \qw & \qw & \qw & \qw & \qw & \qw & \qw & \qw & \qw & \qw & \qw & \qw & \qw & \qw & \qw & \qw & \qw & \qw & \qw & \qw & \qw & \qw & \qw & \qw & \qw & \qw & \qw & \qw & \qw & \qw & \qw & \qw & \qw & \qw & \qw & \qw & \qw & \qw & \qw & \qw & \qw & \qw & \qw & \qw & \qw & \qw & \qw & \qw & \qw & \qw & \qw & \qw & \qw & \qw & \qw & \qw & \qw & \qw & \qw & \qw & \qw & \qw & \qw & \qw & \qw  & \qw & \qw & \qw  & \qw & \qw & \qw & \qw & \qw & \qw & \qw & \qw & \ctrl{3} & \qw & \qw & \ctrl{2} & \qw & \rstick{\zeta_{16}^{(a+b+c)_0} \zeta_8^{(a+b+c)_1}\ket{a}} \\
%   \lstick{\ket{b}} & \gate{\sqrt{T}} & \qw & \rstick{\ket{b}} & \push{\rule{9em}{0em}=\rule{1.5em}{0em}} & & \lstick{\ket{b}} & \targ & \qw & \qw & \ctrl{2} & \targ & \ctrl{1} & \qw & \qw & \qw & \qw & \qw & \qw & \qw & \qw & \qw & \qw & \qw & \qw & \qw & \qw & \qw & \qw & \qw & \qw & \qw & \qw & \qw & \qw & \qw & \qw & \qw & \qw & \qw & \qw & \qw & \qw & \qw & \qw & \qw & \qw & \qw & \qw & \qw & \qw & \qw & \qw & \qw & \qw & \qw & \qw & \qw & \qw & \qw & \qw & \qw & \qw & \qw & \qw & \qw & \qw & \qw & \qw & \qw & \qw & \qw & \qw & \qw & \qw & \qw & \qw & \qw & \qw & \qw & \qw & \qw & \qw & \qw & \qw & \qw & \qw & \qw & \qw & \qw & \qw & \qw & \qw & \qw & \qw & \qw & \qw & \qw & \qw & \qw  & \qw & \qw & \qw  & \qw & \qw & \qw & \qw & \qw & \qw & \qw & \ctrl{1} & \targ & \ctrl{2} & \qw & \targ & \qw & \rstick{\ket{b}} \\
%   \lstick{\ket{c}} & \gate{\sqrt{T}} & \qw & \rstick{\ket{c}}  & & & \lstick{\ket{c}} & \targ & \qw  & \qw & \ctrl{1} & \qw & \targ & \qw & \rstick{\ket{a \oplus b \oplus c}} \qw & & & & & & & & & & & & & & & & & & & & & & & & & & & & & & & & & & & & & & & & & & & & & & & & & & & & & & & & & & & & & & & & & & & & & & & & & & & & & \qw & \qw & \qw & \qw & \qw & \qw & \qw & \qw & \qw & \qw & \qw & \qw & \qw & \qw & \qw & \qw & \qw & \qw & \gate{\sqrt{T}} & \targ & \qw  & \ctrl{1} & \qw & \targ & \qw & \rstick{\ket{c}} \\
%   & & & & & & & & & \lstick{\ket{0}} & \gate{X} & \targ & \qw & \qw & \rstick{\ket{ab \oplus ac \oplus bc}} \qw  & & & & & & & & & & & & & & & & & & & & & & & & & & & & & & & & & & & & & & & & & & & & & & & & & & & & & & & & & & &  & & & & & & & & & & & & & & & & & & & & & & & & & & & & & & & & & & & & \gate{T} & \qw & \targ & \gate{X} & \qw & \rstick{\ket{0}}
%}
%$
%\label{one bit adder a}
%\caption{Three $\sqrt{T}$ gates can be applied with just one $\sqrt{T}$ gate and other third level Clifford unitaries}
%\end{subfigure}
%
%
%\vspace{2mm}
%
%\begin{subfigure}[b]{0.5\textwidth}
%\centering
%$
%\Qcircuit @C=0.5em @R=.7em {
%   & \ctrl{2} & \qw & & & & \qw & \qw & \multigate{1}{CZ} & \qw \\
%   & \ctrl{1} & \qw & & \push{\rule{1em}{0em}=\rule{1.5em}{0em}} & & \qw & \qw & \ghost{CZ} & \qw \\
%   & \gate{X} & \qw & \rstick{\ket{0}} & & & \gate{H} & \measureD{Z} & \ctrl{-1} \cw \cwx
%}
%$
%
%\vspace{4mm}
%
%$
%\Qcircuit @C=0.5em @R=.7em {
%   & \lstick{\ket{a}} & \ctrl{2} & \qw & \rstick{\ket{a}} & & & \lstick{\ket{a}} & \ctrl{2} & \qw & \targ & \gate{T^{\dag}} & \targ & \qw & \qw & \qw & \rstick{\ket{a}} \\
%   & \lstick{\ket{b}} & \ctrl{1} & \qw & \rstick{\ket{b}} & & \push{\rule{1em}{0em}=\rule{1.5em}{0em}} & \lstick{\ket{b}} & \qw & \ctrl{1} & \targ & \gate{T^{\dag}} &  \targ & \qw & \qw & \qw & \rstick{\ket{b}} \\
%   & \lstick{\ket{0}} & \gate{X} & \qw & \rstick{\ket{ab}} & & & \lstick{\ket{T}} & \targ & \targ & \ctrl{-2} & \gate{T} & \ctrl{-2} & \gate{H} & \gate{S} & \qw & \rstick{\ket{ab}} 
%}
%$
%\caption{Simplified circuits given specific input states.}
%\label{one bit adder b}
%\end{subfigure}
%
%\label{one bit adder}
%\end{figure}
%
%Note that the gate exp$(i\theta \ket{1} \bra{1})^{\otimes n}$ acting on an n-qubit register in the computational basis state $\ket{w}$ gives $e^{i \theta \cdot hw(w)}\ket{w}$, where hw($w$) is the Hamming weight of the bit string $w$. With this we can compute the gate exp$(i\theta \ket{1}\bra{1})^{\otimes n}$ by just computing hw($w$) and storing it in a quantum register $\ket{x_k \ldots x_0}$, and for $j$ from 0 to $k$ apply exp($i2^j\theta \ket{1} \bra{1}$) to qubit j in the register. In Figure \ref{one bit adder} we use the adder circuit to compute the Hamming weight of the bit string $a, b,$ and $c$. For bit strings of length three the Hamming weight can be represented using two bits. The lower bit is the parity$a \oplus b \oplus c$ and the higher bit is the majority function $ab \oplus bc \oplus ac$. These are exactly the values computed by the adder. An important efficiency gain comes from the first circuit in Figure \ref{one bit adder b} where you can observe that the one qubit adder can be un-computed by using Clifford gates and single qubit Pauli measurements only. Using this trick, Figure \ref{one bit adder a} applies three $\sqrt{T}$ gates using only one $\sqrt{T}$, and either one $T$ and $CCX$ gate each, or five $T$ gates (if using the second circuit in Figure \ref{one bit adder b} instead).
%
%
%
%
%
%We now give an example that uses the above results to form a resource state conversion protocol: $\ket{\sqrt{T}} + 5.5\ket{T} \rightarrow 3\ket{\sqrt{T}}$. This protocol follows directly when $\ket{+}$ states are fed into Figure \ref{one bit adder a} when the $\sqrt{T}$ gate is implemented by injection of $\ket{\sqrt{T}}$, and when the third level Clifford gates are implemented with $\ket{T}$ resource states. We need to use 5$\ket{T}$ states to implement the first $CCX$ gate and the T gate in Figure \ref{one bit adder a}. The $\sqrt{T}$ gate is implemented with the injection circuit which requires an additional $T$ gate correction $50\%$ of the time, which adds $1/2$ to the number of $\ket{T}$ states consumed on average.
%
%The extra $T$ gate can be amortized to give the conversion $k\ket{CCZ} + (k+ \frac{1}{2})\ket{T} \xRightarrow{\ket{\sqrt{T}}} 2k\ket{\sqrt{T}}$ which is valid for any positive integer $k$. We see that asymptotically $\ket{\sqrt{T}}$ state costs half of $\ket{\sqrt{T}}$ state plus half of $\ket{CCZ}$ state. Using the circuit on Figure \ref{one bit adder a} we can reduce the parallel application of $2k+1 \sqrt{T}$ gates to the parallel application of $2k-1 \sqrt{T}$ gates for any positive integer k. We use the circuit on Figure \ref{one bit adder a} to execute the first three out of $2k+1 \sqrt{T}$ gates by only using one $\sqrt{T}$ gate. Then we observe that the rest of the $2k-2 \sqrt{T}$ gates can be executed in parallel with the newly introduced $\sqrt{T}$ gate. The main observation here is that the cost of applying $\sqrt{T}^3$ gate is the same as the cost of applying $\sqrt{T}$. Figure \ref{conversion example} applies this to show how to reduce the parallel application of five $\sqrt{T}$ gates to only three.
%
%\begin{figure}[H]
%\caption{Conversion example of $k\ket{CCZ} + 1\frac{1}{2}\ket{T} + \ket{\sqrt{T}}$ for k=2}
%\label{conversion example}
%$
%\Qcircuit @C=0.5em @R=.7em {
%   & \lstick{\ket{+}} & \gate{\sqrt{T}} & \qw & & & \ctrl{2} & \qw & \qw & \qw & \qw & \ctrl{3} & \qw & \qw & \qw & \qw & \qw & \ctrl{3} & \qw & \qw & \ctrl{2} & \qw & \rstick{\ket{\sqrt{T}}} \\
%   & \lstick{\ket{+}} & \gate{\sqrt{T}} & \qw & & & \targ & \qw & \qw & \qw & \ctrl{2} & \targ & \ctrl{1} & \qw & \qw & \qw & \ctrl{1} & \targ & \ctrl{2} & \qw & \targ & \qw & \rstick{\ket{\sqrt{T}}} \\
%   & \lstick{\ket{+}} & \gate{\sqrt{T}} & \qw & \push{\rule{1em}{0em}=\rule{1em}{0em}} & & \targ & \qw & \qw & \qw & \ctrl{1} & \qw & \targ & \qswap & \gate{T} & \qswap & \targ & \qw & \ctrl{1} & \qw & \targ & \qw & \rstick{\ket{\sqrt{T}}} \\
%   & & & & & & & & & \lstick{\ket{0}} & \gate{X} & \targ & \qw & \qswap \qwx & \gate{\sqrt{T}} & \qswap \qwx & \qw & \targ & \gate{X} & \qw & \rstick{\ket{0}}  \\
%   & \lstick{\ket{+}} & \gate{\sqrt{T}} & \qw & & & \qw & \qw & \qw & \qw & \qw & \qw & \qw & \qw & \gate{\sqrt{T}} & \qw & \qw & \qw & \qw & \qw & \qw & \qw & \rstick{\ket{\sqrt{T}}} \\
%   & \lstick{\ket{+}} & \gate{\sqrt{T}} & \qw & & & \qw & \qw & \qw & \qw & \qw & \qw & \qw & \qw & \gate{\sqrt{T}} & \qw & \qw & \qw & \qw & \qw & \qw & \qw & \rstick{\ket{\sqrt{T}}}
%}
%$
%\end{figure}
%
%\subsection{Conversion Bounds}
%\label{ConvBounds}
%
%
%Furthermore, if $\mathcal{M}(\ket{\psi} \otimes \ket{\phi}) = \mathcal{M}(\ket{\psi}) + \mathcal{M}(\ket{\phi})$ for all $\ket{\psi}$ and $\ket{\phi}$, then we can say even more. First, we can rule out catalyzed conversions since $\mathcal{M}(\ket{A} \otimes \ket{cat.}) < \mathcal{M}(\ket{B})$ implies that $\mathcal{M}(\ket{B} \otimes \ket{cat.})$ for any catalyzing state $\ket{cat.}$. Tensor powers of states simplify, allowing us to make asymptotic implications, i.e., 
%\begin{center}
%$\mathcal{M}(\ket{A}) < \alpha \cdot \mathcal{M}(\ket{B}) \implies \ket{A}^{\otimes n} \not \Longrightarrow \ket{B}^{\otimes \alpha n} \hspace{2mm} \forall n$.
%\end{center}
%
%In other words this would put an upper bound of $\alpha$ on the catalytic rate of conversion from $\ket{A}$ to $\ket{B}$. Note that equivalent implications hold if the monotone is \textit{multiplicative} rather than additive.
%For example, consider the states $\ket{T}$ and $\ket{CCZ}$ for which the best known conversion algorithms are:
%\begin{center}
%$4\ket{T} \rightarrow \ket{CCZ}$
%\end{center}
%\begin{center}
%$\ket{CCZ} \xRightarrow{\ket{T}} 2\ket{T}$
%\end{center}
%Clearly these algorithms would have loss if feeding the output of one into the other. The best possible conversion algorithms have (for any n and any catalyst) the minimum $r$ and maximum $r'$ in 
%\begin{center}
%$rn\ket{T} \Rightarrow n\ket{CCZ}$,
%\end{center}
%\begin{center}
%$n\ket{CCZ} r'n\ket{T}$.
%\end{center}
%
%Now, since the stabilizer nullity and extent are monotone functions, we can use them to find $r$ and $r'$ values. By straightforward computation, we have that $\nu (\ket{T}) = 1$ and $\nu(\ket{CCZ}) = 3$. Since $\nu$ is additive we can use the above equations to deduce that $r \geq 3$ and $r' \leq 3$. While not straightforward, it is also possible to compute the extent values $\xi (\ket{T}) = 1.17157$ and $\xi (\ket{CCZ}) = 1.7778$. Moreover, $\texttt{log} \xi$ is an additive monotone with respect to collections of $\ket{T}$ states and $\ket{CCZ}$ states (which satisfies Lemma \ref{stabextent}) and therefore $r \geq \texttt{log}[1.77778]/\texttt{log}[1.17157] = 3.63356$ and $r' \leq \texttt{log}[1.77778]/\texttt{log}[1.17157] = 3.63356$ (which implies $r' \leq 3$).



%\section{Lower Bounds for approximate unitary synthesis}
%In this section, we lower bound the number of resource states needed to approximate an arbitrary single-qubit unitary using Clifford gates and Pauli measurements. Unlike the previously known lower bounds, our bounds: allow for Pauli measurements; allow measurement outcomes to affect the subsequent parts of the protocol; and do not depend on the number of ancillary qubits used in the protocol.
%There are some subtleties to be addressed when analysing a protocol containing measurements that can affect the operations applied in subsequent parts of the protocol. In particular, the state the protocol outputs and the number of resource states it consumes are random variables, which depend on the sequence of measurement outcomes obtained. The following definition is convenient for formulating lower bounds in this setting.
%
%\begin{definition}
%\label{propertydef}
%Consider a protocol with measurement outcomes that can affect subsequent parts of the protocol. Fixing a sequence of measurement outcomes in the protocol specifies an associated post-selected quantum circuit. Every input state to such a protocol defines a probability distribution on the set of all measurement outcomes and on their associated post-selected quantum circuits. We say that the protocol has some property $P$ with probability at least p if, for all states input to the protocol, a sample drawn from the distribution of post-selected quantum circuits has the property $P$ with probability at least p.
%\end{definition}
%
%Next, we introduce two definitions used in the upcoming theorem. Firstly, we have the diamond norm, which offers a measure of distance between quantum operations. The definition follows by letting $\mathcal{E}:A \mapsto \mathcal{E}(A)$ be a linear map on the space of matrices on $\mathcal{H}$, so we have the diamond norm $||\mathcal{E}||_\diamond := max\{ ||( \mathcal{E} \otimes \mathcal{1}_\mathcal{H})(B)||_1 : ||B||_1 \leq 1 \}$ (where $||A||_1 := \textit{Tr}(\sqrt{A^\dag A})$). Now, let $\mathcal{N}_{\ket{T}}(U, \epsilon)$ be the number of copies of $\ket{T}$ consumed by a stabilizer circuit that approximates a unitary U to within diamond-norm precision $\epsilon$. Moreover, also recall that the fidelity of two states is a measure of "closeness" and expresses the probability that one state can identify as the other state under some fidelity test.
%
%
%\begin{theorem}
%\label{Sec5mainTheorem}
%Consider a protocol  that uses $\mathcal{N}_{\ket{T}}(U, \epsilon)$ copies of the resource state $\ket{\Psi}$ and stabilizer operations to approximate a one-qubit  unitary $U$ to within precision $\epsilon$ (measured by the diamond norm). For any positive $C>1$ and $\epsilon < 1/(2^8C)$ there exists a unitary $U$ such that the following inequalities must hold
%\begin{center}
%$\mathcal{N}_{\ket{T}}(U, \epsilon) \geq \frac{1}{6}log_2(1/\epsilon) - \frac{1}{6}log_2(C) - 1$
%\end{center}
%\begin{center}
%$\mathcal{N}_{\ket{CCZ}}(U, \epsilon) \geq \frac{1}{8}log_2(1/\epsilon) - \frac{1}{8}log_2(C) - \frac{3}{4}$
%\end{center}
%\begin{center}
%$\mathcal{N}_{\ket{CS}}(U, \epsilon) \geq \frac{1}{6}log_2(1/\epsilon) - \frac{1}{6}log_2(C) - 1$
%\end{center}
%with probability at least $(C-1)/C$. In particular, this is the case for all unitaries $U$ such that $2\sqrt{C \epsilon} \leq |\braket{0|U|1}|^2 \leq 6\sqrt{C \epsilon}$
%\end{theorem}
%
%\begin{theorem}
%Consider a post-selected stabilizer circuit with input $\ket{\psi_{in}}$ and output $\ket{\psi_{out}}$, where $\ket{\psi_{in}}$ is defined on no fewer qubits than $\ket{\psi_{out}}$. Then there exists a set of $k = \nu (\ket{\psi_{in}})-\nu (\ket{\psi_{out}})$ independent commuting Pauli operators $P_1, \dots , P_k$ and a Clifford unitary $C$ such that 
%\begin{center}
%$\ket{\psi_{out}} \otimes \ket{S} \propto CM_{P_1} \dots M_{P_k} \ket{\psi_{in}}$,
%\end{center}
%where $\ket{S}$ is a stabilizer state and where $M_P$ is the projector on the eigenspace of $P$.
%\end{theorem}
%
%The following lemma shows that a protocol that has an average output density matrix which is close to a desired state also has, on average, an output density matrix which is close to the desired state on individual runs of the protocol. It will also consider the diamond norm distance between channels, and using the inequalities below, we can give bounds in both directions between the trace distance and the fidelity $\sqrt{\braket{\psi | \rho | \psi}}$. The aforementioned inequalities are called the Fuchs-van de Graaf inequalities:
%\begin{equation}
%\sqrt{\braket{\psi | \rho | \psi}} \geq 1 -\frac{1}{2}||\ket{\psi} \bra{\psi} - \rho ||_1
%\end{equation}
%
%\begin{equation}
%\label{Fuchs2}
%||\ket{\psi} \bra{\psi} - \rho ||_1 \leq 2\sqrt{1 - \braket{\psi | \rho | \psi}}
%\end{equation}
%
%\begin{lemma}
%\label{lemma5.4}
%Consider a protocol that, when averaged over measurement outcomes, produces a density matrix $\rho$ that has fidelity $\braket{\psi | \rho | \psi}$ at least $1 - \delta$ with a pure state $\ket{\psi}$. Then, for any $C > 1$, with probability at least $(C-1)/C$ the fidelity between $\ket{\psi}$ and the protocol's output is at least $1 - C\delta$ following the convention of Definition \ref{propertydef}.
%\end{lemma}
%
%\begin{proof}
%Suppose the protocol has $N$ possible sequences of measurement outcomes. Let $p_k$ be the probability of the $k^{th}$ sequence of measurement outcomes occurring, and let $\rho_k$ be the normalized density matrix of the output register for that sequence.
%For fixed $C > 1$ we split the set of all fixed sequences of measurement outcomes into two subsets, $S$ and its complement $\overline{S}$. Let the set $S$ contain sequences that output good approximations of $\ket{\psi}$ such that for $k \in S$, $\braket{\psi | \rho_k | \psi} \geq 1 - C \delta$, and let $\overline{S}$ contain sequences that output worse approximations, such that for $k \in \overline{S}$, $\braket{\psi | \rho_k | \psi} < 1 - C \delta$. Since, by assumption, the overall average output $\rho$ has fidelity at least $1 - \delta$ with $\ket{\psi}$, the probability $p_S$ of all outcomes leading to a good approximations can not be small. More explicitly, let $\rho_S$ and $\rho_{\overline{S}}$ be the normalized density matrices corresponding to averaging over the subsets $S$ and $\overline{S}$ respectively:
%\begin{center}
%$\rho_S \propto \sum\limits_{k \in S} p_k \rho_k$ and $\rho_{\overline{S}} \propto \sum\limits_{k \in \overline{S}} p_k \rho_k$
%\end{center}
%The density matrix of the output is then $\rho = p_S \rho_S + (1 - p_S)p_{\overline{S}}$. From above, $\braket{\psi | \rho | \psi} < 1 - C \delta$, so we have the following inequalities:
%\begin{center}
%$1 - \delta \leq \braket{\psi | \rho | \psi} = p_s \braket{\psi | \rho_S | \psi} + (1 - p_S) \braket{\psi | \rho_{\overline{S}} | \psi} \leq p_S + (1 - p_S)(1 - C \delta)$
%\end{center}
%Note that we make use of the fact that the fidelity is always a value between $0$ and $1$, as a consequence of Uhlmann's theorem. Then by solving $1 - C \delta \leq p_S + (1 - p_S)(1 - C \delta)$ you get the desired lower bound: $p_S \geq \frac{C-1}{C}$.
%
%Now, using the equation \ref{Fuchs2}, we have that 
%\begin{equation}
%\begin{split}
%||\ket{\psi} \bra{\psi} - \rho ||_1 & \leq 2\sqrt{1 - (p_s \braket{\psi | \rho_S | \psi} + (1 - p_S) \braket{\psi | \rho_{\overline{S}} | \psi})} \\
%& = 2\sqrt{1 - p_s \braket{\psi | \rho_S | \psi} - (1 - p_S) \braket{\psi | \rho_{\overline{S}} | \psi}} \\
%& \leq 2\sqrt{1 + p_s(C\delta - 1) - (1 - p_S)} \\
%& = 2\sqrt{1 + p_sC\delta - p_s - 1 + p_S} \\
%& = 2\sqrt{p_sC\delta} \\
%& \leq 2\sqrt{C\delta}
%\end{split}
%\end{equation}
%This gives an upper bound on the trace distance.
%\end{proof}
%
%
%
%
%\begin{lemma}
%\label{lemma5.5}
%Consider a protocol that uses $\mathcal{N}_{\ket{\Psi}}(U, \epsilon)$ copies of the resource state $\ket{\Psi}$ and stabilizer operations to approximate a one-qubit unitary $U$ to within precision $\epsilon$ (measured by the diamond norm). For any $C>1$, let $N$ be the minimum number of copies of a resource state $\ket{\Psi}$  needed to approximate the state $\ket{\psi} = U\ket{1}$ to trace distance $2\sqrt{C\epsilon}$ with any protocol composed of stabilizer operations and post-selection. Then $\mathcal{N}_{\ket{\Psi}}(u, \epsilon) \geq N$ with probability at least $(C - 1)/C$, following the convention of Definition \ref{propertydef}.
%\end{lemma}
%
%\begin{lemma}
%\label{lemma5.6}
%Let $N_{\ket{CS}}(\ket{\psi}, \epsilon)$ be the minimum number of $\ket{CS}$ resource states required to approximate the one-qubit state $\ket{\psi}$ to within trace distance $\epsilon$ using stabilizer operations and post-selection. When $\epsilon < 1/8$, there exists a state $ket{\psi}$ such that $N_{\ket{CS}}(\ket{\psi}, \epsilon) \geq 1/3 \cdot log_2(1/\epsilon) - 2/3$. For example, this is the case for all states such that $\epsilon <|\braket{\psi | 0}|^2 < 3\epsilon$.
%\end{lemma}
%
%To generalize the above lemma from $\ket{CS}$ states to an arbitrary $k$-qubit resource state $\ket{\Psi}$, we must establish a lower bound on the quantity:
%\begin{equation}
%\frac{1}{2^m} \sum\limits_{\langle P \in P_1, \dots , P_m \rangle} \braket{\Psi|^{\otimes n} P|\Psi}^{\otimes n}
%\end{equation}
%where $\{ P_1, \dots , P_m \}$ are independent commuting Pauli operators and $m \leq k \cdot n$. Note that replacing $p$ with one in the statement of the lemma leads to a slightly weaker lower bound that does not require the knowledge of $p$. For example, it is not too difficult to generalize the above result to use $\ket{CCZ}$ states in place of $\ket{CS}$ states, because their Pauli expectations also take values $0$, $1$, and $\pm 1/2$. We then get the following lemma.
%
%\begin{lemma}
%\label{lemma5.7}
%Let $N_{\ket{CCZ}}(\ket{\psi}, \epsilon)$ be the minimum number of $\ket{CCZ}$ resource states required to approximate the one-qubit state $\ket{\psi}$ to within trace distance $\epsilon$ using stabilizer operations and post-selection probability $p$. When $\epsilon < 1/8$, there exists a state $ket{\psi}$ such that $N_{\ket{CCZ}}(\ket{\psi}, \epsilon) \geq 1/4 \cdot log_2(1/\epsilon) - 1/2$. 
%\end{lemma}
%
%\begin{proposition}
%Let $\{ P_1, \dots , P_k \}$ be independent commuting Pauli operators and let the probability of measuring the $+1$ eigenvalue of each be
%\begin{equation}
%\label{pvalue}
%p = 1/2^k \sum\limits_{\langle P \in P_1, \dots , P_m \rangle}  \braket{T|^{\otimes n} P|T}^{\otimes n}
%\end{equation}
%If the value of p is non-zero, then $p \geq \frac{1}{2^{2k+n}}$.
%\end{proposition}
%
%Now consider the following set:
%\begin{center}
%$\mathcal{R} = \Big\{ \frac{a+bi+\sqrt{2}(c+di)}{2^j} : \texttt{for } a, b, c, d, j \texttt{ integers} \Big\}$
%\end{center}
%Note that the set $\mathcal{R}$ is closed under addition, negation and multiplication, and contains $0$ and $1$ (and is even closed under complex conjugation). Thus, the set $\mathcal{R}$ is a ring.
%Note that the state $\ket{T}$ and all Pauli operators can be written with entries in $\mathcal{R}$, and hence, so can $p$ as defined in \ref{pvalue}. 
%
%\begin{lemma}
%\label{lemma5.9}
%Let $N_{\ket{T}}(\ket{\psi} , \epsilon)$ be the minimum number of $\ket{T}$ resource states required to approximate the one-qubit state $\ket{\psi}$ to within trace distance $\epsilon$ using stabilizer operations. When $\epsilon < 1/8$, there exists a state $\ket{\psi}$ such that $N_{\ket{T}}(\ket{\psi} , \epsilon) \geq 1/3 \cdot \texttt{log}_2(1/\epsilon) - 2/3$. For example, this is the case for all states such that $\epsilon < |\braket{\psi | 0}|^2 < 3\epsilon$.
%\end{lemma}
%
%\begin{proof}[Proof of theorem \ref{Sec5mainTheorem}]
%First note that setting $p=1$ on the right hand side of the inequalities in Lemma's   \ref{lemma5.6}, \ref{lemma5.7}, and \ref{lemma5.9} form new (yet weaker) inequalities which hold for all $p$. Then apply Lemma \ref{lemma5.5} to each of these inequalities.
%\end{proof}
%
%
%
%
%
%
%
%%\subsection{Lower Bounds for Resource State Conversion}
%In other sections, we have introduced protocols for catalysis assisted conversion of $\ket{CCZ}$ states into states $\ket{\pi j /2^d}$. We have found that for odd $j$ and integer $d \geq 2$, asymptotically, one can create one $\ket{\pi j /2^d}$ state at the cost of $1 - 1/2^{d-1}$ $\ket{CCZ}$ states. Using the dyadic monotone we can show that this is optimal when only Pauli measurements with probability $50 \% $ are allowed.
%
%
%\begin{lemma}
%Consider a protocol that uses stabilizer operations with measurements probability $50 \% $, $\ket{CCZ}$ states and a multi-qubit state as a catalyst. The catalyst has entries in $\mathcal{R}_{d'}$ for some $d'$. Suppose that such a circuit uses $k$ $\ket{CCZ}$ states and produces $n$ states $\ket{\pi j/2^d}$ for odd $j$ and integer $d \geq 2$, then $k \geq n(1-1/2^{d-1})$. Asymptotically, at least $1 - 1/2^{d-1}$ copies of $\ket{CCZ}$ state are needed to produce state $\ket{\pi j/2^d}$.
%\end{lemma}
%\begin{proof}
%Let $\ket{cat}$ be a state used as a catalyst, then $\mu_2$ for the input of our protocol is $\mu_2 \ket{cat} + k$ and for the output the value of $\mu_2$ is $n(1-1/2^{d-1}) + \mu_2\ket{cat}$. This is because for odd $j$ and integer $d \geq 2$, $\mu_2 \ket{\pi j/2^d} = 1 - 1/2^{d-1}$. Above implies that $k \geq n(1 - 1/2^d)$.
%\end{proof}




%\chapter{Main Points of Section 6}
%The section is based around the dyadic monotone below. For this monotone function, we only consider states that have entries in the ring $\mathbb{Z}[i, 1/2] = \big\{\frac{a+ib}{2^k}: a, b, k, \in \mathbb{Z} \big\}$, and note that Pauli's and Clifford unitaries take states with entries in this ring too states with entries still in this ring.
%
%
%
%\begin{definition}
%Let $\ket{\psi}$ be an $n$-qubit state with entries in $\mathbb{Z}[i, 1/2]$, the dyadic monotone is 
%\begin{center}
%$\mu_2 \ket{\psi} = \texttt{max} \big\{-v_2(\braket{\psi | P | \psi}): P \in \{I, X, Y, Z \}^{\otimes n} \big\}$.
%\end{center}
%\end{definition}
%
%
%
%$v_2$ is defined as follows: let $q$ be a non-zero rational number and write it as:
%\begin{center}
%$q = \pm 2^k \cdot p_1^{k(1)} \cdots p_m^{k(m)}$, $p_k$ are odd primes, and $k, k(1), \dots , k(m)$ are integers
%\end{center}
%We let $v_2 (q) = k$. It also has the two properties: $v_2(a \cdot b) = v_2(a) + v_2(b)$ and $v_2(a + b) \geq min(v_2(a), v_2(b))$.
%
%
%\vspace{5mm}
%
%Properties of $\mu_2$ are:
%\begin{itemize}
%\item $\mu_2 (\ket{\phi} \otimes \ket{\psi}) = \mu_2 (\ket{\phi}) + \mu_2 (\ket{\psi})$
%\item Let $\ket{\psi}$ be a state in $\mathbb{Z}[i, 1/2]$, then $\mu_2 \ket{\psi} \geq 0$ with equality achieved if and only if $\ket{\psi}$ is a stabilizer state.
%\item it is non-increasing (under Pauli measurements with probability of $1/2$)
%\end{itemize}
%
%\vspace{5mm}
%
%We need to extend the ring that $\mu_2$ is defined on to:
%\begin{center}
%$\mathcal{R}_d = \mathbb{Z} \big[ exp(i\pi /2^d), 1/2 \big] = \Big\{ \frac{1}{2^k} \sum\limits_{j=0}^{2^d - 1} a_j \texttt{exp}(i\pi j/2^d): \texttt{where } a_j, k \texttt{ are integers} \Big\}$
%\end{center}
%Note that all of its previously mentioned properties still hold (and also for $v_2$).
%
%\vspace{5mm}
%
%Now, a result regarding the Quantum Fourier Transform:
%\begin{center}
%$\ket{QFT_n^a} = \sum\limits_{y=o}^{2^n-1} \texttt{exp}\Big[ \frac{i2\pi ay}{2^n} \Big] \ket{y} = \otimes_{k=1}^n \big( \ket{0} + e^{i2\pi a/2^k}\ket{1}\big)$,
%\end{center}
%For all odd $a$, $\mu_2\ket{QFT_n^a} = n-3 +(1/2)^{n-2}$.
%
%\vspace{5mm}
%
%The following are three possibly relevant results/theorems:
%
%\begin{enumerate}
%\item At least $n-2$ $\ket{CCZ}$ states are needed to implement the $n$-qubit multiply controlled $Z$ gate $C^{n-1}Z$ by using stabilizer operations with measurement probabilities one half.
%
%\item At least $n-2$ $\ket{CCZ}$ states are needed to implement the $n$-qubit modular adder for $n \geq 3$ by using stabilizer operations with measurement probabilities one half.
%
%
%\item Consider a protocol that uses stabilizer operations with measurements probability $50 \% $, $\ket{CCZ}$ states and a multi-qubit state as a catalyst. The catalyst has entries in $\mathcal{R}_{d'}$ for some $d'$. Suppose that such a circuit uses $k$ $\ket{CCZ}$ states and produces $n$ states $\ket{\pi j/2^d}$ for odd $j$ and integer $d \geq 2$, then $k \geq n(1-1/2^{d-1})$. Asymptotically, at least $1 - 1/2^{d-1}$ copies of $\ket{CCZ}$ state are needed to produce state $\ket{\pi j/2^d}$.
%\end{enumerate}

\chapter{Conclusion}
\cite{Aaronson_2004}
\cite{beverland2019lower}

Did it!

\bibliographystyle{plain}
\bibliography{thesis}

\end{document}

% You may ignore or delete these two lines of comments.
% $Id: simple.tex 386 2012-11-12 15:11:16Z vlado $
